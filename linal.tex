\documentclass[a4paper, 12pt]{article}

\usepackage{import}

% Корректность отображения всех шрифтов, кодировок и мат. символов
\usepackage[T2A]{fontenc}
\usepackage[utf8]{inputenc}
\usepackage[english, russian]{babel}
\usepackage{amssymb, amsmath, amsthm, amscd, amstext, mathtools}

% Отображение содержания
\usepackage{tocloft}

% Вставка картинок
\usepackage{graphicx}
\usepackage{tikz}
\usepackage{tkz-euclide}
\usepackage{asymptote}


\usepackage{wrapfig}        % Огибание картинок текстом
\usepackage{cancel}         % Зачёркивания
\usepackage{indentfirst}    % Отступ у первого абзаца
\usepackage{xcolor}         % Цвета
\setlength{\parskip}{.5ex}  % Отступы между абзацами
\usepackage{enumitem}       % Работа со списками
% \usepackage{minted}       % Вставка блоков кода

\usepackage{hyperref}       % гиперссылки
\definecolor{linkcolor}{HTML}{225ae2} % Цвет ссылок
\definecolor{urlcolor}{HTML}{225ae2} % Цвет гиперссылок
\hypersetup{
    pdfstartview=FitH, 
    linkcolor=linkcolor,
    urlcolor=urlcolor,
    colorlinks=true}
\setlength{\arrayrulewidth}{0.5mm} %Толщина линейки в таблицах
\setlength{\tabcolsep}{18pt} %Разделение между столбцами в таблице

% Отступы на странице
\usepackage[left=2cm, right=1.5cm, top=2cm, bottom=2cm]{geometry}

\usepackage{cmap}            % Русский поиск в PDF документе
\usepackage{etoolbox}
\usepackage{soul}            % Разряженный текст \so{} и подчеркивание \ul{}
\usepackage{soulutf8}        % Поддержка UTF8 в soul

\usepackage{titlesec}        % Форматирование заголовков
\titleformat{\section}{\LARGE \bfseries}{\thesection}{1em}{}
\titleformat{\subsection}{\Large\bfseries}{\thesubsection}{1em}{}
\titleformat{\subsubsection}{\large\bfseries}{\thesubsubsection}{1em}{}


\newcommand{\R}{\mathbb R}
\newcommand{\Q}{\mathbb Q}
\newcommand{\Z}{\mathbb Z}
\newcommand{\N}{\mathbb N}
\newcommand{\CC}{\mathbb C}
\newcommand{\aug}{\fboxsep=-\fboxrule\!\!\!\fbox{\strut}\!\!\!}
\newcommand{\sgn}{\operatorname{sgn}}
\newcommand{\id}{\mathrm{id}}
\renewcommand{\phi}{\varphi}
\renewcommand{\epsilon}{\varepsilon}

\newsavebox{\boxedalignbox}
\newenvironment{boxedalign*}
  {\begin{equation*}\begin{lrbox}{\boxedalignbox}$\begin{aligned}}
  {\end{aligned}$\end{lrbox}\fbox{\usebox{\boxedalignbox}}\end{equation*}}

\newcommand\tab[1][.5cm]{\hspace*{#1}}

% Подписи для матриц
\newcommand\undermat[2]{\makebox[0pt][l]{$\smash{\underbrace
{\phantom{\begin{matrix}#2\end{matrix}}}_{\text{$#1$}}}$}#2}
\newcommand\overmat[2]{\makebox[0pt][l]{$\smash{\overbrace
{\phantom{\begin{matrix}#2\end{matrix}}}^{\text{$#1$}}}$}#2}

% Значек "пусть"
\newlength{\tempheight}  
\newcommand{\Let}[0]{  
\mathbin{\text{\settoheight{\tempheight}{\mathstrut}\raisebox{0.5\pgflinewidth}{%
\tikz[baseline,line cap=round,line join=round] \draw (0,0) --++ (0.4em,0) --++ (0,1.5ex) --++ (-0.4em,0);
}}}}


% \newcounter{lemcount}
% \newcounter{thcount}
% \newcounter{offercount}
% \newcounter{concount}
% \newcounter{subthcount}
% \newcounter{defcount}

\theoremstyle{definition}
\newtheorem{definition}{Определение}
% \newtheorem{definitionnum}[defcount]{Определение}
\newtheorem*{example}{Примеры}
\newtheorem*{example1}{Пример}
\newtheorem*{exercise}{Упражнение}


\theoremstyle{plain}
\newtheorem*{theorem}{Теорема}
% \newtheorem{theoremnum}[thcount]{Теорема}
\newtheorem*{consequense}{Следствие}
\newtheorem*{consequenses}{Следствия}
% \newtheorem{consequensenum}[concount]{Следствие}
\newtheorem*{lemma}{Лемма}
% \newtheorem{lemmanum}[lemcount]{Лемма}
\newtheorem*{subtheorem}{Утверждение}
% \newtheorem{subtheoremnum}[subthcount]{Утверждение}
\newtheorem*{algorithm}{Алгоритм}
\newtheorem*{properties}{Свойства}
\newtheorem*{properties1}{Свойство}


\theoremstyle{remark}
\newtheorem*{remark}{Замечание}
\newtheorem*{offer}{Предложение}
% \newtheorem{offernum}[offercount]{Предложение}


\begin{document}
  \begin{titlepage}
    \newpage
    \begin{center}
    \includegraphics[width=4cm]{image/image.png}
    \end{center}
    \vspace{4em}
    
    \begin{center}
    \Large Механико-математический факультет  
    \end{center}
    \vspace{2em}
    
    \begin{center}
    \large{\textsc{\textbf{Линейная алгебра и геометрия, 2 семестр, 2 поток}}}
    \end{center}
    \vspace{6em}
    
    \newbox{\lbox}
    \savebox{\lbox}{\hbox{Молчанов Вячеслав Вадимович}}
    \newlength{\maxl}
    \setlength{\maxl}{\wd\lbox}
    \hfill\parbox{11cm}
    {
    Преподаватель:\hfill\hbox to\maxl{Чубаров Игорь Андреевич\\}\vspace{0.5cm}
    
    Студент:\hfill\hbox to\maxl{Молчанов Вячеслав\\}\vspace{0.5cm}

    Группа:\hfill\hbox to\maxl{108\\}\vspace{0.5cm}
    
    Контакт:\hfill\hbox to\maxl {\href{https://t.me/Slavikvaxye}{Мой телеграм для связи\\}}\vspace{0.5cm}
    }

    \vspace{\fill}
    \begin{center}
    Москва \\Последняя компиляция: \today
    \end{center}
    
  \end{titlepage}
  \tableofcontents
  \fontsize{14pt}{20pt}\selectfont
  \newpage
  \fontsize{14pt}{20pt}\selectfont

%   \import{Sections/}{section1.tex}
  \section{Векторное пространство}
  \begin{definition}
    Множество $V$ называется \textit{векторным пространством} над полем $F$, если заданы операции $"+"$ и $"\cdot"$\ : \ $V\times V \to V$, $F \times V \to V$ и выполнены следующие аксиомы:
    \begin{enumerate}
        \item $\forall v_1, v_2, v_3\in V$ \ : \ $(v_1 + v_2) + v_3 = v_1 + (v_2 + v_3)$
        \item $\exists \ \vec 0 \in V:\ \forall v \in V$ \ : \ $v + \vec 0 = v$
        \item $\forall v \in V \ \ \exists  -v  \in V$: $v + (-v) = \vec 0$
        \item $\forall v_1, v_2 \in V$ \ : \ $v_1 + v_2 = v_2 + v_1$
        \item $\forall \alpha, \beta \in F, v \in V$ \ : \ $(\alpha \beta)v = \alpha (\beta v)$
        \item $\forall v \in V$ $\exists \ 1 \in F$ \ : \ $1 \cdot v = v$
        \item $\forall \alpha, \beta \in F, v \in V$ \ : \ $(\alpha + \beta)v = \alpha v + \beta v$
        \item $\forall \alpha \in F, v_1, v_2 \in V$ \ : \ $\alpha (v_1 + v_2) = \alpha v_1 + \alpha v_2$
    \end{enumerate}
    \textit{Загадка:} Одна из этих аксиом - следствие других. Какая?
  \end{definition}
  \begin{definition}
    $U \subset  V$ - \textit{векторное подпространство} пространства $V$, если оно само является пространством относительно тех же операций в $V$. 
  \end{definition}
  \begin{subtheorem}
    Определение 2 эквивалентно:
    \begin{enumerate}
      \item $\forall U\neq \emptyset $
      \item $\forall u_1, u_2 \in U$ \ : \  $u_1 + u_2 \in U$
      \item $\forall u \in U, \ \lambda \in F$ \ : \ $\lambda u\in U$
    \end{enumerate}
  \end{subtheorem} 
  \begin{definition}
    Векторы $v_1,...,v_n \in V$ называются линойно независимыми, если $\exists \ \lambda_1,..., \lambda_n$ (не все равные 0) \ : \ $\lambda_1v_1+...+\lambda_nv_n = \vec 0$
  \end{definition} 
  \begin{subtheorem}
    Определение 3 $\Longleftrightarrow $ ($m\geq 2$) хотя бы один вектор из векторов $v_i$ выражается как линейная комбинация остальных. 
  \end{subtheorem}
  \begin{definition}
    \textit{Упорядоченный набор векторов} $e = (e_1,...,e_n), e_k \in V$, если это максимальный ЛНЗ набор веторов из $V$.  
  \end{definition} 
  \begin{subtheorem}
    $e$ - базис в $V \Longleftrightarrow$
    \begin{enumerate}
      \item $e_1,...,e_n$ - ЛНЗ
      \item $\forall x \in V \ \exists \ x_1,...,x_n \in F \ : \ x = x_1e_1+...+x_ne_n = \sum \limits_{i=1}^nx_ie_i $ 
    \end{enumerate}
  \end{subtheorem} 
  \begin{consequense}
    Разложение любого вектора в базисе единственно.
  \end{consequense} 
  \begin{proof}
    Если $x = \sum \limits_{i=1}^nx_ie_i = \sum \limits_{i=1}^nx'_ie_i$, то $\vec 0 = x - x = \sum \limits_{i=1}^n(x'_i-x_i)e_i$\\
    Из ЛНЗ все коэффициенты равны 
  \end{proof} 
  Обозначаем: $X_e = \begin{pmatrix}
    x_1\\ \vdots\\ x_n
  \end{pmatrix} \in F^n$, тогда $x = eX_e = e_1x_1+...+e_nx_n$  
  \begin{equation}
    \fbox{$x = eX_e$}
  \end{equation}
  \begin{theorem}
    Если в $V \ \exists$ базис из $k$ векторов, то любой базис $V$ содержит $k$ векторов.    
  \end{theorem}
  \begin{proof} $\\$ 
    Если $\exists$ базис $e'_1,...,e'_m \in V$, где $m>n$, то по ОЛЛЗ $e'_1,...,e'_m$ - ЛЗ, т.е. не базис.\\
    Если же $m<n$, то по ОЛЛЗ (в другую сторону) $e_1,...,e_n$ - ЛЗ $\Longrightarrow$ не базис.       
  \end{proof}
  \begin{properties} матриц перехода
    \begin{enumerate}
      \item $\det C \neq 0$
      \item $C_{e' \to e} = (C_{e \to e'})^{-1}$
      \item $C_{e'' \to e} = C_{e \to e'} \cdot C_{e' \to e''}$
    \end{enumerate}
  \end{properties}
  \begin{proof}\tab
    \begin{itemize}
      \item[$1)$] Столбцы - координаты ЛНЗ векторов $e'_1,...,e'_n \Longrightarrow rkC = n \Longrightarrow \det C \neq 0$
      \item[$2)$] Перепишем определение матрицы перехода в матричный вид. \\
      По определению: 
      $$e'=(e'_1,...,e'_n) = (e_1,...,e_n)C_{e \to e'}, \text{ т.е. } e' = eC_{e \to e'}$$
      \begin{equation}
        \fbox{$e' = eC_{e \to e'}$}
      \end{equation}
      С другой стороны 
      $$C = e'C_{e' \to e} = eC_{e \to e'}C_{e' \to e} \Longrightarrow C_{e \to e'}C_{e' \to e} = E$$ 
      ввиду единственности разложения векторов по базису, т.е. 
      $$C_{e \to e'} = (C_{e' \to e})^{-1}$$
      \item[$3)$] $$e'' = e'C_{e' \to e''} = e(C_{e \to e'}C_{e' \to e''}) = eC_{e \to e''}$$
      В силу единственности разложения $C_{e \to e''} = C_{e \to e'}C_{e' \to e''}$     
    \end{itemize}
  \end{proof} 
  \begin{algorithm}
    Как вычислить матрицу перехода, если известны координаты векторов $e_i$ и $e'_j$ в некотором универсальном?\\
    $e' = eC_{e \to e'}$ можно рассмотреть как матричное уравнение:
    $$(e_1^{\uparrow},...,e_n^{\uparrow})C = ({e_1'}^{\uparrow},...,{e_n'}^{\uparrow})$$
    $$[e_1^{\uparrow},...,e_n^{\uparrow} \ | \ {e_1'}^{\uparrow},...,{e_n'}^{\uparrow}] \overset{\text{строк}}{\rightsquigarrow} [E \ | \ C_{e \to e'}]$$   
  \end{algorithm}
  \subsection{Изменение координат вектора при замене базиса}
  \begin{theorem}
    Формула изменения координат вектора при замене базиса:
    \begin{equation}
      \fbox{$X_e = C_{e \to e'}X_{e'}$} 
    \end{equation}
  \end{theorem} 
  \begin{proof}
    $$\forall x \in V \ : \ x = eX_e = e'X_{e'} = eC_{e \to e'}X_{e'}$$
    $$\Longrightarrow  X_e = C_{e \to e'}X_{e'}$$ 
  \end{proof}
  \section{Векторные подпространства}
  \begin{example} \tab
    \begin{enumerate}
      \item Геометрические вектроры
      \item $F^n$ - пространство слобцов (строк) высоты (длины) $n$ с естественными оперицаями $(+, \cdot \lambda)$ \vspace{0.4cm}\\
      Базис $\vartheta  = \Bigg\{ \begin{pmatrix}
        1 \\ 0 \\ \vdots \\ 0
      \end{pmatrix}, \begin{pmatrix}
        0 \\ 1 \\ \vdots \\ 0
      \end{pmatrix}, ... , \begin{pmatrix}
        0 \\ 0 \\ \vdots \\ 1
      \end{pmatrix} \Bigg\}$ (можно взять столбцы любой\vspace{0.3cm}\\ невырожденной матрицы порядка $n$)
      \begin{remark}
        Доказать, что если $e$ - базис, $C$ - невырожденная матрица,\\ то $eC$ - тоже базис (из \textbf{(2)})
      \end{remark} 
      \begin{exercise}
        Пусть $|F| = q, \ \dim_F V = n \Longrightarrow |V| = q^n$\\
        $\dim M_{m,n} = mn$, стандартный базис - $\{E_{ij}\}$, где $E_{ij}$ содержит 1 на $ij$-ой позиции и $0$ на остальных.  
      \end{exercise}
      \item $V = \{F:\underset{(X\subseteq \R)}{X} \to \R\}$ с операциями сложения и $\lambda F$\\
      Оно бесконечномерно, если $X$ бесконечно.\\
      Если $\lambda_1, ..., \lambda_n$ - попарно различные числа, то $y_1 = e^{\lambda_1x},..., y_n = e^{\lambda_nx}$ ЛНЗ\\
      Допустим, что:
      $$\begin{cases}
        C_1y_1 + ... + C_ny_n \equiv 0\\
        C_1y'_1 + ... + C_ny'_n \equiv 0\\
        \vdots\\
        C_1y_1^{(n-1)} + ... + C_ny_n^{(n-1)} \equiv 0
      \end{cases} \Longrightarrow \begin{cases}
        C_1e^{\lambda_1x} + ... + C_ne^{\lambda_nx} \equiv 0\\
        \lambda_1C_1y'_1 + ... + \lambda_nC_ny'_n \equiv 0\\
        \vdots\\
        \lambda^{n-1}C_1e^{\lambda_1x} + ... + \lambda^{n-1}C_ne^{\lambda_nx} \equiv 0
      \end{cases}$$
      $$\Delta = V(\lambda_1,...,\lambda_n) \neq 0 \Longrightarrow C_1 = ... = C_n = 0$$
      \item $F[t]$ с естественными операциями сложением и умножением на скаляр - бесконечномерное пространство, т.к.: $\forall n \in N_0: \ 1, t, t^2,...$ линейно независимы.\\
      $F[t]_n = \{a_0+a_1t+a_2t^2+...+a_nt^n \ | \ a_k\in F, \ k=0,...,n\} $ - подпространство, $n \in N_0, \ \dim U = n+1$, базис: $1,t,...,t^n$\\
      Тейлоровский базис: $1, t-t_0,...,(t-t_0)^n$ \ $\sum \limits_{k=0}^n\frac{f^{(k)}(t_0)}{k!}(t-t_0)^k$ 
      \item $\varOmega \neq 0$, \ $V = 2^\varOmega $ с операциями вместо сложения: \\
      $A\vartriangle B = (A\cap \overline{B}) \cup (B\cap \overline{A}) \ \forall A,B \subseteq \varOmega$\\
      $F = \Z_2,  \ 0\cdot A = \emptyset , \ 1 \cdot A = A$\\
      Задача доказать $V$ - векторное пространство над $\Z_2$  
    \end{enumerate}
  \end{example}
  \subsection*{Два основных способа задания подпространства в $V$}
  \begin{enumerate}
    \item Линейная оболочна семейства векторов $S\subset V$:
    $$\langle S \rangle = \{\sum \limits_{i\in I}\lambda_iS_i \text{ (канонические суммы)} \ | \ S_i \in S, \lambda_i\in F\}$$
    Частный случай: 
    $$\langle a_1,...,a_n \rangle = \{\sum \limits_{i=1}^n \lambda_i a_i \ | \ \lambda_i \in F\} = U$$
    \begin{subtheorem} $\\$ 
      $\langle a_1,...,a_n \rangle$ - подпространство в $V$, $\dim \langle a_1,...,a_n \rangle = rk \{a_1,...,a_n\}$
    \end{subtheorem} 
    \begin{proof}
      $$\mu \sum \limits_{i=1}^n \lambda_i a_i = \sum \limits_{i=1}^n (\mu \lambda_i)a_i$$
      $$\sum \limits_{i=1}^n \mu_i a_i + \sum \limits_{i=1}^n \lambda_ia_i = \sum \limits_{i=1}^n(\mu_i + \lambda_i)a_i \in U$$
      Если $r = rk \langle a_1,...,a_n \rangle$, то $a_{i1},...,a_{ir}$ - базисные, то $\forall a_i$ через них тоже выражается 
      $$  \forall \sum \limits_{i=1}^n \lambda_ia_i \Longrightarrow \{a_{i1},...,a_{ir}\} - \text{базис } U$$
    \end{proof}
    Алгоритм вычисления $\dim \langle a_{i1},...,a_{ir} \rangle$ и базиса, если известны координаты этих векторов: \\
    Составить матрицу: $$(a_1^{\uparrow},...,a_n^{\uparrow}) \rightsquigarrow \begin{pmatrix}
      1 & \cdots & 0 & \vline & \null \\
      \vdots & \null & \vdots & \vline & 0 \\
      0 & \cdots & 1 & \vline & \null \\ \hline
      \null & 0 & \null & \vline & 0
    \end{pmatrix}$$ 
    Cтолбцы с номерами $\nu_1,...,\nu_r$ - базис в $U$, разложение оставшихся векторов можно сразу считать из преобразованной матрицы    
    \item ($\dim V = n$, известны координаты в некотором базисе)
    $$\forall \sum \limits_{i=1}^n X_ie_i = eX , \ X = \begin{pmatrix}
    X_1 \\ \vdots \\ X_n
    \end{pmatrix}$$
    $$W = \{x = eX \ | \ AX = 0\} - \text{ задание с помощью ОСЛУ}$$   
    \begin{subtheorem}
      $W$ - подпространство в $V$,  \ $\dim W = n - rkA$, \ базис - любая ФСР. Это переход от 2. к 1. способу задания подпространства   
    \end{subtheorem} 
  \end{enumerate}
  \begin{theorem}
    Линейную оболочку конечного числа векторов в конечномерном векторном пространстве $V$ можно задать с помощью ОСЛУ. Два способа:
  \end{theorem}
  \begin{proof}
    \begin{enumerate}
      \item Вектор $x \ ($со столбцами координат $X = \begin{pmatrix}
        X_1 \\ \vdots \\ X_n
      \end{pmatrix})$: $x \in \langle a_1,...,a_m \rangle = U$ 
      $$ \Longleftrightarrow \exists \ \alpha_1,...,\alpha_m \in F :  \sum \limits_{i=1}^m \alpha_i a_i = x,  \text{ или в координатах: } \sum \limits_{i=1}^m \alpha_1a_i^{\uparrow} = X$$
      Т.е. СЛУ с $\widetilde{A} = (a_1^{\uparrow},...,a_m^{\uparrow} \ | \ \begin{pmatrix}
        X_1 \\ \vdots \\ X_n \end{pmatrix})$ совместна $\Longleftrightarrow$ после алгоритма Гаусса $\widetilde{A} \longrightarrow \begin{pmatrix}
          \text{тут матрица блять :)} \ \ <3
        \end{pmatrix}$
        \begin{subtheorem}
          Доказать, что эти уравнения ЛНЗ.
        \end{subtheorem}
      \item Пусть дана ОСЛУ: \ $\underset{(r\times n)}{C} X = 0, \ rkC = r$
      $$C \longrightarrow \begin{pmatrix}
        E_r & \vline & D
      \end{pmatrix} = C'$$
      $$\begin{cases}
        x_1  = -(d_{1,r+1}x_{1,r+1}+...+d_{1n}x_n) \\
        \vdots \\
        x_k  = -(d_{k,r+1}x_{k,r+1}+...+d_{kn}x_n) 
      \end{cases} k = 1,...,r$$
      Фундаментальная матрица: $\mathcal{F} = \begin{pmatrix}
        -D \\ \hline E_{n-r}
      \end{pmatrix}$
      $$C' \cdot \mathcal{F} = E_r \cdot (-D) + D \cdot E_{n-r} = -D + D =0$$
      Рассмотрим матрицу из строк координат векторов $a_1,...,a_r$: 
      $$\begin{pmatrix}
        a_1^{\uparrow} \\ \vdots \\ a_r^{\uparrow}
      \end{pmatrix} \rightsquigarrow \begin{pmatrix}
        M & \vline & E_r
      \end{pmatrix} \longrightarrow \begin{pmatrix}
        M^T \\ \hline E_r
      \end{pmatrix} = \mathcal{F} - \begin{matrix}
        \text{фундаментальная матрица} \\ \text{искомой системы}
      \end{matrix}$$
       Тогда искамая система будет иметь матрицу: \  $C = \begin{pmatrix}
        E_{n-r} & \vline & -M^T
       \end{pmatrix}$
       Пространство $\{X \ | \ CX = 0\}$ имеет размерность $n - (n-r) = r$     
    \end{enumerate}
  \end{proof}
 В процессе...
	123

\end{document}
