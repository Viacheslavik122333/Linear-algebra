\section{Анулирующие многочлены линейных операторов}
    Пусть $\phi: \ V\to V$ - линейный оператор над полем $\F$.
    \begin{definition}
        Линейный оператор $\phi: \ V\to V$ такой, что $\forall v\in V: \ \phi(v) = v$, называется тождественным оператором и обозначается id.
    \end{definition}
    \begin{definition}
        Многочлен $f(t) = a_0+a_1t+\ldots+a_mt^m\in\F[t]$, где $a_1\ldots a_m\in\F$, называется анулирующим многочленом оператора $\phi$ 
        $$f(\phi) = a_0\cdot\text{id}+a_1\phi+\ldots+a_m\phi^m = 0 \Longrightarrow f(A_{\phi}) = 0$$
        $\Longrightarrow A_{f(\phi)} = f\cdot A_{\phi} = a_0E+a_1A_{\phi}+\ldots+a_mA_{\phi}^m$.
    \end{definition}
    \begin{example1}
        $V = \R[t]_n$, \ $\phi = \frac{d}{dt}$
        $$\phi^n(t^n) = n!, \ \phi^{n+1}\equiv0 \Longrightarrow \text{для } \phi = \frac{d}{dt} t^{n+1} - \text{анулирующий многочлен}$$
    \end{example1}
    \begin{subtheorem}
        Если $\dim V = n \Longrightarrow \exists$ многочлен $\deg \leq n^2$, анулирующий $\phi$.
    \end{subtheorem}
    \begin{proof}
        $\dim L(V) = n^2, \ L(V) \cong M_n(\F) \Longrightarrow$ операторы \\ \{$Id, \ \phi, \ \phi^2, \ \ldots, \ \phi^{n^2}$\} - линейно зависимы, так как их больше $n^2$ $\Longrightarrow$
        $$\exists \ a_0,...,a_{n^2} \in \F: \ a_0 \cdot \text{id}+a_1\phi+\ldots+a_{n^2}\phi^{n^2} = 0$$ 
        $\Longrightarrow$ $a_0+a_1t+\ldots+a_{n^2}t^{n^2}$ - многочлен анулирующий многочлен для $\phi$
    \end{proof}
    \begin{definition}
        Могочленной матрицей (матричным многочленом) называется матрица $P = (P_{ij}(\lambda))$, где $P_{ij}(\lambda)$ - многочлены над полем, над которым задано векторное пространство.
    \end{definition}
    \begin{example1}
        $$P = \begin{pmatrix}
        1-\lambda^2 & 2\lambda+1 \\
        3\lambda^2 & \lambda^2+\lambda+1
        \end{pmatrix} = \begin{pmatrix}
        1 & 1\\
        0 & 1
        \end{pmatrix}+\begin{pmatrix}
        0 & 2\\
        0 & 1
        \end{pmatrix}\cdot \lambda+\begin{pmatrix}
        -1 & 0\\
        3 & 1
        \end{pmatrix}\cdot \lambda^2$$
    - многочлен от $\lambda$ с матричными коэффициентами.
    \end{example1}
    \begin{definition}
        Оператор $\phi: \ V \to V$ называется нулевым оператором, если образом любого вектора является нулевой вектор.
    \end{definition}
    \begin{definition}
        Для матрицы $A = (a_{ij})$ присоединённой матрицей называется матрица $\widehat{A} = (A_{ij})$, то есть $\widehat{a_{ij}} = A_{ji}$.
    \end{definition}
    \begin{properties1}
        $$A\cdot\widehat{A} = \begin{pmatrix}
        |A|\\
        \null & \ddots\\
        \null & \null & |A|
        \end{pmatrix} = |A|\cdot E$$
    \end{properties1}
    \begin{theorem} \textbf{Гамильтона-Кэли} \\
        Характеристический многочлен $\chi_{\phi}(\lambda)$ является анулирующим многочленом для линейного оператора $\phi$, то есть $\chi_{\phi}(\phi) = 0$, где 0 - нулевой оператор.\\
        В матричной форме:
        $$\forall A\in M_n(\F): \ \chi_{A}(A) = 0$$
    \end{theorem}
    \begin{proof}
        Пусть $A$ - данная матрица, тогда:  
        $$\chi_A(\lambda)|A-\lambda E| = \sum\limits_{i=0}^{n}p_i\lambda^i$$
        $$p_i\in \F, \ p_n = (-1)^n, \ \chi_A(A) = \sum\limits_{i=0}^n p_i A^i(\text{считаем, что} A^0 = E)$$
        Составим матрицу: 
        $$\widehat{A-\lambda E} = \sum\limits_{j=0}^{n-1}D_j\lambda^j, \ \text{где } D_j\in M_n(\F)$$
        Рассмотрим равенство: 
        $$(A-\lambda E)(\widehat{A-\lambda E}) = \chi_A(\lambda)E$$
        \begin{multline*}
            (A-\lambda E)\cdot\sum\limits_{j=0}^{n-1}D_j\lambda^j = \sum\limits_{j=0}^{n-1}(AD_j\lambda^j)-\sum\limits_{j=0}^{n-1}D_j\lambda^{j+1} = \\
            = AD_0+\sum\limits_{j=1}^{n-1}(AD_j-D_{j-1})\lambda^j-D_{n-1}\lambda^n = \chi_A(\lambda)E = (\sum\limits_{j=0}^{n}p_j\lambda^j)E
        \end{multline*}
        Приравняем матричные коэффициенты при соответствующих степенях $\lambda$:

        % $\lambda^0:$    $E\cdot$ $\vline AD_0 = p_0E$\\
        % $\lambda^1:$    $A\cdot$ $\vline AD_1-D_0 = p_1E\\
        % \vdots$          $\vline\\
        % \lambda^j:$    $A^j\cdot$ $\vline AD_j -D_{j-1}= p_jE\\
        % \lambda^{j+1}:$ $A^{j+1}\cdot$ $\vline AD_{j+1} -D_{j}= p_{j+1}E\\
        % \vdots$                $\vline\\
        % \lambda^n:$    $A^n\cdot$ $\vline -D_{n-1} = p_nE$\\

        $$\begin{matrix}
            E \cdot & \vline & \lambda^0: & AD_0 = P_0E \\
            A \cdot & \vline & \lambda^1: & AD_1 - D_0 = P_1E \\
            \vdots & \vline \\
            A^j \cdot & \vline & \lambda^j: & AD_j - D_{j-1} = P_jE \\
            \vdots & \vline \\
            A^n \cdot & \vline & \lambda^n: & AD_n - D_{n-1} = P_nE
        \end{matrix}$$
        Домножим равенства с любой стороны на соответствующие степени $A$ и сложим:
        $$\Longrightarrow \chi_A(A)E = 0$$
    \end{proof}


    \vspace{2cm}
    Спасибо всем кто верил в меня, я все дописал, спасибо Ярику и Егору за помощь $\heartsuit$ \{может кто-то дочитает до сюда :) \}
    
