%chktex-file 1 %chktex-file 3 %chktex-file 8 %chktex-file 9 %chktex-file 10 %chktex-file 11 %chktex-file 12 %chktex-file 13 %chktex-file 16 %chktex-file 17 %chktex-file 18 %chktex-file 25 %chktex-file 26 %chktex-file 35 %chktex-file 36 %chktex-file 37 %chktex-file 40 %chktex-file 44 %chktex-file 45 %chktex-file 49
\section{Линейные операторы}
    \begin{definition}
        Линейное отображение $\phi: \ V \to V$ называется линейным оператором 
    \end{definition}
    Далее рассматриваем линейные операторы.
    \begin{subtheorem} \tab
        \begin{enumerate}
            \item $\text{Ker} \hspace{0.09cm} \phi$ - подпространство в $V$
            \item $\text{Im} \hspace{0.09cm} \phi$ - подпространство в $V$
            \item Если $U \subset V$, то $\phi(U)$ - подпространство в $V$
        \end{enumerate} 
    \end{subtheorem}
    \begin{definition}
        Подпространство $U \subset V$ называется инвариантным относительно $\phi$ (или $\phi$ - инвариантным), если: 
        $$\forall u \in U: \ \phi(u) \in U, \text{ т.е. }\phi(U) \subseteq U$$   
    \end{definition} 
    \begin{example}\tab
        \begin{enumerate}
            \item Пусть $V = U \oplus W$. Рассмотрим $\phi: \ V \to W$. Пусть $\phi(u + w) = u$ - проекция $V$ на $U$ вдоль $W$. Тогда $U$ и $W$ - $\phi$ инвариантные продрпостранства и $\forall u \in U: \ \phi(u) = u$, а также $\forall w \in W: \ \phi(w) = w$. Итак: $U \cong V/W$
            \item Пусть $V = \R[t], \ \phi = \frac{d(...)}{dt}$ и $p(t) \to p'(t)$. Здесь инвариантным является подпространство $\R[t]_n \supset \R[t]_{n-1}, \ n \in \N \cup \{0\}$               
        \end{enumerate}
    \end{example}
    \begin{theorem}
        Если $\phi: \ V \to V$ - линейный оператор, $\dim V = n, \ U$ - инвариантное подпространство, то существует базис, в котором $A_\phi$ имеет блочный вид: 
        $$A_\phi = \begin{pmatrix}
            B & \vline & D \\ \hline 0 & \vline & C
        \end{pmatrix}$$
        Где $B$ и $C$ - квадратные: \ $B_{m \times m}, \ m = \dim U$   
    \end{theorem}
    \begin{proof}
        Выберем базис $e_1,...,e_m$ в $U$ и дополним до базиса в $V$. Тогда в полученном базисе $A_\phi$ имеет нужный вид. 
    \end{proof}
    \begin{remark}
        Пусть $U \in V$ - инвариантное подпространство для линейного оператора $\phi: \ V \to V$\\
        Ограничение $\phi$ на подпространство $U$: 
        $$\phi \hspace{0.05cm} |_u : \ U \to U; \ \  \forall u \in U: \ \phi \hspace{0.05cm} |_u (u) = \phi(u)$$
        Рассмотрим факторпространтсво: $$\overline{V} = V/U : \ \{v + u \ | \ u \in U\}$$
        и фактор-оператор: 
        $$\overline{\phi}(\overline{v}):=\overline{\phi(v)}$$
        $\forall \overline{v} \in \overline{V}: \ \overline{v} = v + u, \ u \in U \Longrightarrow \phi(\overline{v}) = \phi(v) + \phi(u) \in U \Longrightarrow \phi(\overline{v}) = \phi(v)$\\
        Т.о. $\overline{\phi}: \ \overline{V} \to \overline{V}$ - линейный оператор. 
    \end{remark}
    \tab
    \begin{theorem}\tab
        \begin{enumerate}
            \item Если существует инвариантное подпространство $U \subset V$, то в подходящем базисе:
            $$A_\phi = \begin{pmatrix}
            B & \vline & D \\ \hline 0 & \vline & C
            \end{pmatrix}\eqno(I)$$
            Где $B_{m \times m}, \ m = \dim U$, а точнее: $B$ - матрица оператора $\phi \hspace{0.05cm} |_u, \\ 
            C$ - матрица оператора $\overline{\phi}$
            \item Если $V = U \oplus W, \ U$ и $W$ - инвариантные для $\phi$, то в подходящем базисе:
            $$A_\phi = \begin{pmatrix}
                B & \vline & 0 \\ \hline 0 & \vline & C
            \end{pmatrix}\eqno(II)$$
            Причем $B = A_{\phi \hspace{0.05cm} |_u}, \ C = _{\phi \hspace{0.05cm} |_w}$. \\
            Верно и обратное, если в некотором базисе матрица $A_\phi$ имеет вид $(I)$, то для $\phi \ \exists $ инвариантное подпространство, а если $A_\phi$ имеет вид $(II)$, то $V$ - прямая сумма двух инвариантных подпространств.
        \end{enumerate}
    \end{theorem}
    \begin{proof} Обозначим $\dim V = n, \ \dim U = m, \ 0 < m < n$
        \begin{enumerate}
            \item Выберем базис в $U: \ e_1,...,e_m$ и произвольно дополним его до базиса $V$ векторами $e_{m+1},...,e_n$.
            $$\forall u \in U: \ u = \sum \limits_{i=1} ^mu_ie_i \Longrightarrow \phi(u) = \sum \limits_{i=1}^mu_i \phi(e_i)$$
            В частности, столбцы $\phi(e_1)^\uparrow, ...,\phi(e_m)^\uparrow$ имеют вид: $\left(\begin{smallmatrix}
                a_{1i} \\ \vdots \\ a_{mi}
            \end{smallmatrix}\right) \Longrightarrow $ они составляют матрицу $\begin{pmatrix}
                B \\ \hline 0
            \end{pmatrix}$.
            Столбцы матрицы $\phi(e_{m+1}^\uparrow,...,e_{n}^\uparrow)$ соответствуют номерам координат. Видно, что:
            $$B = \begin{pmatrix}
                a_{11} & \cdots & a_{1m} \\ \vdots & \null & \vdots \\ a_{m1} & \cdots & a_{mm}
            \end{pmatrix} = A_{\phi \hspace{0.05cm} |_u}$$
            $\overline{e_j} = e_j + U, \ j = m+1,...,n$ - базис в факторпространстве $\overline{V} = V/U$. 
            $$\overline{\phi(e_j)} = \sum \limits_{i=1}^ma_{ij}e_i + \sum \limits_{k=m+1}^na_{kj}e_k+U = \sum \limits_{k=m+1}^na_{kj}e_k + U = \sum \limits_{k=m+1}^na_{kj} \overline{e_k}$$
            $$\Longrightarrow C = \begin{pmatrix}
                a_{m+1,m+1} & \cdots & a_{m+1,n} \\ \vdots & \null & \vdots \\ a_{n,m+1} & \cdots & a_{nn}
            \end{pmatrix} - \text{матрица оператора } \overline{\phi}$$  
            \item Если $V = U \oplus W$, векторы $e_{m+1},...,e_n$ надо выбирать в $W$. Остальное аналогично.   
        \end{enumerate}
    \end{proof}
    \begin{theorem}(Обратная)\\
        Для второго случая, если в базисе $e_1,...,e_n$ матрица имеет вид $(II)$, то положим $U := \langle e_1,...,e_m \rangle, \ W := \langle e_{m+1},....,e_n \rangle$\\
        Из определения матрицы $A_{\phi,e}$ следует, что $U, W$ - инвариантные относительно $\phi, \ \phi \hspace{0.05cm} |_u$ имеет матрицу $B$, \ $\phi \hspace{0.05cm} |_w$ - матрицу $C$.
    \end{theorem}
    \begin{remark}
        В общем случае, если $V = U_1 \oplus ... \oplus U_s, \ U_i$ - инвариантны относительно $\phi: \ V \to V$, то в базисе, согласованным с этим разложением:
        $$A_\phi = \begin{pmatrix}
            B_1 & \null & 0\\
            \null & \ddots & \null \\
            0 & \null & B_s
        \end{pmatrix}$$   
        где $B_i$ - матрица $\phi|_{u_i}$.
    \end{remark}
    \begin{example} $\phi: \ V \to V$ 
        \begin{enumerate}
            \item $\text{Ker} \hspace{0.09cm} \phi, \ \text{Im} \hspace{0.09cm} \phi$, любое подпространство $U \supseteq \text{Im} \hspace{0.09cm} \phi$ - инвариантны.
            \item Если $U_1, U_2 - \phi$-инвариантные подпространства, то $U_1 + U_2$ и $U_1 \cap U_2$ - инвариантны   
        \end{enumerate}
    \end{example}
