%chktex-file 1 %chktex-file 3 %chktex-file 8 %chktex-file 9 %chktex-file 10 %chktex-file 11 %chktex-file 12 %chktex-file 13 %chktex-file 16 %chktex-file 17 %chktex-file 18 %chktex-file 25 %chktex-file 26 %chktex-file 35 %chktex-file 36 %chktex-file 37 %chktex-file 40 %chktex-file 44 %chktex-file 45 %chktex-file 49
\section{Анулирующие многочлены линейных операторов}
    Пусть $\phi: \ V\to V$ - линейный оператор над полем $\F$.
    \begin{definition}
        Линейный оператор $\phi: \ V\to V$ такой, что $\forall v\in V: \ \phi(v) = v$, называется тождественным оператором и обозначается id.
    \end{definition}
    \begin{definition}
        Многочлен $f(t) = a_0+a_1t+\ldots+a_mt^m\in\F[t]$, где $a_1\ldots a_m\in\F$, называется анулирующим многочленом оператора $\phi$ 
        $$f(\phi) = a_0\cdot\text{id}+a_1\phi+\ldots+a_m\phi^m = 0 \Longrightarrow f(A_{\phi}) = 0$$
        $\Longrightarrow A_{f(\phi)} = f\cdot A_{\phi} = a_0E+a_1A_{\phi}+\ldots+a_mA_{\phi}^m$.
    \end{definition}
    \begin{example1}
        $V = \R[t]_n$, \ $\phi = \frac{d}{dt}$
        $$\phi^n(t^n) = n!, \ \phi^{n+1}\equiv0 \Longrightarrow \text{для } \phi = \frac{d}{dt} t^{n+1} - \text{анулирующий многочлен}$$
    \end{example1}
    \begin{subtheorem}
        Если $\dim V = n \Longrightarrow \exists$ многочлен $\deg \leq n^2$, анулирующий $\phi$.
    \end{subtheorem}
    \begin{proof}
        $\dim L(V) = n^2, \ L(V) \cong M_n(\F) \Longrightarrow$ операторы \\ \{$Id, \ \phi, \ \phi^2, \ \ldots, \ \phi^{n^2}$\} - линейно зависимы, так как их больше $n^2$ $\Longrightarrow$
        $$\exists \ a_0,...,a_{n^2} \in \F: \ a_0 \cdot \text{id}+a_1\phi+\ldots+a_{n^2}\phi^{n^2} = 0$$ 
        $\Longrightarrow$ $a_0+a_1t+\ldots+a_{n^2}t^{n^2}$ - анулирующий многочлен для $\phi$
    \end{proof}
    \begin{definition}
        Многочленной матрицей (матричным многочленом) называется матрица $P = (P_{ij}(\lambda))$, где $P_{ij}(\lambda)$ - многочлены над полем, над которым задано векторное пространство.
    \end{definition}
    \begin{example1}
        $$P = \begin{pmatrix}
        1-\lambda^2 & 2\lambda+1 \\
        3\lambda^2 & \lambda^2+\lambda+1
        \end{pmatrix} = \begin{pmatrix}
        1 & 1\\
        0 & 1
        \end{pmatrix}+\begin{pmatrix}
        0 & 2\\
        0 & 1
        \end{pmatrix}\cdot \lambda+\begin{pmatrix}
        -1 & 0\\
        3 & 1
        \end{pmatrix}\cdot \lambda^2$$
    - многочлен от $\lambda$ с матричными коэффициентами.
    \end{example1}
    \begin{definition}
        Оператор $\phi: \ V \to V$ называется нулевым оператором, если образом любого вектора является нулевой вектор.
    \end{definition}
    \begin{definition}
        Для матрицы $A = (a_{ij})$ присоединённой матрицей называется матрица $\widehat{A} = (A_{ij})$, то есть $\widehat{a_{ij}} = A_{ji}$.
    \end{definition}
    \begin{properties1}
        $$A\cdot\widehat{A} = \begin{pmatrix}
        |A|\\
        \null & \ddots\\
        \null & \null & |A|
        \end{pmatrix} = |A|\cdot E$$
    \end{properties1}
    \begin{theorem} \textbf{Гамильтона-Кэли} \\
        Характеристический многочлен $\chi_{\phi}(\lambda)$ является анулирующим многочленом для линейного оператора $\phi$, то есть $\chi_{\phi}(\phi) = 0$, где 0 - нулевой оператор.\\
        В матричной форме:
        $$\forall A\in M_n(\F): \ \chi_{A}(A) = 0$$
    \end{theorem}
    \begin{proof}
        Пусть $A$ - данная матрица, тогда:  
        $$\chi_A(\lambda)|A-\lambda E| = \sum\limits_{i=0}^{n}p_i\lambda^i$$
        $$p_i\in \F, \ p_n = (-1)^n, \ \chi_A(A) = \sum\limits_{i=0}^n p_i A^i(\text{считаем, что } A^0 = E)$$
        Составим матрицу: 
        $$\widehat{A-\lambda E} = \sum\limits_{j=0}^{n-1}D_j\lambda^j, \ \text{где } D_j\in M_n(\F)$$
        Рассмотрим равенство: 
        $$(A-\lambda E)(\widehat{A-\lambda E}) = \chi_A(\lambda)E$$
        \begin{multline*}
            (A-\lambda E)\cdot\sum\limits_{j=0}^{n-1}D_j\lambda^j = \sum\limits_{j=0}^{n-1}(AD_j\lambda^j)-\sum\limits_{j=0}^{n-1}D_j\lambda^{j+1} = \\
            = AD_0+\sum\limits_{j=1}^{n-1}(AD_j-D_{j-1})\lambda^j-D_{n-1}\lambda^n = \chi_A(\lambda)E = (\sum\limits_{j=0}^{n}p_j\lambda^j)E
        \end{multline*}
        Приравняем матричные коэффициенты при соответствующих степенях $\lambda$:
        $$\begin{matrix}
            E \cdot & \vline & \lambda^0: & AD_0 = P_0E \\
            A \cdot & \vline & \lambda^1: & AD_1 - D_0 = P_1E \\
            \vdots & \vline \\
            A^j \cdot & \vline & \lambda^j: & AD_j - D_{j-1} = P_jE \\
            \vdots & \vline \\
            A^n \cdot & \vline & \lambda^n: & AD_n - D_{n-1} = P_nE
        \end{matrix}$$
        Домножим равенства с любой стороны на соответствующие степени $A$ и сложим:
        $$\Longrightarrow \chi_A(A)E = 0$$
    \end{proof}
    \subsection{Минимальный анулирующий многочлен линейного оператора}
    \begin{definition}
        Минимальным анулирующим многочленом линейного оператора $\phi: \ V \to V$ - это анулирующий многочлен минимальной степени, анулирующий $\phi$\\
        Обозначается: $\mu_\phi(\lambda)$ \ (Зачастую его выбирают со старшим коэффициентом $= 1$)\\
        Ясно, что: 
        $$m = \deg \mu_\phi(\lambda) \leq n \leq \deg \chi_\phi(\lambda)$$  
    \end{definition}
    \begin{theorem} \tab
        \begin{enumerate}
            \item $\mu_\phi(\lambda)$ делит анулирующий многочлен оператора $\phi$ (в частности $\chi_\phi(\lambda)$)
            \item Если $\mu_\phi(\lambda)$ - тоже минимальный многочлен $\phi$, то: 
            $$\mu'_\phi(\lambda) = \alpha \mu_\phi(\lambda), \ \alpha \neq 0$$ 
            Он определен единственным образом с условием, что страший коэффициент $ = 1$
            \item Если все корни $\lambda_i$ характеристического многочлена принадлежат $\F$, то они являются и корнями минимального многочлена
        \end{enumerate}
    \end{theorem}
    \begin{proof} \tab 
        \begin{enumerate}
            \item Пусть $p(\phi) =0$, для некоторого $p(\lambda) \in \F[\lambda]$\\
            Разделим $p$ с остатком на $\mu_\phi$:
            $$p(\lambda) = \mu_\phi(\lambda) \cdot p(\phi) + r(\phi) \Longrightarrow r(\phi) = 0$$
            $\Longrightarrow  \deg \mu_\phi(\lambda) = min \Longrightarrow r(\lambda) = 0$

            \item Т.к. $\mu_\phi(\lambda) \ | \ \mu'_\phi(\lambda)$ и $\mu'_\phi(\lambda) \ | \ \mu_\phi(\lambda) \Longrightarrow \frac{\mu'_\phi}{\mu_\phi} = \alpha \in \F^* = \F\backslash \{0\}$\\
            Если $\mu_\phi(\lambda) = \lambda^m + ... $ и $\mu'_\phi(\lambda) = \lambda^m + ... \Longrightarrow \alpha = 1$
            \item Допустим, что $\exists j: \ \mu_\phi(\lambda_j) \neq 0$, т.е. в разложение $\mu_\phi$ не входит $(\lambda - \lambda_j) \\ 
            \Longrightarrow \exists \text{ вектор } v \in V: \ \phi(v) = \lambda_jr$
            $$0 = \mu_\phi(\lambda)(v) = \prod\limits_{i\neq j}(\phi-\lambda_i)(v) \neq 0$$ 
            - противоречие             
        \end{enumerate}
    \end{proof}
    \begin{example}\tab
        \begin{enumerate}
            \item $$A_\phi = \begin{pmatrix}
                2 & 1 & 0 \\
                0 & 2 & 1 \\
                0 & 0 & 2 
            \end{pmatrix}, \ \chi_\phi(\lambda) = (2-\lambda)^3 $$
            $$A_\phi - 2E = \begin{pmatrix}
                0 & 1 & 0 \\
                0 & 0 & 1 \\
                0 & 0 & 0 
            \end{pmatrix}, \ (A - 2E)^2 \neq 0, \ (A - 2E)^3 = 0 \Longrightarrow \mu_\phi = - \chi_\phi$$
            \item $$A_\phi = \begin{pmatrix}
                2 & 0 & 0 \\
                0 & 2 & 0 \\
                0 & 0 & 1 
            \end{pmatrix}, \ \chi_\phi = (2-\lambda)^2 (1-\lambda)$$
            $$(A_\phi - 2E)(A_\phi - E) = \begin{pmatrix}
                0 & 0 & 0 \\
                0 & 0 & 0 \\
                0 & 0 & -1 
            \end{pmatrix} \cdot \begin{pmatrix}
                1 & 0 & 0 \\
                0 & 1 & 0 \\
                0 & 0 & 0 
            \end{pmatrix} = 0 \Longrightarrow \mu_\phi(\lambda) = (\lambda-2)(\lambda-1)$$   
        \end{enumerate}
    \end{example}
    \textit{Вопросы:}
    \begin{enumerate}
        \item Для каких операторов $\phi$ (или $A_\phi$) $\chi_\phi(\lambda) = \pm \mu_\phi(\lambda)$?
        \item Для каких $\phi$ корни $\mu_\phi(\lambda)$ простые?   
    \end{enumerate}
    \begin{definition}
        Оператор $\phi$ нильпотентный, если:
        $$\exists \ L \in \N: \ \phi^L = 0$$
        Если $L$ - минимальный с этим условием, то $L$ - индекс нильпотентности    
    \end{definition}
    \begin{example1}
        $D = \frac{d}{dt}$ в пространстве $\R[t]_n$, то $D^{n+1} = 0$   
    \end{example1}
    \begin{subtheorem}
        Все собственные значения нильпотельного оператора $= 0$
    \end{subtheorem} 
    \begin{proof}
        Если $v \neq 0, \ \phi(v) = \lambda v:$
        $$\Longrightarrow \phi^L = \lambda^Lv \neq 0 \Longrightarrow \lambda = 0 \Longrightarrow \chi_\phi(\lambda) = \pm \lambda^n$$
    \end{proof}  
    




    % \vspace{2cm}
    % Спасибо всем кто верил в меня, я все дописал, спасибо Ярику и Егору за помощь $\heartsuit$ \{может кто-то дочитает до сюда :) \}
    
