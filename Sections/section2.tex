%chktex-file 1 %chktex-file 3 %chktex-file 8 %chktex-file 9 %chktex-file 10 %chktex-file 11 %chktex-file 12 %chktex-file 13 %chktex-file 16 %chktex-file 17 %chktex-file 18 %chktex-file 25 %chktex-file 26 %chktex-file 35 %chktex-file 36 %chktex-file 37 %chktex-file 40 %chktex-file 44 %chktex-file 45 %chktex-file 49
\section{Векторные подпространства}
  \subsection{Примеры}
  \begin{enumerate}
    \item Геометрические векторы
    \item $F^n$ - пространство столбцов (строк) высоты (длины) $n$ с естественными операциями $(+, \cdot \lambda)$ \vspace{0.4cm}\\
    Базис $\vartheta  = \bigg\{ \Biggl( \begin{smallmatrix}
      1 \\ 0 \\ \vdots \\ 0
    \end{smallmatrix}\Biggr), \Biggl(\begin{smallmatrix}
      0 \\ 1 \\ \vdots \\ 0
    \end{smallmatrix}\Biggr), ... , \Biggl(\begin{smallmatrix}
      0 \\ 0 \\ \vdots \\ 1
    \end{smallmatrix}\Biggr) \bigg\}$ (можно взять столбцы любой\vspace{0.3cm}\\ невырожденной матрицы порядка $n$)
    \begin{remark}
      Доказать, что если $e$ - базис, $C$ - невырожденная матрица,\\ то $eC$ - тоже базис (из \textbf{(2)})
    \end{remark} 
    \begin{exercise}
      Пусть $|F| = q, \ \dim_F V = n \Longrightarrow |V| = q^n$\\
      $\dim M_{m,n} = mn$, стандартный базис - $\{E_{ij}\}$, где $E_{ij}$ содержит 1 на $ij$-ой позиции и $0$ на остальных.  
    \end{exercise}
    \item $V = \{F:\underset{(X\subseteq \R)}{X} \to \R\}$ с операциями сложения и $\lambda F$\\
    Оно бесконечномерно, если $X$ бесконечно.\\
    Если $\lambda_1, ..., \lambda_n$ - попарно различные числа, то $y_1 = e^{\lambda_1x},..., y_n = e^{\lambda_nx}$ ЛНЗ\\
    Допустим, что:
    $$\begin{cases}
      C_1y_1 + ... + C_ny_n \equiv 0\\
      C_1y'_1 + ... + C_ny'_n \equiv 0\\
      \vdots\\
      C_1y_1^{(n-1)} + ... + C_ny_n^{(n-1)} \equiv 0
    \end{cases} \Longrightarrow 
    \begin{cases}
      C_1e^{\lambda_1x} + ... + C_ne^{\lambda_nx} \equiv 0\\
      \lambda_1C_1e^{\lambda_1x} + ... + \lambda_nC_ne^{\lambda_nx} \equiv 0\\
      \vdots\\
      \lambda_1^{n-1} C_1e^{\lambda_1x} + ... + \lambda_n^{n-1} C_ne^{\lambda_nx} \equiv 0
    \end{cases}$$
    $$\Delta = V(\lambda_1,...,\lambda_n) \neq 0 \Longrightarrow C_1 = ... = C_n = 0$$
    \item $F[t]$ с естественными операциями сложения и умножения на скаляр - бесконечномерное пространство, т.к.: $\forall n \in N_0: \ 1, t, t^2,...$ - линейно независимы.\\
    $F[t]_n = \{a_0+a_1t+a_2t^2+...+a_nt^n \ | \ a_k\in F, \ k=0,...,n; \ n \in N_0\} $ - подпространство, $\dim U = n+1$, базис: $1,t,...,t^n$\\
    Тейлоровский базис: $1, t-t_0,...,(t-t_0)^n$; \ $\sum \limits_{k=0}^n\frac{f^{(k)}(t_0)}{k!}(t-t_0)^k$ 
    \item $\varOmega \neq 0$, \ $V = 2^\varOmega $ с операциями вместо сложения:
    $$A\vartriangle B = (A\cap \overline{B}) \cup (B\cap \overline{A}) \ \forall A,B \subseteq \varOmega$$
    $$F = \Z_2,  \ 0\cdot A = \emptyset , \ 1 \cdot A = A$$
    \begin{exercise}
      Доказать, что $V$ - векторное пространство над $\Z_2$
    \end{exercise}   
  \end{enumerate}
  
  \subsection{Два основных способа задания подпространства в V}
  \begin{enumerate}
    \item[\textbf{1.}] Линейная оболочка семейства векторов $S\subset V$:
    $$\langle S \rangle = \{\sum \limits_{i\in I}\lambda_is_i \text{ (канонические суммы)} \ | \ s_i \in S, \lambda_i\in F\}$$
    Частный случай: 
    $$\langle a_1,...,a_m \rangle = \{\sum \limits_{i=1}^m \lambda_i a_i \ | \ \lambda_i \in F\} = U$$
    \begin{subtheorem}
      $\langle a_1,...,a_m \rangle \subseteq V \Longrightarrow \dim \langle a_1,...,a_m \rangle = rk \{a_1,...,a_m\}$
    \end{subtheorem} 
    \begin{proof}
      $$\mu \sum \limits_{i=1}^m \lambda_i a_i = \sum \limits_{i=1}^m (\mu \lambda_i)a_i$$
      $$\sum \limits_{i=1}^m \mu_i a_i + \sum \limits_{i=1}^m \lambda_ia_i = \sum \limits_{i=1}^m(\mu_i + \lambda_i)a_i \in U$$
      Если $r = rk \langle a_1,...,a_m \rangle$, то $a_{j1},...,a_{jr}$ - базисные, то $\forall a_i$ через них тоже выражается 
      $$  \forall \sum \limits_{i=1}^m \lambda_ia_i \Longrightarrow \{a_{j1},...,a_{jr}\} - \text{базис } U$$
    \end{proof}
    \begin{algorithm}
      Алгоритм вычисления $\dim \langle a_{1},...,a_{m} \rangle$ и базиса, если известны координаты этих векторов:
      \begin{itemize}
        \item[1)] Составить матрицу: $$(a_1^{\uparrow},...,a_m^      {\uparrow}) \xrightarrow[\text{строк}]{\text{ЭП}}
        \begin{pmatrix} 
          \overmat{j_1 \ \cdots \ j_r}{1 & \null  & 0 & \vline & \null }\\
          \null & \ddots & \null & \vline & 0 \\
          0 & \null & 1 & \vline & \null \\ \hline 
          \null & 0  & \null & \vline & 0
          \end{pmatrix}$$
        \item[2)] Cтолбцы с номерами $j_1,...,j_r$ - базис в $U$, разложение оставшихся векторов можно сразу считать из преобразованной матрицы
      \end{itemize}
    \end{algorithm}
    \item[\textbf{2.}] ($\dim V = n$, известны координаты в некотором базисе)
    $$\forall \sum \limits_{i=1}^n x_ie_i = eX , \ X = \begin{pmatrix}
    x_1 \\ \vdots \\ x_n
    \end{pmatrix}$$
    $$W = \{x = eX \ | \ x \in V, \ AX = 0\} - \text{ задание с помощью ОСЛУ}$$
    \begin{subtheorem}
      $W$ - подпространство в $V$,  \ $\dim W = n - rkA$, \ базис - любая ФСР (это переход от \textbf{2.} к \textbf{1.} способу задания подпространства).
    \end{subtheorem} 
  \end{enumerate}
  \begin{theorem}
    Линейную оболочку конечного числа векторов в конечномерном векторном пространстве $V$ можно задать с помощью ОСЛУ.
  \end{theorem}
  \begin{proof} Два способа:
    \begin{enumerate}
      \item[1)] Вектор $x \ ($со столбцами координат $X = \begin{pmatrix}
        x_1 \\ \vdots \\ x_n
      \end{pmatrix})$: 
      $$x \in \langle a_1,...,a_m \rangle = U$$ 
      $$ \Longleftrightarrow \exists \ \alpha_1,...,\alpha_m \in F :  \sum \limits_{i=1}^m \alpha_i a_i = x,  \text{ или в координатах: } \sum \limits_{i=1}^m \alpha_1a_i^{\uparrow} = x$$
      т.е. СЛУ с $\widetilde{A} = (a_1^{\uparrow},...,a_m^{\uparrow} \ | \ \begin{pmatrix}
        x_1 \\ \vdots \\ x_n \end{pmatrix})$ совместна $\Longleftrightarrow$ после алгоритма Гаусса: 
        $$\widetilde{A} \longrightarrow \begin{pmatrix}
          K & \vline &  \sum \limits_jC_{kj}x_j \\ \hline
          0 & \vline &  \sum \limits C_{r+1, j}x_j = 0 \\
          \null & \vline & \sum \limits C_{nj}x_j = 0
        \end{pmatrix}$$
        $\begin{pmatrix}
          K
        \end{pmatrix}$ имеет ступенчатый вид, а $\begin{pmatrix}
        \sum \limits C_{r+1, j}x_j = 0 \\
        \sum \limits C_{nj}x_j = 0
        \end{pmatrix}$ - нужная нам система. 
        \begin{exercise}
          Доказать, что эти уравнения ЛНЗ.
        \end{exercise}
      \item[2)] Пусть дана ОСЛУ: \ $\underset{(r\times n)}{C} X = 0, \ rkC = r$
      $$C \xrightarrow[\text{строк}]{\text{ЭП}} \begin{pmatrix}
        E_r & \vline & D
      \end{pmatrix} = C'$$
      $$\begin{cases}
        x_1  = -(d_{1,r+1}x_{1,r+1}+...+d_{1n}x_n) \\
        \vdots \\
        x_k  = -(d_{k,r+1}x_{k,r+1}+...+d_{kn}x_n) 
      \end{cases} k = 1,...,r$$
      Фундаментальная матрица: \ $\mathcal{F} = \begin{pmatrix}
        -D \\ \hline E_{n-r}
      \end{pmatrix}$
      $$C' \cdot \mathcal{F} = E_r \cdot (-D) + D \cdot E_{n-r} = -D + D =0$$
      Рассмотрим матрицу из строк координат векторов $a_1,...,a_r$: 
      $$\begin{pmatrix}
        a_1^{\to} \\ \vdots \\ a_r^{\to}
      \end{pmatrix} \xrightarrow[\text{ступенчатый вид}]{\text{улучшенный}} \begin{pmatrix}
        M & \vline & E_r
      \end{pmatrix} \xrightarrow{\text{Транспонируем}} \begin{pmatrix}
        M^T \\ \hline E_r
      \end{pmatrix} = \mathcal{F}$$
       Тогда искомая система будет иметь матрицу: \  $C = \begin{pmatrix}
        E_{n-r} & \vline & -M^T
       \end{pmatrix}$\\
       Пространство $\{X \ | \ CX = 0\}$ имеет размерность $n - (n-r) = r$     
    \end{enumerate}
  \end{proof}