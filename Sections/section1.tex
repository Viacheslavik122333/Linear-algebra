% \section{Векторное пространство}
%     \begin{definition}
%         Множество $V$ называется \textit{векторным пространством} над полем $F$, если заданы операции $+$ и $\cdot$\ :$V\times V \to V$, $F \times V \to V$ и выполнены следующие аксиомы:
%         \begin{enumerate}
%             \item $\forall v_1, v_2, v_3\in V$ \ : \ $(v_1 + v_2) + v_3 = v_1 + (v_2 + v_3)$
%             \item $\exists \ \vec 0 \in V:\ \forall v \in V$ \ : \ $v + \vec 0 = v$
%             \item $\forall v \in V \ \ \exists  -v  \in V$: $v + (-v) = \vec 0$
%             \item $\forall v_1, v_2 \in V$ \ : \ $v_1 + v_2 = v_2 + v_1$
%             \item $\forall \alpha, \beta \in F, v \in V$ \ : \ $(\alpha \beta)v = \alpha (\beta v)$
%             \item $\forall v \in V$ \ : \ $1 \cdot v = v$
%             \item $\forall \alpha, \beta \in F, v \in V$ \ : \ $(\alpha + \beta)v = \alpha v + \beta v$
%             \item $\forall \alpha \in F, v_1, v_2 \in V$ \ : \ $\alpha (v_1 + v_2) = \alpha v_1 + \alpha v_2$
%         \end{enumerate}       
%     \end{definition}