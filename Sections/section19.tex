%chktex-file 1 %chktex-file 3 %chktex-file 8 %chktex-file 9 %chktex-file 10 %chktex-file 11 %chktex-file 12 %chktex-file 13 %chktex-file 16 %chktex-file 17 %chktex-file 18 %chktex-file 25 %chktex-file 26 %chktex-file 35 %chktex-file 36 %chktex-file 37 %chktex-file 40 %chktex-file 44 %chktex-file 45 %chktex-file 49
\section{Квадратичные формы}
Пусть $k(x) = b(x,x)$ - квадратичная функция на пространстве $\E$. $B$ - её матрица в некотором базисе  ( $B^T =B$ )
\begin{theorem}
    В $B \ \exists$ ортонормированный базис $f = eC$, в котором эта квадратичная форма имеет вид: 
    $$k = \lambda_1y_1^2 + ... + \lambda_ny_n^2 \ \ \text{ (где $\lambda_1,...,\lambda_n$ - собственные значения матрицы $B$ )}$$
    $\lambda = C \gamma$ - соответствующая замена координат, \ $f_1,...,f_n$ - "главные оси" для квадратичной формы $k(x)$\\
    Нахождение замены - процесс приведения к главным осям 
\end{theorem} 
\begin{proof}
    Примем $B$ за матрицу самосопряженного оператора $\phi$ в некотором ортонормированном базисе.\\
    Т.к. $\phi$ - самосопряжен, то $\exists$ ортонормированный базис $f_1,...,f_n$ из собственных векторов оператора $\phi$, т.е. $\exists$ ортогональная матрица $C$:
    $$C^{-1}BC = \begin{pmatrix}
        \lambda_1 & \null & 0 \\ \null & \ddots \\ 0 & \null & \lambda_n
    \end{pmatrix}$$
    Но, так как $C$ - ортогональный, то $C^{-1} = C^T$ 
    $$\Longrightarrow C^TBC = \begin{pmatrix}
        \lambda_1 & \null & 0 \\ \null & \ddots \\ 0 & \null & \lambda_n
    \end{pmatrix}$$        
\end{proof} 

 
 