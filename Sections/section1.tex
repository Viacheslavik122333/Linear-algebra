%chktex-file 1 %chktex-file 3 %chktex-file 8 %chktex-file 9 %chktex-file 10 %chktex-file 11 %chktex-file 12 %chktex-file 13 %chktex-file 16 %chktex-file 17 %chktex-file 18 %chktex-file 25 %chktex-file 26 %chktex-file 35 %chktex-file 36 %chktex-file 37 %chktex-file 40 %chktex-file 44 %chktex-file 45 %chktex-file 49
\section{Векторное пространство}
  \begin{definition}
    Множество $V$ называется \textit{векторным пространством} над полем $F$, если заданы операции $"+"$ и $"\cdot"$\ : \ $V\times V \to V$, $F \times V \to V$ и выполнены следующие аксиомы:
    \begin{enumerate}
        \item $\forall v_1, v_2, v_3\in V$ \ : \ $(v_1 + v_2) + v_3 = v_1 + (v_2 + v_3)$
        \item $\exists \ \vec 0 \in V:\ \forall v \in V$ \ : \ $v + \vec 0 = v$
        \item $\forall v \in V \ \ \exists  -v  \in V$: $v + (-v) = \vec 0$
        \item $\forall v_1, v_2 \in V$ \ : \ $v_1 + v_2 = v_2 + v_1$
        \item $\forall \alpha, \beta \in F, v \in V$ \ : \ $(\alpha \beta)v = \alpha (\beta v)$
        \item $\forall v \in V$ $\exists \ 1 \in F$ \ : \ $1 \cdot v = v$
        \item $\forall \alpha, \beta \in F, v \in V$ \ : \ $(\alpha + \beta)v = \alpha v + \beta v$
        \item $\forall \alpha \in F, v_1, v_2 \in V$ \ : \ $\alpha (v_1 + v_2) = \alpha v_1 + \alpha v_2$
    \end{enumerate}
    \textit{Загадка:} Одна из этих аксиом - следствие других. Какая?
    \begin{proof}
      Cначала докажем два свойства.
      \begin{enumerate}
        \item $0\cdot\overline{a} = 0\cdot\overline{a}+\overline{0} = 0\cdot\overline{a}+(0\cdot\overline{a}+(-0\cdot\overline{a})) = (0\cdot\overline{a}+0\cdot\overline{a}) + (-0\cdot\overline{a})$ (по аксиоме ассоциативности) $= 0\cdot\overline{a}+(-0\cdot\overline{a}) = \overline{0}$
        \item $(-1)\overline{a}+\overline{0} = (-1)\overline{a}+(\overline{a}+(-\overline{a})) = ((-1)\overline{a}+\overline{a})+(-\overline{a})$ (по аксиоме ассоциативности) $= 0\cdot\overline{a}+(-\overline{a}) = -\overline{a}$.
      \end{enumerate}
      Теперь докажем первую аксиому (аксиому коммутативности).
      $$(\overline{a}+\overline{b})+\overline{0} = (\overline{a}+\overline{b})+(-(\overline{b}+\overline{a})+(-(-(\overline{b}+\overline{a}))) = $$
      (по второму свойству) 
      $$ = (\overline{a}+\overline{b})+(-(\overline{b}+\overline{a})+(\overline{b}+\overline{a})) = $$ 
      (по аксиоме ассоциативности) 
      $$=(\overline{a}+\overline{b}+(-(\overline{b}+\overline{a})))+(\overline{b}+\overline{a}) = (((\overline{a}+\overline{b})+(-(\overline{b})))+(-\overline{a}))+(\overline{b}+\overline{a}) = $$
      $$= ((\overline{a}+(\overline{b}+(-(\overline{b}))))+(-\overline{a}))+(\overline{b}+\overline{a}) =
      ((\overline{a}+\overline{0})+(-\overline{a}))+(\overline{b}+\overline{a}) =$$
      $$(\overline{a}+(-\overline{a}))+(\overline{b}+\overline{a})
      = \overline{0}+(\overline{b}+\overline{a}) = \overline{b}+\overline{a}$$
    \end{proof} 
  \end{definition}
  \begin{definition}
    $U \subset  V$ - \textit{векторное подпространство} пространства $V$, если оно само является пространством относительно тех же операций в $V$. 
  \end{definition}
  \begin{subtheorem}
    Определение 2 эквивалентно:
    \begin{enumerate}
      \item $U\neq \varnothing $
      \item $\forall u_1, u_2 \in U$ \ : \  $u_1 + u_2 \in U$
      \item $\forall u \in U, \ \lambda \in F$ \ : \ $\lambda u\in U$
    \end{enumerate}
  \end{subtheorem} 
  \begin{definition}
    Векторы $v_1,...,v_n \in V$ называются линейно зависимыми, если $\exists \ \lambda_1,..., \lambda_n$ (не все равные 0) \ : \ $\lambda_1v_1+...+\lambda_nv_n = \vec 0$
  \end{definition} 
  \begin{subtheorem}
    Определение 3 $\Longleftrightarrow $ ($m\geq 2$) хотя бы один вектор из векторов $v_i$ выражается как линейная комбинация остальных. 
  \end{subtheorem}
  \begin{definition}
    Упорядоченный набор векторов $e = (e_1,...,e_n), e_k \in V$ называется базисом $V$, если $e$ - максимальный ЛНЗ набор веторов из $V$.  
  \end{definition} 
  \begin{subtheorem}
    $e$ - базис в $V \Longleftrightarrow$
    \begin{enumerate}
      \item $e_1,...,e_n$ - ЛНЗ
      \item $\forall x \in V \ \exists \ x_1,...,x_n \in F \ : \ x = x_1e_1+...+x_ne_n = \sum \limits_{i=1}^nx_ie_i $ 
    \end{enumerate}
  \end{subtheorem} 
  \begin{consequense}
    Разложение любого вектора в базисе единственно.
  \end{consequense} 
  \begin{proof}
    Если $x = \sum \limits_{i=1}^nx_ie_i = \sum \limits_{i=1}^nx'_ie_i$, то $\vec 0 = x - x = \sum \limits_{i=1}^n(x'_i-x_i)e_i$\\
    Из ЛНЗ все коэффициенты равны 
  \end{proof} 
  Обозначаем: $X_e = \begin{pmatrix}
    x_1\\ \vdots\\ x_n
  \end{pmatrix} \in F^n$, тогда $x = eX_e = e_1x_1+...+e_nx_n$  
  \begin{equation}
    \fbox{$x = eX_e$}
  \end{equation}
  \begin{theorem}
    Если в $V \ \exists$ базис из $k$ векторов, то любой базис $V$ содержит $k$ векторов.    
  \end{theorem}
  \begin{proof} $\\$ 
    Если $\exists$ базис $e'_1,...,e'_m \in V$, где $m>n$, то по ОЛЛЗ $e'_1,...,e'_m$ - ЛЗ, т.е. не базис.\\
    Если же $m<n$, то по ОЛЛЗ (в другую сторону) $e_1,...,e_n$ - ЛЗ $\Longrightarrow$ не базис.       
  \end{proof}
  \begin{properties} матриц перехода
    \begin{enumerate}
      \item $\det C \neq 0$
      \item $C_{e' \to e} = (C_{e \to e'})^{-1}$
      \item $C_{e \to e''} = C_{e \to e'} \cdot C_{e' \to e''}$
    \end{enumerate}
  \end{properties}
  \begin{proof}\tab
    \begin{itemize}
      \item[$1)$] Столбцы - координаты ЛНЗ векторов $e'_1,...,e'_n \Longrightarrow rkC = n \Longrightarrow \det C \neq 0$
      \item[$2)$] Перепишем определение матрицы перехода в матричный вид. \\
      По определению: 
      $$e'=(e'_1,...,e'_n) = (e_1,...,e_n)C_{e \to e'}, \text{ т.е. } e' = eC_{e \to e'}$$
      \begin{equation}
        \fbox{$e' = eC_{e \to e'}$}
      \end{equation}
      С другой стороны 
      $$e = e'C_{e' \to e} = eC_{e \to e'}C_{e' \to e} \Longrightarrow C_{e \to e'}C_{e' \to e} = E$$ 
      ввиду единственности разложения векторов по базису, т.е. 
      $$C_{e \to e'} = (C_{e' \to e})^{-1}$$
      \item[$3)$] $$e'' = e'C_{e' \to e''} = e(C_{e \to e'}C_{e' \to e''}) = eC_{e \to e''}$$
      В силу единственности разложения $C_{e \to e''} = C_{e \to e'}C_{e' \to e''}$     
    \end{itemize}
  \end{proof} 
  \begin{algorithm}
    Как вычислить матрицу перехода, если известны координаты векторов $e_i$ и $e'_j$ в некотором универсальном базисе?\\
    $e' = eC_{e \to e'}$ можно рассмотреть как матричное уравнение:
    $$(e_1^{\uparrow},...,e_n^{\uparrow})C = ({e_1'}^{\uparrow},...,{e_n'}^{\uparrow})$$
    $$[e_1^{\uparrow},...,e_n^{\uparrow} \ | \ {e_1'}^{\uparrow},...,{e_n'}^{\uparrow}] \overset{\text{строк}}{\rightsquigarrow} [E \ | \ C_{e \to e'}]$$   
  \end{algorithm}
  \subsection{Изменение координат вектора при замене базиса}
  \begin{theorem}
    Формула изменения координат вектора при замене базиса:
    \begin{equation}
      \fbox{$X_e = C_{e \to e'}X_{e'}$} 
    \end{equation}
  \end{theorem} 
  \begin{proof}
    $$\forall x \in V \ : \ x = eX_e = e'X_{e'} = eC_{e \to e'}X_{e'}$$
    $$\Longrightarrow  X_e = C_{e \to e'}X_{e'}$$ 
  \end{proof}