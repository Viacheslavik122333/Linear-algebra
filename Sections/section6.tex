%chktex-file 1 %chktex-file 3 %chktex-file 8 %chktex-file 9 %chktex-file 10 %chktex-file 11 %chktex-file 12 %chktex-file 13 %chktex-file 16 %chktex-file 17 %chktex-file 18 %chktex-file 25 %chktex-file 26 %chktex-file 35 %chktex-file 36 %chktex-file 37 %chktex-file 40 %chktex-file 44 %chktex-file 45 %chktex-file 49
\section{Линейные функции}
    Пусть $V$ - векторное пространство над $\F$
    \begin{definition}
        Отображение $f: \ V \to \F$ - линейная функция со значениями в $\F$, если:
        \begin{enumerate}
            \item $\forall v_1, \ v_2 \in V: \ f(v_1 + v_2) = f(v_1) = f(v_2)$
            \item $\forall v \in V, \forall \lambda : \ f(\lambda v) = \lambda f(v)$  
        \end{enumerate}
        Обозначается: $V^{*} = \{f : \ V \to \F\}$ - множество линейных функций на $V$  
    \end{definition}
    \begin{remark}
        $V_2 = \F, \ \dim V_2 = 1$ 
    \end{remark}
    \begin{lemma}
        Если $f \not \equiv 0$, то $\text{Ker}(f)$ имеет в $V$ коразмерность = 1 
    \end{lemma} 
    \begin{proof}
        Пусть $\exists v_1 \in V, \ f(v_1) \neq 0$. Пусть $v \in V$, либо $v \in \text{Ker}(f)$, либо $f(v) = \alpha \neq 0$ 
        $$\beta = f(v_1) \neq 0 \Longrightarrow f(\frac{v_1}{\beta}) = 1, \ f(\alpha - \frac{v_1}{\beta}) = \alpha$$
        Рассмотрим выражение $r - \frac{\alpha}{\beta}v_1$:
        $$f(r - \frac{\alpha}{\beta}v_1) = f(v) - f(\frac{\alpha}{\beta}v_1) = \alpha - \alpha = 0$$
        $\Longrightarrow r - \frac{\alpha}{\beta}v_1 \in \text{Ker}(f)$ и $v = \frac{\alpha}{\beta}v_1 + u, \ u \in \text{Ker}(f)$  
    \end{proof}
    \begin{remark}
        $\forall x \in V: \ (f_1 + f_2)(x) = f_1(x) + f_2(x)$  и $(\lambda f)(x) = \lambda f (x)$ 
    \end{remark}
    \begin{lemma}
        Множество $V^{*}$ с введенными операциями - векторное пространство. 
    \end{lemma}
    \begin{definition}
        $V^{*}$ - векторное пространство, сопряженное с $V$ (двойственное для $V$)\\
        Зафиксируем базис $e = (e_1, ..., e_n)$ в $V$ и линейную функцию $f:V \rightarrow F$
        $$\forall x \in V: \ x = \sum \limits_{i=1}^n x_i e_i \Rightarrow f(x) = \sum \limits_{i=1}^n x_i f(e_i) = \sum \limits_{i=1}^n a_i x_i, \ \text{где } a_i = f(e_i)$$ 
        Удобно записывать это так: $f(x) = \begin{pmatrix}
            a_1 & \cdots & a_n
        \end{pmatrix}\begin{pmatrix} x_1 \\ \vdots \\ x_n \end{pmatrix}$
    \end{definition} 
    
    \begin{definition}
        Координатные функции - функции вида: 
        $$f_i: \ f_i(x) = x_i$$
        Будем использовать обозначение: $e^i = f_i$ \vspace{0.4cm}\\
        В частности: $f_i(e_j) = e^i(e_j) = \begin{cases}
            1, \ i=j\\
            0, \ i \neq j
        \end{cases}$
    \end{definition}
    \begin{subtheorem}
        Функции $e^i$ - базис в $V^{*}$
    \end{subtheorem}
    \begin{proof} $\\$ 
        Докажем ЛНЗ: Пусть $\exists \ \lambda_1, ..., \lambda_n: \ \sum \limits_{i=1}^n \lambda_i e^i \equiv 0$. Подставим  $e_j$:
        \[(\sum \limits_{i=1}^n \lambda_i e^i)(e_j) = \sum \limits_{i=1}^n \lambda_i e^i(e_j) = \lambda_j = 0\]
        Отсюда после подстановки всех $e_1,...,e_n$ получим, что $\forall i = 1,...,n: \ \lambda_i = 0$.
        Разложим произвольную функцию $f \in V^{*}$:
        \[f(x) = \sum \limits_{i=1}^n a_i x_i = \sum \limits_{i=1}^n a_i e^i(x) = (\sum \limits_{i=1}^n a_i e^i)(x) \ \ \forall x\in V \ f \equiv \sum \limits_{i=1}^n a_i e^i\]
    \end{proof}
    \begin{consequense}
        Если $\dim V < \infty$, то $V^{*} \cong V$, т.к. $\dim V^{*} = \dim V$.
    \end{consequense}
    \begin{definition}
        Базис $e^{*} = (e^1,...,e^n)$ называется базисом $V^{*}$, сопряжённым (дуальным, двойственным, биортогональным) к базису $e$ в $V$.
    \end{definition}

    Посмотрим, как изменится строка координат функции $f\in V^{*}$ при замене базиса $e$ в $V$.\\
    Пусть $e' = (e_1',...,e_n') = e\cdot C_{e\rightarrow e'}$ - новый базис в $V$. 
    Как известно, $X = C_{e\rightarrow e'} \cdot X'$.\\
    Отсюда если $x = \sum \limits_{i=1}^n x_i e_i = \sum \limits_{i=1}^n x_i' e_i'$, то
    $$\forall f \in V: \ f(x) = \sum \limits_{i=1}^n a_i' x_i'$$  
    $$f(x) = (a_1,...,a_n)X = (a_1,...,a_n)(C_{e\rightarrow e'}X') = ((a_1,...,a_n)C_{e\rightarrow e'})X'$$ 
    $$((a_1,...,a_n)C_{e\rightarrow e'})X' = ((a_1',...,a_n'))X' \ \ \forall X' \in \mathbb{F}^n$$ 
    Беря по очереди $X' = \left( \begin{smallmatrix} 1 \\ 0 \\ \vdots \\ 0 \end{smallmatrix} \right), \dots , \left( \begin{smallmatrix} 0 \\ 0 \\ \vdots \\ 1 \end{smallmatrix}\right)$, покоординатно получим равенство \[(a_1,...,a_n)C_{e\rightarrow e'} = (a_1',...,a_n')\]

    \begin{example1}
        Возьмём $V = \mathbb{R}[t]_n = \{p(t) \in \mathbb{R}[t] \ | \ \deg p = n\}$\\
        Выберем в нём базис $\{1, (t-t_0), ... , (t-t_0)^n\} \Longrightarrow  p(t) = \sum \limits_{i=0}^n \frac{p^{(i)}(t_0)}{i!}(t-t_0)^i$\\
        Если $e_i = (t-t_0)^i, \ 0\leqslant i\leqslant n$, то $e^i(p) = \frac{p^{(i)}(t_0)}{i!}$
    \end{example1}
    \begin{definition}
        Вторым сопряжённым пространством к $V$ (обозначается $V^{**}$) называется пространство, сопряженное к $V^{*}$ - пространство линейных функций от линейных функций над $V$.
        $$V^{**} = \{\phi: \ V^{*} \to \F\}$$ 
    \end{definition}
    \begin{theorem}
        Если $\dim V < \infty$, то $V^{**} \cong V$, причём изоморфизм не зависит от выбора базиса (такой изоморфизм называется каноническим). 
    \end{theorem}
    \begin{proof}
        Рассмотрим отображение: 
        $$\phi: V \rightarrow V^{**}: \ \forall x \in V: \ \phi(x) = \phi_x \in V^{**}$$
        $$\Longrightarrow \forall f\in V^{*}, \phi_x(f) = f(x)$$
        Это линейное отображение:
        \begin{itemize}
            \item $\forall f\in V^{*}, \ \phi_{x_1+x_2}(f) = f(x_1 + x_2) = f(x_1) + f(x_2) = \phi_{x_1}(f) + \phi_{x_2}(f) \Longrightarrow  \phi(x_1 + x_2) = \phi(x_1) + \phi(x_2);$ 
            \item $\forall f\in V^{*}, \ \phi_{\lambda x}(f) = f(\lambda x) = \lambda f(x) = \lambda\phi_{x}(f) \Longrightarrow \phi(\lambda x) = \lambda\phi(x);$ 
        \end{itemize}
        Чтобы проверить, что $\phi$ - изоморфизм, достаточно проверить, что $\text{Ker} (\phi) = \{0\}$ (так как $\dim V^{**} = \dim V$).\\
        Пусть $x \in \text{Ker} (\phi)$, т.е. $\phi_x \equiv 0$. Значит,  $\forall f \in V^{*} : \ f(x) = 0$\\
        Если $x \neq 0$, то его можно дополнить до базиса: $x, e_2, ... , e_n$, где $n = \dim V$.\\
        Тогда $e^1(x) = 1 \neq 0$ - противоречие с условием $\forall f \in V^{*} : \ f(x) = 0$.
    \end{proof}
    \textit{Задача.} Доказать, что $a_1,...,a_n\in V$ ЛНЗ $\Leftrightarrow \exists$ лин. ф-ции $f^1,...,f^n\in V^{*}$ такие, что $\det(f^i(a_j)) \neq 0$.
    \begin{remark}
        Если $dim V = \infty$, то $V^{*} \ncong V$ в общем случае.
    \end{remark}
    \begin{example1}
        $V = \mathbb{Q}[t]$ - $V$ счётно. Зафиксируем число $t\in \mathbb{Q}$ и рассмотрим произвольную $f \in V^{*}$:\\ 
        $f(t^k) = b_k \Rightarrow f \leftrightarrow (b_0, b_1,..., b_k,...) \Rightarrow$ $V^{*}$ континуально.\\
        Отсюда мощность $V^{*}$ больше мощности $V$, и они, очевидно, не изоморфны.
    \end{example1}