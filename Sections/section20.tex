%chktex-file 1 %chktex-file 3 %chktex-file 8 %chktex-file 9 %chktex-file 10 %chktex-file 11 %chktex-file 12 %chktex-file 13 %chktex-file 16 %chktex-file 17 %chktex-file 18 %chktex-file 25 %chktex-file 26 %chktex-file 35 %chktex-file 36 %chktex-file 37 %chktex-file 40 %chktex-file 44 %chktex-file 45 %chktex-file 49
\section{Полуторалинейные, эрмитовы формы. Унитарные (эрмитовы) пространства}
Далее всюду $F = \CC, V$ - в.п. над $\CC$.
\begin{definition}
    Функция $f(x, y): V\times V \rightarrow \CC$ называется полуторалинейной, если:
    \begin{enumerate}
        \item $f(x_1 + x_2, y) = f(x_1, y) + f(x_2, y)$;\\
              $f(\lambda x, y) = \lambda f(x, y) \ (\lambda \in \CC)$;
        \item $f(x, y_1 + y_2) = f(x, y_1) + f(x, y_2)$;\\
              $f(x, \lambda y) = \overline{\lambda} f(x, y) \ (\lambda \in \CC)$
    \end{enumerate}
\end{definition}
\begin{definition}
    $f(x,y)$ называется эрмитово симметричной (эрмитовой), если 
    \begin{enumerate}
        \item $f(x, y)$ линейна по $x$;
        \item $f(y, x) \equiv \overline{f(x,y)} \ (\Longrightarrow f(x, \lambda y) = \overline{\lambda}f(x,y) \ \forall \lambda \in \CC)$
    \end{enumerate}
\end{definition}
Заметим, что если $f(x,y)$ эрмитова, то $f(x, x) \equiv \overline{f(x,x)} \Rightarrow f(x,x) \in \R$.
\begin{definition}
    Квадратичная функция, порождённая эрмитовой формой - это функция $k(x) \equiv f(x,x)$.
\end{definition}
\begin{exercise}
    Доказать, что для любой квадратичной формы $k(x) \ \exists!$ эрмитова форма $f(x, y)$ такая, что $f(x,x) \equiv k(x)$. 
\end{exercise}
Если $f(x,y)$ полуторалинейна и эрмитова, то обозначим $F = (f(e_i, e_j))$, и тогда $f(e_j, e_i) = \overline{f(e_i, e_j)} \Longrightarrow F^T = \overline{F} \Longleftrightarrow \overline{F}^T = F$.
\begin{definition}
    $F^* = \overline{F}^T$ - эрмитово сопряжённая матрица к $F$.\\
    Если $F^* = F$, то $F$ - эрмитова матрица.
\end{definition} 
\begin{definition}
    Скалярное произведение на пр-ве $V$ - функция $(x,y)$ такая, что
    \begin{enumerate}
        \item $(x,y)$ линейна по $x$;
        \item $(y,x) \equiv \overline{(x,y)}$;
        \item $(x,x) > 0 \ \forall x \neq 0$
    \end{enumerate}
\end{definition}
Скалярное произведение в координатах:
$$(\sum \limits_{k=1}^n x_ke_k, \sum \limits_{j=1}^n y_je_j) = \sum \limits_{k=1}^n x_k(e_k, \sum \limits_{j=1}^n y_je_j) = \sum \limits_{k, j=1}^n x_k\overline{y_j}(e_k, e_j)$$
Матрица Грама базиса $e$: 
$$G_e = ((e_k, e_j)). \ G_{e^*} = \overline{G_e}^T = G_e$$
\begin{definition}
    $x \perp y \Longleftrightarrow (x, y) = 0$.\\
    Базис $e_1,...,e_n$ ортогональный, если $(e_k, e_j) = 0, \ k \neq j$.\\
    Базис $e_1,...,e_n$ ортонормированный, если $(e_k, e_j) = \delta_{ij}$.
\end{definition}
В ортонормированном базисе $(x, y) = \sum \limits_{j=1}^n x_j\overline{y_j}$.

Изменение матрицы полуторалинейной формы при замене базиса:\\
Если $f(x, y)$ - полуторалинейная форма, то в некотором базисе $e:$ 
$$f(x, y) = X^TF\overline{Y}, \text{ где } F = (f(e_i, e_j))$$ 
Если $f$ эрмитово симметричная, т.е. $\overline{f(y, x)} = f(x, y)$, то $\overline{F}^T = F$.\\
Тогда если $e' = Ce$, то в случае полуторалинейной формы:
$$X = CX', Y = CY' \Rightarrow f(x, y) = (X')^T(C^TF\overline{C})\overline{Y'} = (X')^TF'\overline{Y'}$$
В случае эрмитовой квадратичной формы $k(x) = f(x,x)$:
$$k(x) = \sum \limits_{k,j = 1}^n x_k\overline{x}_jf_{kj} = ... + f_{kj}x_k\overline{x}_j + ... , \ f_{jk} = \overline{f}_{kj}$$
Отсюда $f_{ii} = \overline{f}_{ii}$, т.е. $f_{ii} \in \R$.
\begin{theorem}
    Эрмитову квадратичную форму можно привести к диагональному виду $\alpha_1|x_1|^2 + ... + \alpha_r|x_r|^2$, где $r = \text{rk} F, \ \alpha_1,...,\alpha_r \in \R, \alpha_j \neq 0$. Количество положительных коэффициентов $p$ и отрицательных коэффициентов $q$ - инварианты для данной формы. 
\end{theorem}
\begin{proof}
    Применим следующий вариант алгоритма Лагранжа:
    Основной случай. Если $b_{11} \neq 0$, то необходимо выделить все одночлены, содержащие $x_1$ и $\overline{x}_1$:
    \begin{multline*}
    k(x_1,...,x_n) = (b_{11}x_1\overline{x}_1 + ... + b_{n1}x_n\overline{x}_1) + (b_{12}x_1\overline{x}_2 + ... + b_{1n}x_1\overline{x}_n) + \tilde{k}(x_2,...,x_n) =\\= \frac{1}{\overline{b}_{11}}(b_{11}x_1 + ... + b_{n1}x_n)(\overline{b_{11}x_1} + ... + \overline{b_{n1}x_n}) + \tilde{k}(x_2,...,x_n) =\\= \frac{1}{\overline{b}_{11}}|b_{11}x_1 + ... + b_{n1}x_n|^2 + \tilde{k}(x_2,...,x_n)
    \end{multline*}
    Заменяем $y_1 = b_{11}x_1 + ... + b_{n1}x_n$ и далее преобразуем $\tilde{k}$.\\
    Особый случай: $b_{ii} = 0, i = 1,...,n$. По условию $k \not \equiv 0$, т.е. $\exists b_{ij} = \overline{b}_{ji} \neq 0$ и при замене $\begin{cases}
        x_i = b_{ji}(y_i + y_j)\\
        x_j = y_i - y_j
    \end{cases}$ форма содержит члены: 
    $$b_{ij}x_i\overline{x}_j + b_{ji}x_j\overline{x}_i = 2b_{ij}^2|y_i|^2 - 2b_{ij}^2|y_j|^2$$
    Далее можем продолжать по основному случаю.
\end{proof}

Сохраняют силу следующие утверждения и понятия:
\begin{enumerate}
    \item Теорема Якоби: $\Delta_1,...,\Delta_{n-1} \neq 0 \Longrightarrow k = \frac{\Delta_1}{\Delta_0}|y_1|^2 + ... + \frac{\Delta_n}{\Delta_{n-1}}|y_n|^2$;
    \item Критерий Сильвестра: $k > 0 \Longleftrightarrow \Delta_i > 0, i = 1,...,n$;
    \item Понятие $u^\perp$ и утверждение $V = U \oplus U^\perp$.
\end{enumerate}
\begin{remark}
    Если $A^* = \overline{A}^T = A$, то $|A| = |\overline{A}^T| = |\overline{A}| = \overline{|A|}$, т.е. $|A| \in \R$
\end{remark}
\begin{algorithm} Процесс ортогонализации:\\
    Дан произвольный базис $e_1,...,e_n \in V$. Необходимо построить ортогональный базис $e'_1,...,e'_n$ такой, что $\langle e_1,...,e_k \rangle = \langle e'_1,...,e'_k \rangle$.\\
    Возьмём $e'_1 = e_1$.\\
    \textbf{Шаг алгоритма}: \\
    Если $k > 1$ и $e'_1,...,e'_{k-1}$ уже построены, то будем искать $e'_k$ в виде: 
    $$e_k - \sum \limits_{j=1}^{k-1} \lambda_j^{(k)}e'_j$$ 
    Тогда:
    $$0 = (e'_k, e'_i) = (e_k, e'_i) - \sum \limits_{j=1}^{k-1} \lambda_j^{(k)}(e'_j, e'_i) = (e_k, e'_i) - \lambda_i^{(k)}(e'_i, e'_i) \Longrightarrow \lambda_i^{(k)} = \frac{(e_k, e'_i)}{(e'_i, e'_i)}$$ 
\end{algorithm}
\subsection{Линейные операторы в унитарном пространстве}
\begin{enumerate}
    \item Сопряжённый оператор $\phi^*$ к линейному оператору $\phi: V \rightarrow V$:
    $$\forall x,y \in V, (\phi(x), y) = (x, \phi^*(y)) \eqno(1)$$
    \item Самосопряжённый оператор: $$\phi = \phi^* \eqno(2)$$
    \item Унитарный оператор:
    $$\forall x,y \in V, (\phi(x), \phi(y)) = (x, y) \eqno(3)$$
\end{enumerate}

Для самосопряжённого оператора:
$$(2) \Longleftrightarrow (\phi(x), y) \equiv (x, \phi(y)) \Longrightarrow (A_\phi X)^TG\overline{Y} = X^T(A_\phi^TG)\overline{Y} = X^T(G\overline{A}_\phi)\overline{Y}$$
Отсюда
$$A_\phi^TG = G\overline{A}_\phi \eqno(2')$$
Если базис ортонормированный, то $A_\phi^T = \overline{A}_\phi \Longleftrightarrow A = A^*$

Для унитарного оператора:
$$(3) \Longleftrightarrow X^TG\overline{Y} = (A_\phi X)^TG\overline{A_\phi Y} = X^T(A_\phi^TG\overline{A}_\phi)\overline{Y} \Longrightarrow A_\phi^TG\overline{A}_\phi = G  \eqno(3')$$
Если базис ортонормированный, то $A_\phi^T\overline{A}_\phi = E \Longleftrightarrow A^{-1} = A^*$ (унитарная матрица).

\begin{theorem}
    Если $\phi$ - самосопряжённый линейный оператор в $V$, то
    \begin{enumerate}
        \item Все его характеристические корни $\in \R$;
        \item Собственные векторы, соответствующие попарно различным собственным значениям, ортогональны;
        \item Если $U$ - $\phi$-инвариантно в $V$, то $U^\perp$ также $\phi$-инвариантно;
        \item В $V \ \exists$ ортонормированный базис из собственных векторов $\phi \Longleftrightarrow \phi = \phi^*$ ($\Longrightarrow$ при условии, что все собственные значения $\in \R$)
    \end{enumerate}
\end{theorem}
\begin{theorem}
    Если $\phi$ - унитарный линейный оператор в $V$, то 
    \begin{enumerate}
        \item Все собственные значения имеют модуль 1;
        \item Собственные векторы, соответствующие попарно различным собственным значениям, ортогональны;
        \item Если $U$ - $\phi$-инвариантно в $V$, то $U^\perp$ также $\phi$-инвариантно;
        \item В $V \ \exists$ базис из собственных векторов $\phi$, причём в этом базисе
        $$A'_\phi = \begin{pmatrix}e^{i\omega_1}&\null&0\\\null&\ddots&\null\\0&\null&e^{i\omega_n}\end{pmatrix}$$
    \end{enumerate}
\end{theorem}
\begin{proof}
    За исключением примечаний ниже доказательство аналогично случаю евклидова пространства.\\
    \textbf{К пункту 1 обоих теорем:}\\
    Так как $\CC$ замкнуто, любой корень $\lambda$ характеристического многочлена для $\phi$ является собственным значением и имеет отвечающийй ему собственный вектор.\\
    Для самосопряжённого оператора:
    $$\lambda(x,x) = (\phi(x), x) = (x, \phi(x)) = \overline{\lambda}(x,x) \Longrightarrow \lambda \in \R$$
    Для унитарного оператора:
    $$(x,x) = (\phi(x), \phi(x)) = \lambda\overline{\lambda}(x, \phi(x)) \Longrightarrow \lambda\overline{\lambda} = |\lambda|^2 = 1 \Longrightarrow |\lambda| = 1$$
    \textbf{К пункту 4 теоремы 2:}\\
    Индукция по $n$:\\
    База: $n = 1 \Rightarrow \phi(x) = e^{i\omega}x$;
    Шаг: Выберем собственное значение $\lambda_1 = e^{i\omega_1}$, найдём для него собственный вектор $e_1$ и нормируем его. $\langle e_1 \rangle - \phi$-инвариантное подпространство $\Longrightarrow \ \langle e_1 \rangle^\perp - \phi$-инвариантно, и тогда по предположению индукции $\exists$ ортонормированный базис $e_2,...,e_n$ нужного вида для $\phi|_{\langle e_1 \rangle^\perp}$, а из ортогональности $e_1$ всем векторам этого базиса получаем, что $e_1,...,e_n$ - искомый базис.
\end{proof}