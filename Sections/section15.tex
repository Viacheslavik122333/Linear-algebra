\section{Теорема Жордана}
    Основное условие: \ \  $\phi: \ V \to V$ - линейный оператор, все его корни $\in \F$
    $$\chi_{\phi}(\lambda) = (-1)^n(\lambda-\lambda_1)^{k_1}\cdot\ldots\cdot(\lambda-\lambda_s)^{k_s} \ (\forall i\neq j: \ \lambda_i\neq\lambda_j \text{ и } \sum \limits_{i=1}^s k_i = \dim V)$$
    $$V = K_1\oplus\ldots\oplus K_s, \ \text{где } K_i = \text{Ker}(\phi-\lambda_\text{id}d)^{k_i} - \text{корневое подпространство}$$
    $$V_{\lambda_i} = \{x\in V|\phi(x) = \lambda_ix\}, \  \dim V_{\lambda_i}\leqslant k_i = \dim K_i$$
    Так как $K_i$ - инвариантное подпространство относительно оператора $\phi$, можно рассмотреть ограничение: 
    $$(\phi-\lambda_i \text{ id})|_{K_i} := B_i$$
    Из определения $k_i$ следует, что $B_i^{k_i}=0$, то есть $B_i$ - нильпотентный оператор.\\
    Обозначим $h_i$ - показатель нильпотентности оператора, т.е. $B_i^{h_i}=0$,\\ 
    но $B_i^k\neq0$ $\forall k < h_i$\\
    В базисе, согласованном с этим разложением: 
    $$A_{\phi} = \begin{pmatrix}
    \text{\fbox{$A_1$}}\\
    \null & \text{\fbox{$A_2$}}\\
    \null & \null & \ddots\\
    \null & \null & \null & \text{\fbox{$A_s$}}
    \end{pmatrix}$$
    где $A_i = A_{\phi_{k_i}}$ - марица порядка $k_i, \  A_i-\lambda_iE_{k_i} = B_i, \ B_i^{k_i}=0$\\
    Обозначим $K_i :=K$, $B_i :=B$, $k_i :=k$, тогда: 
    $$\forall \bar{x}\in K: \  B^k(x) = 0$$
    если $x\neq0$, то $\exists$ наименьшее значение $m$: 
    $$B^m(x) = 0, \ B^{m-1}(x)\neq_0 \ (m\leqslant h)$$ 
    Назовём это высотой вектора $x$.\\
    Для фиксированного вектора $x\neq0$ (высоты $m$) рассмотрим векторы: 
    $$x, \ B^0x, \ Bx, \ldots,B^{m-1}x, \ B^mx = 0$$
    \begin{definition}
        Векторы \{$x,\ B^0x,\ Bx,\ \ldots,\  B^{m-1}x = 0$\} называются жордановой цепочкой.
    \end{definition}  
    \begin{lemma}
        Вышеуказанные векторы являются линейно независимыми.
    \end{lemma}
    \begin{proof}
        Предположим, что: 
        $$\alpha_0x+\alpha_1B^0x+\ldots+\alpha_{m-1}B^{m-1}x=0$$
        Подействуем на это равенство оператором $B^{m-1}$: $$\alpha_0B^{m-1}x = 0 \ \Longrightarrow \ \alpha_0 = 0$$ 
        На оставшиеся векторы подействуем оператором $B^{m-2}$:
        $$\alpha_1B^{m-1}x = 0 \ \Longrightarrow \ \alpha_1 = 0$$
        и т.д. Получим, что $\forall i = \overline{0,m-1}: \ \alpha_i = 0 \ \Longrightarrow$ векторы являются линейно независимыми.
    \end{proof}
    \begin{definition}
        Подпространство, натянутое на эти векторы: $$\langle x,\ B^0x,\ Bx,\ \ldots,\  B^{m-1}x \rangle$$
        называется циклическим подпространством, порождённым жордановой цепочкой. Данное подпространство обозначим $U$, $\dim U_x = m$.\\
        Обычно векторы жордановой цепочки нумеруют с конца, то есть:
        $$a_1 = B^{m-1}x, \ a_2 = B^{m-2}x, \ldots, a_m = x$$ 
        Тогда $a_1$ - собственный вектор для $B$, и для $\forall j = \overline{2,m}: \ a_{j-1} = Ba_j$\\
        Векторы $a_j$ называются присоединёнными к вектору $a_{j-1}$\\
        К вектору $a_1$: \ $a_2$ - присоединённый, $a_3$ - второй присоединённый и т.д. 
    \end{definition}
    \begin{definition} $\\$ 
        Матрица ограничения оператора $B$ на подпространство $U_x = \langle a_1\ldots a_m\rangle$ : 
        $$B|_{U_x} = \begin{pmatrix}
        0 & 1\\
        \null & 0 & 1\\
        \null & \null & \ddots & \ddots\\
        \null & \null & \null & 0 & 1\\
        \null & \null & \null & \null & 0
        \end{pmatrix} = J_k(0)$$ 
        называется жордановой клекткой с собственным значением $\lambda = 0$
        $$\lambda=\lambda_i: \ A_{\phi|_{U_x}} = \begin{pmatrix}
        \lambda_i & 1\\
        \null & \lambda_i & 1\\
        \null & \null & \ddots & \ddots\\
        \null & \null & \null & \lambda_i & 1\\
        \null & \null & \null & \null & \lambda_i
        \end{pmatrix}=J_k(\lambda_i)$$ 
        - жорданова клектка с собственным значением $\lambda = \lambda_i$, где: 
        $$\phi(a_2) = a_1+\lambda_ia_2, \ \phi(a_{j+1}) = a_j+\lambda_ia_{j+1}$$
    \end{definition} 
    \begin{theorem} \textbf{Жордана} \\
        \tab[0.5cm]Если все характеристические корни опертора $\phi: \ V \to V$ принадлежат полю $\F$, то $V$ является прямой суммой циклических подпространств для оператора $\phi$. Это равносильно тому, что в $V$ существует базис, составленный из жордановых цепочек. Такой базис называется жордановым базисом.\\
        \tab[0.5cm]Если жордановыйй бадис уже построен: Пусть имеются $r$ жордановых цепочек, отвечающих собственным значениям $\lambda_1, \ldots, \lambda_r$, необязательно различным, длины которых $m_1,\ldots,m_r$ соответственно, тогда в этом базисе:
        $$A_{\phi} = \begin{pmatrix}
        \text{\fbox{$J_{m_1}(\lambda_1)$}} & \null & \null & 0 \\
        \null & \text{\fbox{$J_{m_2}(\lambda_2)$}}\\
        \null & \null & \ddots\\
        0 & \null & \null & \text{\fbox{$J_{m_r}(\lambda_r)$}}\\
        \end{pmatrix} \ \ \ \sum \limits_{i=1}^rm_i = n = \dim V$$ 
        - жорданова матрица - жорданова нормальная форма (ЖНФ) матрицы $A_{\phi}$.
    \end{theorem}
    \begin{theorem}\textbf{Жордана (матричная формулировка)} \\
        Для любой матрицы $A \in M_n(\F)$, все характеристические корни которой $\in \F$, $\exists$ матрица $C \ (\det C \neq 0)$ такая, что: 
        $$C^{-1}AC = J$$
        - жорданова матрица. При этом жордановы клетки определены для матрицы $A$ единственным образом с точностью до расположения клеток на диагонали жордановой матрицы.  
    \end{theorem}
    \begin{remark}
        Матрицу $A$ можно интерпретировать как матрицу линейного оператора $\phi$, для него верна теорема Жордана.  
    \end{remark}
    \begin{proof} (См. Шафаревич И.Р. "Линейная алгебра")\\
        Доказательство достаточно провести для ограничения оператора на каждое корневое подпространство $K_i$, в этом случае обозначаем $B = (\phi - \lambda_i \text{id})$,\\ 
        где $B$ - нильпотентный оператор
        % P.S. - Доказательство после колока по Матану разберу и напишу :)
        \begin{lemma}
            Если $B$ - такой оператор в пространстве $V$, что: 
            $$\text{Im}B = B(V) \subset V$$
            то $V$ обладает $(n-1)$-мерным инвариантным подпространством.     
        \end{lemma} 
        Пусть $e_1,...,e_m$ - базис в $\text{Im}B, \ m<n = \dim V$\\
        Дополним его до базиса в $V$ векторами $e_{m+1},...,e_n$, тогда:
        $$\langle e_1,...,e_{n-1} \rangle - \text{инвариантное подпространство:}$$
        $$\forall w = \sum \limits_{i=1}^{n-1}\beta_ie_i \Longrightarrow B_w = \sum \limits_{i=1}^{n-1}\beta_iBe_i \in \text{Im}B \subseteq W$$ 
        $\Longrightarrow $ W - инвариантное подпространство\\
        Ниже будем считать, что $B: \ V \to V$ - нильпотентный оператор, \\
        $\dim V = n, \ W - (n-1)$-мерное инвариантное подпространство в $V$\\
        Будем проводить индукцию по $n$:\\
        Если $n=1$, то $B = 0$ и любой базис - жорданов\\
        Пусть $n>1$, предположение индукции: в $W \ \exists$ базис для $B|_w$\\
        Выберем вектора $a \in V\setminus W$. По предположению индукции:
        $$W = U_1 \oplus ... \oplus U_r$$
        Если вектор $a$ - собственные для $B$, а т.к. он ЛНЗ с векторами из $W$, то: 
        $$V = U_1 \oplus ... \oplus U_r \oplus \langle a \rangle$$ 
        - нужное разложение пространства $V$\\
        Если $a$ - не собственный, то он порождает жорданову цепочку некоторой длины $m$, которая не содержится в $W$.          
    \end{proof}
    \begin{lemma}
        Путсь $U$ - циклическое подпространство, порожденное корневым вектором $e$ высоты $m$. Тогда $\forall y \in U$ представляется в виде:
        $$y = f(B)e, \ \text{где } f(t) - \text{минимальной степени } \leq n-1$$     
    \end{lemma}
    \begin{lemma}
        Если $U = \langle e, \ Be,\ ...,B^{m-1}e \rangle$, то:
        $$\forall y \in U \ \exists \ f(t) \in F[t]: \ y = f(B)e, \ \deg f \leq m-1$$
        Если $f(0) \neq 0$, то высота $y = m$, то есть $y$ пораждает то же самое циклическое подпространство.     
    \end{lemma}
    \begin{proof}
        Пусть $B: \ V \to V$ - нильпотентный оператор, \\ $\dim V = n, \ W - (n-1)$-мерное подпространство, содержащее $\text{Im}B$.\\ Предположение индукции: $W = U_1 \oplus ... \oplus U_r$, т.е. $\forall w \in W$:
        $$w = u_1+... +u_r, \ U_i = \langle e_1, \ Be_i,... \rangle$$
        Выбераем вектор $e \in V\setminus W$, тогда $e$ ЛНЗ с векторами из $W$.\\
        Рассмотрим $Be \in W$ так, что $(*)Be = u_1 + ... + u_r, \ u_i \in U_i$. \\
        Если $Be = 0$, то:
        $$V = \langle e \rangle \oplus U_1 \oplus ... \oplus U_r - \text{то, что нам и надо}$$
        Если $Be \neq 0$, то найдется $i$, что $u_i \neq 0$. Тогда $h(e) = m+1$, т.к. $h(Be) = m$ \\
        (это значит, что $B^{m-1}e = 0, \ B^me \neq 0$ )\\
        Если в разложении $(*) \ u_i \in B(U_i) \Longrightarrow \exists \ v_i: \ u_i = Bv_i$\\
        Рассмотрим вместо $e$ вектор $e-v_i: \ B(e-v_i) = u_1 + ... + \not u_i+...+u_r - \not u_i \Longrightarrow $ в разложение такого вектора $u_i$ не входит.\\
        Заменяя $e$ на нужные разности $e-v_i$, можно считать либо $u_i \not \in B(U_i)$, либо $u_i = 0$  \\
        Хотя бы один из векторов $u_i \neq 0$, выберем из них вектор, имеющий максимальную высоту $m \ (m=\max (\dim U_i))$ \\
        Скажем, пусть это будет вектор $u_1$, тогда $h(e) = m+1$\\
        Докажем, что:
        $$V = \langle e, \ Be, \ ..., \ B^me \rangle \oplus U_2 \oplus ... \oplus U_r$$
        Сумма размерностей подпространств в правой части: 
        $$(m_1+1)+...+m_r = n = \dim V$$
        Достаточно доказать, что: 
        $$\langle e, \ Be, \ ..., \ B^me \rangle \cap (U_2 \oplus ... \oplus U_r) = \{0\}$$
        Пусть $v = \lambda_1 e +...+ \lambda_{m+1}B^me \in U_2 \oplus ... \oplus U_r$ \\
        Т.к. $e \not \in W$, то $\lambda_1 = 0$, но $Be_i = u_1+...+u_r$ - проекция разложения на $U_1 \Longrightarrow $ $$\lambda_2u_1+\lambda_3Bu_1+...+\lambda_{n+1}B^{n-1}u_1 = 0 \Longrightarrow \lambda_2=...=\lambda_{n+1} = 0 \Longrightarrow v_i = 0$$
    \end{proof} 
    \begin{remark}
        $r$ - количество векторов циклического подпространства в разложении корневого подпространства $K$, отвечающего корню $\lambda_0$, равно геометрической кратности корня $\lambda_0$ характеристического многочлена.    
    \end{remark}

    \subsection{Изображение разложения корневых подпространств}
    Обозначим: $r = \dim \text{Ker}B$ - размерность собственного подпространства\\
    Занумеруем собственные векторы, входящии в цепочки, располагая цепочки по убыванию высоты. $m$  - максимальная высота цепочки, $1$ - минимальная\\
    Также введем обозначение для последовательных присоединённых векторов: есть $P_1$ цепочек высоты $m$, \ $P_2$ - высоты $m-1$,..., $r-(P_1+...+p_{r-1})$ - высоты $1$
    \begin{center}
        \begin{asy}
            size(15cm, 0);
            draw((-0.15,0)--(10.5,0)--(10.5,-1)--(-0.15,-1)--cycle);

            draw((0,-0.85)--(0,3.2)--(1,3.2)--(1,-0.85)--cycle);
            draw((2,-0.85)--(2,3.2)--(3,3.2)--(3,-0.85)--cycle);
            draw((3.5,-0.85)--(3.5,2.6)--(4.5,2.6)--(4.5,-0.85)--cycle);
            draw((5.5,-0.85)--(5.5,2.6)--(6.5,2.6)--(6.5,-0.85)--cycle);

            label(" $e$ ", (0.35,-0.5), fontsize(21pt));
            label(" $0$ ", (0.5,-0.31), fontsize(10pt));
            label(" $1$ ", (0.5,-0.67), fontsize(10pt));

            label(" $e$ ", (0.35,0.6), fontsize(21pt));
            label(" $1$ ", (0.5,0.41), fontsize(10pt));
            label(" $1$ ", (0.5,0.76), fontsize(10pt));

            label(" $\vdots$ ", (0.5, 1.3), fontsize(20pt));
            label(" $\vdots$ ", (0.5, 2), fontsize(20pt));
            
            label(" $e$ ", (0.35,2.6), fontsize(21pt));
            label(" $0$ ", (0.5,2.41), fontsize(10pt));
            label(" $m-1$ ", (0.67,2.81), fontsize(10pt));

            label(" $\cdots$ ", (1.5,-0.5), fontsize(21pt));

            label(" $e$ ", (2.35,-0.5), fontsize(21pt));
            label(" $0$ ", (2.5,-0.31), fontsize(10pt));
            label(" $p_1$ ", (2.55,-0.67), fontsize(10pt));

            label(" $e$ ", (3.85,-0.5), fontsize(21pt));
            label(" $0$ ", (4,-0.31), fontsize(10pt));
            label(" $p_1+1$ ", (4.17,-0.69), fontsize(10pt));

            label(" $e$ ", (3.85,0.5), fontsize(21pt));
            label(" $1$ ", (4,0.63), fontsize(10pt));
            label(" $p_1+1$ ", (4.2,0.29), fontsize(10pt));

            label(" $\vdots$ ", (4, 1.3), fontsize(20pt));

            label(" $e$ ", (3.85,2), fontsize(21pt));
            label(" $m-2$ ", (4.2,2.21), fontsize(10pt));
            label(" $p_1+1$ ", (4.17,1.81), fontsize(10pt));

            label(" $\cdots$ ", (5,-0.5), fontsize(21pt));

            label(" $e$ ", (5.85,-0.5), fontsize(21pt));
            label(" $0$ ", (6,-0.3), fontsize(10pt));
            label(" $p_1+p_2$ ", (6.22,-0.69), fontsize(10pt));


            label(" $e$ ", (5.85,0.5), fontsize(21pt));
            label(" $1$ ", (6,0.64), fontsize(10pt));
            label(" $p_1+p_2$ ", (6.22,0.29), fontsize(10pt));

            label(" $\vdots$ ", (6, 1.3), fontsize(20pt));

            label(" $e$ ", (5.85,2), fontsize(21pt));
            label(" $m-2$ ", (6.2,2.21), fontsize(10pt));
            label(" $p_1+p_2$ ", (6.22,1.81), fontsize(10pt));

            label(" $\cdots$ ", (7,-0.5), fontsize(19pt));

            draw((7.5,-0.85)--(7.5,-0.15)--(8.5,-0.15)--(8.5,-0.85)--cycle);
            label(" $e$ ", (7.85,-0.5), fontsize(19pt));
            label(" $0$ ", (8,-0.31), fontsize(10pt));
            label(" $p_1+...+p_{r-1}$ ", (8.3,-0.67), fontsize(7pt));
 

            draw((9,-0.85)--(9,-0.15)--(10,-0.15)--(10,-0.85)--cycle);
            label(" $e$ ", (9.35,-0.5), fontsize(19pt));
            label(" $0$ ", (9.5,-0.31), fontsize(10pt));
            label(" $p_1+...+p_r$ ", (9.8,-0.67), fontsize(7pt));

            label(" $r=p_1 + ... + p_r$ ", (8.8,1), fontsize(14pt));
        \end{asy}
    \end{center}
    $V = U_1 \oplus U_r, \ \dim U_{i+1}\leq \dim U_i$
    \begin{center}
        $BV = BU_1 \oplus ... \oplus BU_r$\\
        $\vdots \tab[4cm]$ \\ 
        $B^kV = B^kU_1 \oplus ... \oplus B^kU_r$  
    \end{center}
    Если $\dim U_i = m_i, \ \dim (B_kU_i) = \left[\begin{matrix}
        m_i - k, \ \text{если } k<m_i\\
        0, \ \text{если } k\geq m_i
    \end{matrix} \right. \Longrightarrow$
    $$\dim (B^kV) = \sum \limits_{i=1}^r \dim B^kU_i = q_{k+1}+2q_{k^2} + ... + (m-1)q_m$$ 
    Пусть $q_i$ - число циклических подпространств размерности $i$, \ $1\leq i \leq r$\\
    Обозначим $r_k = \text{rk}B^k$\\
    Для $k = 0$ до $m-1$ получим равенства: 
    \begin{center}
        $k=0: \ q_1 + 2q_2 + ... + mq_m = n\tab[2.9cm]$\\
        $k=1: \ q_2 + 2q_3 + ... + (m-1)q_m = r_1 = \text{rk}B$\\
        $\cdots$\\
        $q_m = r_{m-1} = \text{rk}B^{m-1} \neq 0$ \\
        $B^m = 0$ на корневом подпространстве     
    \end{center}
    Вычитая их каждого уравнения слудующее, получим систему:
    $$\begin{cases}
        q_1 + q_2 + ... + q_m = n - r_1 \tab[0.13cm] \Longrightarrow q_1 = n - 2r_1 + r_2 \\
        \tab[1.05cm] q_2 + ... + q_m = r_1 - r_2 \Longrightarrow q_2 = r_1 - 2r_2 + r_3\\
        \tab[3.5cm]\cdots\\
        \tab[6.9cm]q_m = r_{m-1} - r_m \ (r_m = 0)
    \end{cases}$$
    $$\Longrightarrow q_i = r_{i-1} - 2r_i + r_{i+1} \ (i = 1,...,m-1)$$
    Вывод: количество и порядок (высоты цепочек) однозначно опреледяется по матрице $B=A|_{\phi-\lambda \text{id}}$ - эти ранги не зависят от конкретного разложения $\Longrightarrow$ определяются единственным образом.
    \begin{consequense}
        Пусть: 
        $$\chi_\phi = (-1)^n(\lambda-\lambda_1)^{k_1}\cdot ... \cdot (\lambda-\lambda_s)^{k_s}$$
        - характеристический многочлен
        $$\mu_\phi = (\lambda-\lambda_1)^{m_1}\cdot ... \cdot (\lambda-\lambda_s)^{m_s}$$
        - минимальный многочлен\\
        Тогда $\forall i = \overline{1,s}: \ m_i$ равна $\max$ размерности жордановой клетки, отвечающей корню $\lambda_i$   
    \end{consequense}
    \begin{consequense}\textbf{ Критерий диагонализируемости в терминах min многочлена:} \\
        Оператор $\phi$ диагонализируем $\Longleftrightarrow m_1 = ... =m_s=1$  
    \end{consequense}
    \begin{proof}
        Достаточно доказать для каждого корневого подпространства $k_i$
        $$B = A_{\phi-\lambda_i \text{id}}|_{k_i}$$
        - блочно-диагональная матрица с клетками размера $m_j$   
    \end{proof}  

    
