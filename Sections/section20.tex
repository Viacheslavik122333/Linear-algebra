%chktex-file 1 %chktex-file 3 %chktex-file 8 %chktex-file 9 %chktex-file 10 %chktex-file 11 %chktex-file 12 %chktex-file 13 %chktex-file 16 %chktex-file 17 %chktex-file 18 %chktex-file 25 %chktex-file 26 %chktex-file 35 %chktex-file 36 %chktex-file 37 %chktex-file 40 %chktex-file 44 %chktex-file 45 %chktex-file 49
\section{Полуторалинейные, эрмитовы формы. Унитарные (эрмитовы) пространства}
Далее всюду $F = \CC, V$ - в.п. над $\CC$.
\begin{definition}
    Функция $f(x, y): V\times V \rightarrow \CC$ наз. полуторалинейной, если:
    \begin{enumerate}
        \item $f(x_1 + x_2, y) = f(x_1, y) + f(x_2, y)$;\\
              $f(\lambda_x, y) = \lambda f(x, y) \ (\lambda \in \CC)$;
        \item $f(x, y_1 + y_2) = f(x, y_1) + f(x, y_2)$;\\
              $f(x, \lambda y) = \overline{\lambda} f(x, y) \ (\lambda \in \CC)$
    \end{enumerate}
\end{definition}
\begin{definition}
    $f(x,y)$ наз. эрмитово симметричной (эрмитовой), если 
    \begin{enumerate}
        \item $f(x, y)$ линейна по $x$;
        \item $f(y, x) \equiv \overline{f(x,y)} \ (\Longrightarrow f(x, \lambda y) = \overline{\lambda}f(x,y) \ \forall \lambda \in \CC)$
    \end{enumerate}
\end{definition}
Заметим, что если $f(x,y)$ эрмитова, то $f(x, x) \equiv \overline{f(x,x)} \Rightarrow f(x,x) \in \R$.
\begin{definition}
    Квадратичная функция, порождённая эрмитовой формой - это функция $k(x) \equiv f(x,x)$.
\end{definition}
\begin{exercise}
    Доказать, что для любой квадратичной формы $k(x) \ \exists!$ эрмитова форма $f(x, y)$ такая, что $f(x,x) \equiv k(x)$. 
\end{exercise}
Если $f(x,y)$ полуторалинейна и эрмитова, то обозначим $F = (f(e_i, e_j))$, и тогда $f(e_j, e_i) = \overline{f(e_i, e_j)} \Longrightarrow F^T = \overline{F} \Longleftrightarrow \overline{F}^T = F$.
\begin{definition}
    $F^* = \overline{F}^T$ - эрмитово сопряжённая матрица к $F$.\\
    Если $F^* = F$, то $F$ - эрмитова матрица.
\end{definition} 
\begin{definition}
    Скалярное произведение на пр-ве $V$ - функция $(x,y)$ такая, что
    \begin{enumerate}
        \item $(x,y)$ линейна по $x$;
        \item $(y,x) \equiv \overline{(x,y)}$;
        \item $(x,x) > 0 \ \forall x \neq 0$
    \end{enumerate}
\end{definition}
Скалярное произведение в координатах:
$$(\sum \limits_{k=1}^n x_ke_k, \sum \limits_{j=1}^n y_je_j) = \sum \limits_{k=1}^n x_k(e_k, \sum \limits_{j=1}^n y_je_j) = \sum \limits_{k, j=1}^n x_k\overline{y_j}(e_k, e_j)$$
Матрица Грама базиса $e$: $G_e = ((e_k, e_j))$. $G_e* = \overline{G_e}^T = G_e$.
\begin{definition}
    $x \perp y \Longleftrightarrow (x, y) = 0$.\\
    Базис $e_1,...,e_n$ ортогональный, если $(e_k, e_j) = 0, k \neq j$.\\
    Базис $e_1,...,e_n$ ортонормированный, если $(e_k, e_j) = \delta_{ij}$.
\end{definition}
В ортонормированном базисе $(x, y) = \sum \limits_{j=1}^n x_j\overline{y_j}$.