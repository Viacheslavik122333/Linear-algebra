%chktex-file 1 %chktex-file 3 %chktex-file 8 %chktex-file 9 %chktex-file 10 %chktex-file 11 %chktex-file 12 %chktex-file 13 %chktex-file 16 %chktex-file 17 %chktex-file 18 %chktex-file 25 %chktex-file 26 %chktex-file 35 %chktex-file 36 %chktex-file 37 %chktex-file 40 %chktex-file 44 %chktex-file 45 %chktex-file 49
\section{Аффинные пространства и их преобразования}
\begin{definition}
    Аффинным пространством над полем $\F$ называется пара $\\(\A, V)$, где $\A$ - множество точек, $V$ - ассоциированное с ним векторное пространство (над $\F$), если задано отображение $\A \times V \rightarrow \A$ - операция прибавления вектора к точке (откладывание вектора от точки) со следующими аксиомами:
    \begin{enumerate}
        \item $\forall p \in \A, x,y \in V : \ p + (x + y) = (p + x) + y$;
        \item $\forall p \in \A : \ p + 0 = p$;
        \item $\forall p, q \in \A \ \exists! x \in V : p + x = q$
    \end{enumerate}
    Размерность аффинного пространства: $\dim \A = \dim V$.
\end{definition}
\begin{remark}
    Если имеется векторное пространство $V$, то можно принять $\A = V$, понимая точки как радиус-векторы, если задать начальную точку $0 \in V$.
\end{remark}

\begin{subtheorem}
    $\overrightarrow{pq} + \overrightarrow{qs} = \overrightarrow{ps}$
\end{subtheorem}
\begin{proof}
    $q = p + x; \ s = q + y = (p + x) + y = p + (x + y)$
\end{proof}

\subsection*{Аффинная система координат}
Задаётся точкой $o \in \A$ - началом координат и базисом $e$ ассоциированного векторного пространства $V$. Обозначается $(o, e_1,...,e_n)$.\\
Координаты точки $p$ - координаты радиус-вектора $\overrightarrow{op}$ в базисе $e$.
$$\overrightarrow{op} = \sum \limits_{i=1}^n x_ie_i \Longrightarrow \overrightarrow{pq} = \overrightarrow{oq} - \overrightarrow{op} = \sum \limits_{i=1}^n (y_i - x_i)e_i$$
Также можно задать точки $o, p_1,...,p_n$ общего положения (т.е. $\overrightarrow{op_1},...,\overrightarrow{op_n}$ линейно независимы) - тогда $(o, \overrightarrow{op_1},...,\overrightarrow{op_n})$ - система координат.

\subsection*{Изменение координат точки при замене системы координат}
Пусть $(o, e)$ - старая система координат, $(o, e')$ - новая система координат.\\
Заметим, что $\overrightarrow{op} = \overrightarrow{oo'} + \overrightarrow{o'p}$. Поэтому если $X$ - столбец координат точки $p$ в старых координатах, $X'$ - в новых координатах, а $X_o$ - столбец старых координат точки $o'$, то $$X = X_o + CX, \ \ (C = C_{e\rightarrow e'}) \eqno(1)$$
Можно ввести аффинную матрицу перехода $\tilde{C} = \begin{pmatrix} C&X_0 \\ 0&1 \end{pmatrix}$ порядка $n + 1$ ($n = \dim V$) и дополненный столбец $\tilde{X} = \begin{pmatrix} X \\ 1 \end{pmatrix}$ высоты $n + 1$. Тогда из $(1)$:
$$\tilde{X} = \tilde{C}\tilde{X}' \eqno(1')$$

\subsection*{Барицентрическая комбинация точек}
Пусть даны $p_0,p_1,...,p_m (1 \leqslant m \leqslant n)$ с коэффициентами $\lambda_0,\lambda_1,...,\lambda_m, \ \sum \lambda_i = 1$.\\
Барицентрической комбинацией будем называть 
$$\sum \limits_{i=0}^m \lambda_ip_i := p + \sum \limits_{i=0}^m \lambda_i \overrightarrow{pp_i} = p + \sum \limits_{i=0}^m \lambda_i(p_i - p)\text{ для некоторой точки } p$$
Покажем, что результат не зависит от выбора точки $p$: если $q = p + v$ - другая точка, то:
$$q + \sum \limits_{i=0}^m \lambda_i \overrightarrow{qp_i} = p+ v + \sum \limits_{i=0}^m \lambda_i(-v) + \sum \limits_{i=0}^m \lambda_i \overrightarrow{pp_i} = p + \sum \limits_{i=0}^m \lambda_i \overrightarrow{pp_i}$$

\begin{consequense}
    Если $m = n$ и $p_0,...,p_n$ - точки общего положения, то любую точку можно единственным образом представить в виде барицентрической комбинации этих точек: $p = \sum \limits_{i=0}^m x_ip_i, \ \sum \limits_{i=0}^m x_i = 1$.\\
    $(x_0,...,x_n)$ называются барицентрическими координатами точки $p$.
\end{consequense}
\subsection{Аффинные плоскости (подпространства)}
\begin{definition}
    Зафиксируем точку $p_0 \in \A$ и подпространство $U \subseteq V$.\\
    Аффинная плоскость $P$ с начальной точкой $p_0$ и направляющим подпространством $U$ - это множество точек $P := p_0 + U = \{p_0 + u | u \in U\}$.\\
    Размерность плоскости: $\dim P = \dim U$.
\end{definition}
\begin{subtheorem}
    $P$ не зависит от выбора точки $p_0$.
\end{subtheorem}
\begin{proof}
    Пусть $P = p_0 + U, p_0' \in P$. Тогда:
    $$p_0' = p_0 + u_0, u_0 \in U \Longrightarrow P' = p_0' + U = p_0 + u_0 + U = p_0 + U = P$$
\end{proof}
\begin{subtheorem}
    Если $P = p_0 + U = p'_0 + U'$, то $U = U'$ (т.е. направляющее подпространство для плоскости определено однозначно).
\end{subtheorem}
\begin{proof}
    $p_0' \in p_0 + U \Longrightarrow p_0 + U = p_0' + U = p_0' + U' \Longrightarrow U = U'$
\end{proof}
\begin{subtheorem}
    $(P, U)$ является аффинным пространством относ. операции $p \rightarrow p + x$ для $x \in U$.
\end{subtheorem}
\begin{proof}
    Проверим аксиомы:
    \begin{enumerate}
        \item $p + u \in p + U$ - операция определена на $P$ и $U$;
        \item $p + (u_1 + u_2) = (p + u_1) + u_2 \in p' + U = P$;
        \item Если $p, q \in P$, то $P = p + U, q = p + u \Longrightarrow \overrightarrow{pq} = u \in U$ - существует и единственный.
    \end{enumerate}
\end{proof}

$\forall p \in P: \ p = p_0 + \sum \limits_{i=1}^m x_ie_i$ ($e_1,..,e_m$ - базис в $U$).\\
Вместо точки $p_0$ и базиса $e_1,...,e_m$ можно рассмотреть точки $p_0,p_1,...,p_m$ общего положения - любую точку $p \in P$ можно представить в виде барицентрической комбинации точек $p_0,...,p_n$. 

\subsection*{Задание аффинной плоскости неоднородной СЛУ}
Пусть $P = p_0 + U, \dim U = m, \dim V = n$. Тогда $\exists$ матрица $A$ такая, что: 
$$U = \{x = eX | AX = 0\} \ (e - \text{ базис } V)$$
$\forall p \in P$ имеет координаты $X_0 + X$, где $X_0$ - столбец координат $p_0$, а $X$ - координаты $u \in U$.
Тогда:
$$b: = A(X_0 + X) = AX_0 + AX = AX_0 $$ 
$\Longrightarrow$ координаты $p \in P$ удовлетворяют системе $AX = b$.\\
Если $p_0$ заменить на $p'_0$ с координатами $X_0 + X', AX' = 0$, то: 
$$A(X_0 + X') = AX_0 = b$$ 
Остюда получаем следующее утверждение:
\begin{subtheorem}
    Любую аффинную плоскость можно задать (неоднородной) системой линейных уравнений.
\end{subtheorem}
\begin{definition}
    Аффинная оболочка множества точек $M$ - это наименьшая по включению аффинная плоскость, содержащая все точки $M$. В частности, если 
    $$M = \{p_0,...,p_k\} \text{ то }  \langle M \rangle = p_0 + \langle \overrightarrow{p_0p_1},...,\overrightarrow{p_0p_k} \rangle$$
\end{definition}

\begin{remark}
    Аффинная плоскость $P=p_0+U$ представляет собой некоторый смежный класс пространства $V$ по $U$:
    \[p_0'+U=p_0+U=P \Longleftrightarrow  \overline{p_op_0'}\in U\]
\end{remark}
\subsection*{Взаимное расположение двух плоскостей:}
Пусть $P_1=p_1+U_1,\ P_2=p_2+U_2$
\begin{enumerate}
    \item $P_1 \parallel P_2$ (в широком смысле), если $U_1\subseteq U_2$ или $U_2\subseteq U_1$. В истинном смысле: если они параллельны в широком смысле и не пересекаются. 
    \item $P_1 \cap P_2\ne \emptyset$, но не параллельны.
    \item $P_1$ и $P_2$ скрещиваются: $P_1\cap P_2=\emptyset$ и $U_1\cap U_2 =\{0\}$.
\end{enumerate}
\begin{subtheorem}
    $P_1\cap P_2\ne \emptyset \Longleftrightarrow \overline{p_1p_2}\in U_1+U_2$
\end{subtheorem} 
\begin{proof} \tab
    \begin{itemize}
        \item[$\underline{\Longrightarrow}$] Пусть $p=p_1+u_1=p_2+u_2 \Rightarrow \overline{p_1p_2}=u_1-u_2\in U_1+U_2$
        \item[$\underline{\Longleftarrow}$] Пусть существуют $u_i\in U_i,\ i=1,2: \overline{p_1p_2}=u_1-u_2$. 
        Значит:
        \[p_1+u_1=p_2+u_2\in P_1\cap P_2\]
    \end{itemize}
\end{proof}

\begin{definition}
    Аффинная оболочка подмножества $M\subset \mathbb{A}$ - это
    \[\text{Aff}(M)\equiv \langle M \rangle:=p_0+\langle\overline{pq} \ | \ p,q\in M\rangle,\ p_0\in M\]
    Видно, что $\langle M \rangle$ - аффинная плоскость с направляющим подпространством  
    \[U_0=\langle\overline{pq}: p,q\in M\rangle\]
    Если $P=p_0+U\supseteq M \Longrightarrow P\ni p_0+\overline{pq},\ p,q\in M \Longrightarrow P\supseteq \langle M \rangle$.
    Если $P_1, P_2$ - аффинные плоскости, то:
    \[\langle P_1,P_2 \rangle=p_0+\langle \overline{p_1p_2}, U_1, U_2 \rangle\]
\end{definition} 
\begin{theorem}
    \[\dim{\langle P_1,P_2 \rangle}=\begin{cases}
        \dim(U_1+U_2),\ \text{если}\ P_1\cap P_2\ne \emptyset,\\
        \dim(U_1+U_2)+1,\ \text{если}\ P_1\cap P_2= \emptyset
    \end{cases}\]
\end{theorem}
\begin{proof}
    $\langle P_1,P_2 \rangle$ имеет направляющее подпространство:
    \[\langle \overline{p_1p_2}, U_1, U_2 \rangle,\ \forall p_1\in P_1,\ p_2\in P_2\]
    \[\dim{\langle \overline{p_1p_2}, U_1, U_2 \rangle}=\begin{cases}
        \dim(U_1+U_2),\ \text{если}\ \overline{p_1p_2}\in U_1+U_2 \Longleftrightarrow P_1\cap P_2\ne \emptyset,\\
        \dim(U_1+U_2)+1,\ \text{если}\ \overline{p_1p_2}\not\in U_1+U_2.
    \end{cases}\]
\end{proof}





