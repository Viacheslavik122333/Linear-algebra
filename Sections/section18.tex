%chktex-file 1 %chktex-file 3 %chktex-file 8 %chktex-file 9 %chktex-file 10 %chktex-file 11 %chktex-file 12 %chktex-file 13 %chktex-file 16 %chktex-file 17 %chktex-file 18 %chktex-file 25 %chktex-file 26 %chktex-file 35 %chktex-file 36 %chktex-file 37 %chktex-file 40 %chktex-file 44 %chktex-file 45 %chktex-file 49
\section{Общие линейные операторы в евклидовом пространсве}
\begin{theorem}
    Любой невырожденный линейный оператор $\phi$ в евклидовом пространсве $\E$ единственным образом может быть представлен в виде: $\phi = \theta \cdot \rho$, где $\theta$ - ортогональный оператор и $\rho$ - самосопряжённый оператор с положительными собственными значениями. 
\end{theorem}
\begin{theorem} \textbf{Матричная версия}\\
    Любую вещественную матрицу $A$ с $\det A \neq 0$ можно представить в виде произведения $A = Q \cdot R$, где $Q$ - ортогональная, $R$ - симметричная с положительными собственными значениями.      
\end{theorem}
\begin{lemma}
    Если оператор $\phi: \ \E \to \E$ невырожденный, то все собственные значения оператора $\phi^* \cdot \phi$ положительны.  
\end{lemma} 
\begin{subtheorem}
    $$(\psi \cdot \phi)^* = \phi^* \cdot \psi^*$$ 
\end{subtheorem} 
 \begin{proof}
    Оператор $\phi^* \cdot \phi$ - самосопряжённый: 
    $$(\phi^* \cdot \phi)^* = \phi^* \cdot (\phi^*)^* = \phi^* \cdot \phi$$
    $\Longrightarrow $ все его собственые значения $\in \R$. Путсь $\mu$ - какое-то их них: $(\phi^* \cdot \phi)(v) = \mu v$ для подходящего $v \neq 0$. Вычислим $\mu$: 
    $$((\phi^* \cdot \phi)(v),v) = \mu(v,v) = (\phi(v),(\phi^*)^*(v)) = (\phi(v), \phi(v)) \Longrightarrow \mu = \frac{(\phi(v),\phi(v))}{(v,v)}$$
    $\Longrightarrow \mu>0$     
 \end{proof} 
 \begin{remark}
    Для любой вещественной матрицы $A$ она является $A = A_\phi$ в подходящем базисе (этот базис можно выбрать ортонормированным). Будем доказывать матричную версию, используя тот факт, что в ортонормированном базисе: $A_{\phi^*} = A_\phi^T$  
 \end{remark} 