%chktex-file 1 %chktex-file 3 %chktex-file 8 %chktex-file 9 %chktex-file 10 %chktex-file 11 %chktex-file 12 %chktex-file 13 %chktex-file 16 %chktex-file 17 %chktex-file 18 %chktex-file 25 %chktex-file 26 %chktex-file 35 %chktex-file 36 %chktex-file 37 %chktex-file 40 %chktex-file 44 %chktex-file 45 %chktex-file 49
\section{Билинейные и квадратичные формы}
\begin{definition}
    Функция $b: \ V\times V \to \F$ называется билинейной функцией, если:
    \begin{enumerate}
        \item аддитивность: $$\forall x_1, x_2, y: \ b(x_1+x_2, y) = b(x_1, y)+ b(x_2, y)$$
        $$\forall x, y_1, y_2: \ b(x, y_1 + y_2) = b(x, y_1)+ b(x, y_2)$$
        \item однородность: 
        $$\forall x,y \in V , \  \forall \lambda \in \F: \ b(\lambda x, y) = \lambda b(x, y) = b(x, \lambda y)$$ 
    \end{enumerate}
\end{definition}
\begin{definition}
    $b(x, y)$ - называется симметрической, если: 
    $$\forall x, y \in V: \ b(y, x) = b(x, y)$$ 
\end{definition}
\begin{example}\tab
    \begin{enumerate}
        \item Симметрическая билинейная функция - скалярное произведение
        \item $V = M_n(\F): \ b(X,Y) = tr(XY)$
        \item $\beta(f, g) = \int \limits_a^b f(x)g(x)dx$  
    \end{enumerate}
\end{example}
\subsection{Запись билинейной функции в координатах}
Пусть в $V$ задан базис $e_1,...,e_n$, тогда:
$$b(\sum \limits_{i=1}^nx_ie_i, \sum \limits_{j=1}^ny_je_j) = \sum \limits_{i,j=1}^nb(x_ie_i, y_je_j) = \sum \limits_{i,j=1}^n x_iy_jb(e_i,e_j)$$
\begin{definition}
    Обозначим $b_{ij} = b(e_i, e_j)$, тогда $B_e=b_{ij}$ - матрица билинейной функции $b(x, y)$ в базисе $e$\\
    Тогда:
    $$b(x, y) = \sum \limits_{i,j=1}^n x_ib_{ij}y_j = \begin{pmatrix}
        x_1 & \cdots & x_n
    \end{pmatrix}B_e\begin{pmatrix}
        y_1\\
        \vdots \\
        y_n
    \end{pmatrix} = X^TB_eY \eqno (1)$$ 
    
\end{definition} 
\subsection{Изменение матрицы билинейной формы при замене базиса}
Пусть $e' = eC$, т.е. $C$ - матрица перехода от $e$ к $e'$\\
Тогда: 
$$X = CX', \ Y = CY' \eqno (2)$$
По определению матрицы билинейной функции, в новом базисе: 
$$b(x, y) = X'^TB'Y' \ \ (B' = B_{e'})$$ 
Подставим в формулу $(1)$ выраженеие $(2)$:
$$b(x, y) = X'^TC^TBCY' = X'^T(C^TBC)Y' = X'^TB'Y' \ \ (\forall X',Y' \in \F^n)$$
$$\Longrightarrow B' = C^TBC \ \ (\forall i,j: \ X':= E_i, \ Y':=E_j)$$    
\begin{consequense}\tab
    \begin{enumerate}
        \item $\text{rk}B' = \text{rk}B$
        \item $\F = \R \Longrightarrow \sgn (\det B') = \sgn (\det B)$  
    \end{enumerate}
\end{consequense} 
\begin{definition}
    Билинейная функция $b(x,y)$ называется кососимметрической (при char $\F \neq 2$), если: 
    $$\forall x, y \in V: \ b(x, y) = -b(y, x)$$  
\end{definition} 
\begin{subtheorem} $(*)$ 
    Любая билинейная функция над $\F: \ \text{char}\F \neq 2$ единственным образом представляется в виде:
    $$b(x, y) = b_+(x, y) + b_-(x, y), \ \text{ где } b_+(x, y) \equiv b_+(y, x), \ b_-(x, y) \equiv -b(y,x)$$  
\end{subtheorem}
\begin{proof}
    $$\begin{cases}
        b(x, y) = b_+(x, y) + b_-(x, y)\\
        b(y, x) = b_+(x, y) - b_-(x, y)
    \end{cases} \Longrightarrow $$
    $$b_+(x,y) = \frac{b(x,y)+b(y,x)}{2}, \ b_-(x,y) = \frac{b(x,y)-b(y,x)}{2}
    $$ 
\end{proof} 

\begin{subtheorem}
    Билинейная функция $b(x,y)$ симметрична (кососимметрична) $\Longleftrightarrow$ в любом базисе $e$: 
    $$B_e^T=B_e \ (B_e^T=-B_e)$$
\end{subtheorem}
\begin{proof} (Докажем для симметрической, для кососимметрической \\аналогично)
    \begin{itemize}
        \item[ $\underline{\Longrightarrow}$] Пусть $B = (b_{ij})$, тогда $b_{ij}=b(e_i, e_j)$.
        $$\forall x, y\in V, \ b(x,y)= b(y,x) \Longrightarrow  
        b(e_j, e_i) = b(e_i, e_j)$$
        \item[$\underline{\Longleftarrow}$]  
        $$b(x,y)= X^TBY, \ b(y,x) = Y^TBX = (X^TB^TY)^T = (X^TBY)^T = b(x,y)$$
    \end{itemize}
\end{proof}
  Утверждение (1) $\Longleftrightarrow$ $\forall$ матрицы $B$ некоторой билинейной функции верно, что $B = B_++B_-$, где $B_+$ - матрица симметрической билинейной функции, а $B_-$ - матрица кососимметрической билинейной функции.
\begin{definition}
    Квадратичная функция, порождённая билинейной функцией $b(x,y)$ - это функция на $V$. \\
    Обозначаем: $k(x):=b(x,x)$, если $k(x)\not\equiv0$.
\end{definition}
    Если $b$ - кососимметрическая функция, то $b(x,x)=0$ $\Longrightarrow$ $k(x)\equiv0$. В общем случае существует бесконечно много билинейных функций, порождающих одну и ту же квадратичную, таких, что: 
    $$b(x,y)=b_+(x,y)+b_-(x,y) \Longrightarrow  b(x,x)=b_+(x,x)$$
\begin{theorem}
    $\forall$ квадратичной функции $\exists!$ симметрическая билинейная функция, которая её порождает.
\end{theorem}
\begin{proof}
    Допустим, что $b(x,y) = b(y,x)$ - симметрическая билинейная функция и $k(x) = b(x,x)$. Тогда $\forall x, y\in V$:
    \begin{multline*}
        k(x+y) = b(x+y, x+y) = b(x,x)+b(x,y)+b(y,x)+b(y,y)=\\ 
        = b(x,x)+2b(x,y)+b(y,y) = k(x)+2b(x,y)+k(y)
    \end{multline*}
    Так как $\text{char} \F \neq 2$, то: 
    $$b(x,y)=\frac{k(x+y)-k(x)-k(y)}{2}$$
\end{proof}
\begin{definition}
    Билинейная функция $b(x,y) = \frac{k(x+y)-k(x)-k(y)}{2}$ называется поляризацией квадратичной функции $k$.
\end{definition}
    Далее будем считать матрицу квадратичной формы матрицей её полярной симметрической билинейной функции $b(x,y)$
    $$b(x,y)=\sum\limits_{i=1}^nb_{ii}x_iy_i+\sum\limits_{i<j}b_{ij}x_iy_j+\sum\limits_{i>j}b_{ij}x_iy_j$$
    $$\forall i, j: \ b_{ij}=b_{ji} \Longrightarrow  b(x,x)=k(x)=\sum\limits_{i=1}^nb_{ii}x_i^2+2\sum\limits_{1\leqslant i<j\leqslant n}b_{ij}x_ix_j\eqno{\textbf{(1)}}$$

\begin{example1}
    Пусть $k(x_1,x_2, x_3)=3x_1^2+2x_1x_2-x_1x_3+x_2^2+6x_2x_3-7x_3^2$, тогда:
        $$B=\begin{pmatrix}
        3 & 1 & -\frac{1}{2}\\
        1 & 1 &  3\\
        -\frac{1}{2} &  3 & -7
        \end{pmatrix}$$
\end{example1}
\begin{definition}
    Пусть $b(x,y)$ - симметрическая или кососимметрическая билинейная функция и $\varnothing \neq L \subset V$ - подпространство. Ортогональным дополнением к $L$ относительно билинейной формы $b(x,y)$ называется: 
    $$L^{\perp}:=\{y\in V \ | \ b(x,y)=0, \ \forall x\in L\}$$
\end{definition}
\begin{remark}
    Запись $x\perp y$ означает, что $b(x,y)=0$.
\end{remark}
\begin{definition}
    $V^{\perp}=\{y\in V \ | \ b(x,y)=0, \ \forall x\in V\}$ - ядро формы.
\end{definition}
\begin{definition}
    Билинейная функция $b(x,y)$ называется невырожденной, если: 
    $$\text{Ker}(b)=V^{\perp}=\{0\}$$
\end{definition}
\begin{exercise}
    $b(x,y)$ - невырожденная функция $\Longleftrightarrow$ $\det B\neq0$.
\end{exercise}
\subsection{Квадратичные формы}
\begin{definition}
    Квадратичная форма в некотором базисе называется диагональной, если в этом базисе: 
    $$k(x_1, \ldots, x_n)=\sum\limits_{i=1}^n\alpha_ix_i^2, \text{ где } \alpha_i\in\F$$
\end{definition}
\begin{theorem}
    В конечномерном пространстве $V$ ($\textup{char }\F\neq2$) $\exists$ базис, в котором эта форма диагональна.
\end{theorem}
\begin{proof} (Алгоритм Лагранжа - метод выделения полных квадратов)\\
    По формуле \textbf{(1)}: 
    $$k(x)=\sum\limits_{i=1}^nb_{ii}x_i^2+2\sum\limits_{i<j}b_{ij}x_ix_j$$
    \begin{enumerate}
        \item Основной случай: \\
        $\exists i$ : $b_{ii}\neq0$ $\Longrightarrow$ можно перенумеровать неизвестные $x_1$, $\ldots$, $x_n$ так, что $b_{11}\neq0$. Выделим в $k(x)$ все одночлены, содержащие $x_1$:
        $$k(x)=b_{11}x_1^2+2x_1\sum\limits_{i=2}^nb_{1i}x_i+\widetilde{k}(x_2,\ldots, x_n)$$ 
        и дополним выражение до квадрата:
        \begin{multline*}
            k(x) = b_{11}(x_1^2+2x_1\sum\limits_{i=2}^n\frac{b_{1i}}{b_{11}}x_i+(\sum\limits_{i=2}^n\frac{b_{1i}}{b_{11}}x_i)^2)-\frac{(\sum\limits_{i=2}^nb_{1i}x_i)^2}{b_{11}}+\widetilde{k} = \\
            =b_{11}(x_1+\sum\limits_{i=2}^n\frac{b_{1i}}{b_{11}}x_i)^2+k_2(x_2, \ldots,x_n)
        \end{multline*}
        Затем для формы $k_2(x_2,\ldots, x_n)=\sum\limits_{i=2}^nb_{ii}'x_i^2+\sum\limits_{2\leqslant i<j\leqslant n}b_{ij}'x_ix_j$ найдём коэффициент $b_{jj}'\neq0$ и выделим квадрат как на предыдущем шаге. На каждом шаге число переменных уменьшается на единицу, а значит, за конечное число шагов (а именно $\leqslant n-2$) форма приобретёт диагональный вид.
        \item Особый случай: \\
        $\forall i: \ b_{ii}=0$, но так как $k(x)\not\equiv0$ $\Longrightarrow$ $\exists$ индексы $i$ и $j$ такие, что $b_{ij}\neq0$, то есть в выражение $k(x)$ входит одночлен $2b_{ij}x_ix_j$.\\
        Пусть $x_i=x_i'+x_j'$ и $x_j=x_i'-x_j'$, тогда $x_ix_j = x_i'^2-x_j'^2$, то есть появился квадрат с коэффициентом, не равным нулю $\Longrightarrow$ можно перейти к общему случаю. (Квадраты появятся только в этом одночлене, т.к. $x'_i$ и $x'_j$ ни в одном другом не встретятся дважды, поэтому и после приведения подобных коэффициенты перед ними будут ненулевые)
    \end{enumerate}
\end{proof}
\begin{remark}
    В благоприятном случае, когда на первом шаге коэффициент при $x_1$ не равен нулю, на втором шаге коэффициент при $x_2$ не равен нулю и т.д., матрица замены будет иметь вид:
    $$C_{e\rightarrow e'}^{-1}=\begin{pmatrix}
    1 & \frac{b_{12}}{b_{11}} & \ldots & \frac{b_{1n}}{b_{11}}\\
    0 & 1 & \ldots & \frac{b_{2n}}{b_{22}}\\
    \vdots & \null & \ddots & \vdots\\
    0 & 0 & \ldots & 1
    \end{pmatrix}$$ 
    - матрица с 1 на диагонали $\Longrightarrow$ $|C_{e\rightarrow e'}^{-1}|=1\neq0$.
\end{remark}
\begin{definition}
    Форма $k(x_1,\ldots,x_n)$ называется канонической(нормальной), если:
    \begin{enumerate}
        \item (над $\R$) в диагональном виде $\forall \alpha_i$ принимает только значения: -1, 0, 1
        \item (над $\CC$) в диагональном виде $\forall \alpha_i$ принимает только значения: 0, 1
    \end{enumerate}
\end{definition}
\begin{example}\tab
    \begin{enumerate}
        \item Пусть $\F=\R$:
        $$k(x)=b_{11}x_1^2+b_{22}x_2^2+\ldots+b_{nn}x_n^2 = \alpha_1x_1^2+\alpha_2x_2^2+\ldots+\alpha_nx_n^2$$
        Если $rkB=r \Longrightarrow  k(x)=\alpha_1x_1^2+\alpha_2x_2^2+\ldots+\alpha_rx_r^2(\alpha_{r+1}=\ldots=\alpha_n=0)$.\\
        Если $\alpha_i>0$, то введём обозначение: 
        $$\widehat{x}_i=\sqrt{\alpha_i}x_i \Longrightarrow k=\widehat{x}_1^2+\ldots+\widehat{x}_p^2-\widehat{x}_{p+1}^2-\ldots-\widehat{x}_r^2$$
        где $p$ - количество коэффициентов $\alpha_i>0$.\\
        Если $\alpha_i < 0 \Longrightarrow  \widehat{x_i} = -\sqrt{\alpha_i}x_i$.
        \item Пусть $\F=\CC$: 
        $$\forall i=\overline{1,r}: \ \widehat{x}_i=\sqrt{\alpha_i}x_i \Longrightarrow k=\widehat{x}_1^2+\ldots+{\widehat{x}_r}^2$$
    \end{enumerate}
\end{example}
Таким образом, в вещественном случае для любой квадратичной формы $k(x)$ существует замена координат $X = CY (|C| \neq 0)$ такая, что в новых координатах $k = \sum \limits_{i=1}^p x_i^2 - \sum \limits_{j=p+1}^{p+q} x_j^2$.
\begin{definition}
    $p$ в такой записи называется положительным индексом инерции, $q$ - отрицательным индексом инерции.
\end{definition} 
\begin{theorem} \textbf{ Единственности (закон инерции)} \\
    %Для данной квадратичной формы над $\R$ числа $p$ и $q$ (количество +1 и \\количество -1 в каноническом виде) определены единственным образом
    Если в некоторых базисах $e_1,...e_n$ и $f_1,...,f_n$ квадратичная форма $k$ имеет канонические виды:
    $$k = \sum \limits_{i=1}^p y_i^2 - \sum \limits_{j=p+1}^{p+q} y_j^2 = \sum \limits_{i=1}^{p'} z_i^2 - \sum \limits_{j=p'+1}^{p'+q'} z_j^2$$
    то $p = p', q = q'$.
\end{theorem} 
\begin{proof}
    Так как $p + q = rk B = p' + q'$, достаточно доказать, что $p = p'$.
    От противного: пусть $p' < p$. Рассмотрим подпространства: 
    $$U_1 = \langle e_1,..,e_p \rangle, \ U_2 = \langle f_{p'+1},..,f_n \rangle$$ 
    Очевидно, что: $\dim U_1 = p, \dim U_2 = n- p'$.
    $$\dim U_1 + \dim U_2 = p - p' + n > n; \ \ U_1 + U_2 \subset V \Rightarrow dim(U_1 + U_2) \leq n$$
    Из формулы Грассмана:
    $$\dim (U_1 + U_2) = \dim U_1 + \dim U_2 - \dim (U_1 \cap U_2) \Rightarrow dim(U_1 \cap U_2) > 0$$
    Рассмотрим вектор $0 \neq v \in U_1 \cap U_2$:
    $$v = \sum \limits_{i=1}^{p} \alpha_i e_i \Rightarrow k(v) = \sum \limits_{i=1}^{p} \alpha_i^2 \geq 0$$
    С другой стороны:
    $$v = \sum \limits_{k=p'+1}^{n} \beta_k f_k \Rightarrow k(v) = -\sum \limits_{k=p'+1}^{n} \beta_k^2 \leq 0$$
    Отсюда $k(v) = 0 \Longrightarrow \forall i = 1,...,p \ \ \alpha_i = 0 \Longrightarrow v = 0$ - противоречие.
\end{proof}
\subsection{Знакоопределённые квадратичные формы}
\begin{definition}
    Пусть $b(x, y)$ - симметрическая билинейная форма. Векторы $u, v$ называются \textit{ортогональными}, если $b(u, v) = 0$. Обозначается: $u \perp v$.
\end{definition}
\begin{definition} 
    Базис $e_1,...,e_n$ в $V$ - \textit{ортогональный}, если $b(e_i, e_j) = 0 \ (i \neq j)$.
\end{definition}
\begin{definition}
    Для квадратной матрицы $B$ главными минорами (угловыми минорами) называются миноры $\Delta_1,\Delta_2,...,\Delta_{n-1}$, где $\Delta_i = \begin{vmatrix} b_{11}&\dots&b_{1i}\\\vdots&\null&\vdots\\b_{i1}&\dots&b_{ii}\end{vmatrix}$. \\
    Определим $\Delta_n = |B|, \Delta_0 = 1$.
\end{definition}
\begin{theorem} \textbf{Якоби}
    Пусть $k(x) \ (k(x)=b(x, x), b - \text{симм. б. ф.})$ такова, что главные миноры её матрицы $B$ в нек. базисе $e: \Delta_1,\Delta_2,...,\Delta_{n-1} \neq 0$
    Тогда в $V$ существует базис (и замена координат $X = CY$), в котором:
    $$k = \sum \limits_{i=1}^{n} \frac{\Delta_i}{\Delta_{i-1}}y_i^2$$
\end{theorem}
\begin{proof}
    Будем строить базис $e'$ из базиса $e$, ортогональный относительно $b(x, y)$ (алгоритм ортогонализации Грама/Шмидта).
    $$e'_1 := e_1; \ \ \forall k \geq 1 \ \langle e'_1,...,e'_k\rangle = \langle e_1,...,e_k\rangle$$
    причём $b(e'_i, e'_j) = 0 \ (1 \leq i \neq j \leq k)$\\
    Шаг алгоритма: допустим, что $k > 1$ и векторы $e'_1,...,e'_{k-1}$ уже построены. Будем искать $e'_{k}$ в виде 
    $$e'_k = e_k + \sum \limits_{i=1}^{k-1} \lambda_i e'_i$$
    где $\lambda_i$  найдём из условия $b(e'_k, e'_j) = 0, \ j = 0,...,k-1$
    $$b(e'_k, e'_j) = b(e_k, e'_j) + \sum \limits_{i=1}^{k-1} \lambda_j b(e'_i, e'_j) = b(e_k, e'_j) + \lambda_j b(e'_j, e'_j) = 0 \Rightarrow \lambda_j = -\frac{b(e_k, e'_j)}{b(e'_j, e'_j)}$$
    Покажем по индукции, что $b(e'_j, e'_j) = \frac{\Delta_j}{\Delta_{j-1}} \neq 0$.\\
    Обратим внимание, что матрица перехода от $e_1,...,e_{k-1}$ к $e'_1,...,e'_{k-1}$ - верхняя треугольная с 1 по диагонали (предп. индукции).
    Запишем: 
    $$C_{(e_1,...,e_k)\rightarrow (e'_1,...,e'_k)} = \begin{pmatrix}C_{k-1}&*\\0&1\end{pmatrix}, \text{ где } C_{k-1} = \begin{pmatrix} 1&\null&* \\ \null&\ddots&\null \\ 0&\null&1 \end{pmatrix}$$
    $B$ - матрица билин. формы $b(x, y)$ в базисе $e$, $B'$ - в базисе $e'$, который мы строим.
    $$B'_{k | \langle e'_1,...,e'_k \rangle} = C_k^T B_{k | \langle e_1,...,e_k \rangle} C_k \Rightarrow \det(B'_{k | \langle e'_1,...,e'_k \rangle}) = (\det C_k)^2 \cdot \det B_{k | \langle e_1,...,e_k \rangle}$$
    $$\Delta'_k = \det(B'_{k | \langle e'_1,...,e'_k \rangle}) = b'_{11}...b'_{kk} = \Delta_k$$
    $$\frac{\Delta_1}{\Delta_0}\cdot\frac{\Delta_2}{\Delta_1}\cdot...\cdot\frac{\Delta_{k-1}}{\Delta_{k-2}}\cdot b'_{kk} = \Delta_k \Rightarrow b'_{kk} = \frac{\Delta_k}{\Delta_{k-1}}$$
\end{proof}
Далее рассматриваем $F = \R$.
\begin{definition}
    Квадратичная форма $k(x)$ на пр-ве $V$ над $\R$ называется 
    \begin{itemize}
        \item положительно определённой, если $\forall x \neq 0 \ k(x) > 0$ (обозн. $k > 0$);
        \item отрицательно определённой, если $\forall x \neq 0 \ k(x) < 0$ (обозн. $k < 0$);
        \item неотрицательно определённой, если $\forall x \ k(x) \geq 0$ (обозн. $k \geq 0$);
        \item неположительно определённой, если $\forall x \ k(x) \leq 0$ (обозн. $k \leq 0$).
    \end{itemize}
\end{definition}
\begin{subtheorem}
    Квадратичная форма $k(x)$ является
    \begin{enumerate}
        \item положительно определённой $\Longleftrightarrow  p = n, q = 0$;
        \item отрицательно определённой $\Longleftrightarrow  p = 0, q = n$;
        \item неотрицательно определённой $\Longleftrightarrow  q = 0$;
        \item неположительно определённой $\Longleftrightarrow  p = 0$;
        \item знаконеопределённой $\Longleftrightarrow  p,q > 0$.
    \end{enumerate}
\end{subtheorem}
\begin{proof}
    Очевидно.
\end{proof}
\begin{lemma}
    Если кв. форма $k > 0$, то $\det B = \Delta_n \neq 0$.
\end{lemma}
\begin{proof}
    Т.к. $k>0$, $p = n$, т.е. существует базис, в котором: $$k(x') = {x'_1}^2 + ... + {x'_n}^2 \Longrightarrow  \Delta'_n = 1 > 0$$
    А так как $B' = C^TBC$, $|B'| = |C|^2\cdot|B| \Longrightarrow  \det B > 0$.
\end{proof}
\begin{theorem} \textbf{Критерий Сильвестра}\\
    Квадратичная форма $k(x)$, имеющая в некотором базисе матрицу $B$, является
    \begin{enumerate}
        \item положительно определённой $\Leftrightarrow \Delta_1 > 0,...,\Delta_n > 0$.
        \item отрицательно определённой $\Leftrightarrow \forall k \ (-1)^k\Delta_k > 0$.
    \end{enumerate} 
\end{theorem}
\begin{proof} Для положительной определённости:
    \begin{itemize}
        \item[ $\underline{\Longleftarrow }$ ] По теореме Якоби $\exists$ базис, в котором $k = \sum \limits_{i=1}^{n} \frac{\Delta_i}{\Delta_{i-1}}y_i^2$. Т.к все $\Delta_i > 0$ (знакочередующиеся для отрицательного случая), все коэффициенты $> 0 (< 0)$, т.е. значение формы на любом ненулевом векторе имеет необходимый нам знак.
        \item[ $\underline{\Longrightarrow }$ ] $k>0 \Longrightarrow  \Delta_1\cdot\Delta_2\cdot...\cdot\Delta_n \neq 0$ (k-ый минор ненулевой по лемме для угловой подматрицы) \ $\Longrightarrow $ применима т. Якоби, из которой следуют необходимые нам знаки на всех $\Delta$
    \end{itemize}
    Для отрицательной определённости: $k < 0 \Leftrightarrow -k > 0$, причём при домножении матрицы на $-1$ знак меняют только миноры нечётного порядка. 
\end{proof}
\begin{remark}
    Т.к. $b_{ii} = k(e_i)$, у положительно определённой формы все $b_{ii} > 0$, у отрицательной все $b_{ii} < 0$.
\end{remark}
\begin{remark}
    Пусть $k(x)$ такая, что $\Delta_1,...,\Delta_r \neq 0, \Delta_{r+1} = ... = \Delta_n = 0$. Тогда $p$ - число сохранений знака в последовательности $\Delta_0,\Delta_1,...,\Delta_r$, а $q$ - число перемен знака в этой последовательности.
\end{remark}
\begin{proof}
    Из теоремы Якоби в подходящем базисе
    $$k(x) = \frac{\Delta_1}{\Delta_0}y_1^2 + ... + \frac{\Delta_k}{\Delta_{k-1}}y_k^2 + ... + \frac{\Delta_r}{\Delta_{r-1}}y_r^2$$
    Тогда каждое сохранение знака соответствует положительному коэффициенту, а каждая перемена знака - отрицательному коэффициенту, откуда и следует необходимое равенство. 
\end{proof}
\subsection{Кососимметрические билинейные формы}
$$\forall\ x,y \in V: \ \ b(y,x) = -b(x,y) \ \ \ (\text{char } \F \neq 2)$$
Заметим, что $\forall x\in V: \ b(x,x) = 0$. Если же $b(x, x) \equiv 0$:
$$0 = b(x+y,x+y) = b(x,x) + b(y,y) + b(x,y) + b(y,x) \Longrightarrow  b(x,y) = -b(y,x)$$
Поэтому условие $b(x,x) \equiv 0$ не только эквивалентно кососимметричности формы, но и применимо в случае $\text{char } F = 2$.
\begin{lemma}
    Пусть $b(x,y)$ - симметрическая или кососимметрическая билинейная форма на $V \ (\dim V = n < \infty), \ U\subset V$. Тогда если $e_1,...,e_m$ - базис $U$, то $y \in U^\perp \Longleftrightarrow b(e_i, y) = 0, \ i = 1,...,m$.
\end{lemma}
\begin{proof}
    $U^{\perp} = \{y\in V: \ b(x,y) = 0  \ \forall x\in U\}$.\\
    $\Rightarrow: \ \ y \in U^\perp \Longrightarrow b(x, y) = 0 \ \forall x \in U \Longrightarrow b(e_i, y) = 0, \ i = 1,...,m$;\\
    $\Leftarrow: \ b(e_i, y) = 0 \Longrightarrow \forall x \in U \ b(x, y) = b(\sum \limits_{i=1}^m x_ie_i, y) = \sum \limits_{i=1}^m x_i b(e_i, y) = \sum \limits_{i=1}^m 0 = 0 \Longrightarrow y \in U^\perp$.
\end{proof}
\label{oplus}
\begin{theorem}
    Пусть $b(x,y)$ - симметрическая или кососимметрическая билинейная форма на $V \ (\dim V = n < \infty), \ U\subset V$ такое, что $b|_U$ невырождена. Тогда $V = U \oplus U^{\perp}$.
\end{theorem}
\begin{proof}
    Из леммы  $y \in U^\perp \Longleftrightarrow b(e_i, y) = 0, \ i = 1,...,m$.\\
    Запишем систему уравнений для нахождения $y$, выбрав базис $e_1,...,e_m \in U$ и дополнив его до базиса $e_1,...,e_n \in V$. В этом базисе $e_i^{\uparrow} = \begin{pmatrix} 0 \\ \vdots \\ 1 \\ \vdots \\ 0 \end{pmatrix}, \ b(e_i, y) = (0,...,1,...,0)B \begin{pmatrix} y_1 \\ \vdots \\ y_n \end{pmatrix} = (b_{i1},...,b_{in})Y^{\uparrow}$\\
    Система имеет вид 
    $$\begin{pmatrix}
        b_{11}&\dots&b_{1n}\\
        \vdots&\null&\vdots\\
        b_{m1}&\dots&b_{mn}
    \end{pmatrix} Y = 0$$
    Т.к. матрица $B|_U$ невырождена, $\text{rk} B = m$, т.е. система имеет $n-m$ ЛНЗ решений, а значит, $\dim U^\perp = \dim V - \dim U$.\\
    Если же $v \in U\cap U^\perp$, то $\forall \ x\in U \ b(x,v) = 0$, а из невырожденности формы $b|_U$ тогда следует, что $v = 0$. 
\end{proof}
\begin{theorem}
    Для любой кососимметрической билинейной формы $b(x,y) \not \equiv 0 \\ \exists$ такой базис $f_1,...,f_n \in V$, в котором матрица этой формы имеет вид
    $$\begin{pmatrix}
    \text{\fbox{$I_1$}}\\
    \null & \ddots\\
    \null & \null & \text{\fbox{$I_s$}}\\
    \null & \null & \null & 0\\
    \null & \null & \null & \null & \ddots\\
    \null & \null & \null & \null & \null & 0
    \end{pmatrix}$$ 
    где $I_j = \begin{pmatrix} 0&1 \\ -1&0 \end{pmatrix}$, $\text{rk} B = 2s$
\end{theorem}
\begin{proof}
    Т.к. $b(x,y) \not \equiv 0$, $\exists$ векторы $e_1, e_2 \in V$: \ $b(e_1, e_2) = \beta_{12} \neq 0$. \\
    Рассмотрим:
    $$f_1 = \frac{e_1}{\beta_{12}}, f_2 = e_2 \Longrightarrow  b(f_1,f_2) = 1, b(f_2,f_1) = -b(f_1,f_2) = -1$$
    Пусть $U = \langle e_1, e_2\rangle$. Возьмём $W = U^{\perp}$ в пр-ве $V$. Тогда $V = U \oplus U^{\perp}$ и $\tilde{b} = b|_{U^{\perp}}$ также кососимметрическая форма, т.е. можем провести индукцию по $n = \dim V$ (базу $n = 2$ доказали). $\dim W = n-2$, т.е. $\exists$ базис $f_3,...,f_n$, в котором матрица $b|_{U^{\perp}}$ имеет вид $\begin{pmatrix}
    \text{\fbox{$I$}}\\
    \null & \ddots\\
    \null & \null & \text{\fbox{$I$}}\\
    \null & \null & \null & 0\\
    \null & \null & \null & \null & \ddots\\
    \null & \null & \null & \null & \null & 0
    \end{pmatrix}$, т.е в базисе $f_1,...,f_n$ матрица $b$ имеет нужный вид.
\end{proof}