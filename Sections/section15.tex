\section{Теорема Жордана}
    Основное условие: \ \  $\phi: \ V \to V$ - линейный оператор, все его корни $\in \F$
    $$\chi_{\phi}(\lambda) = (-1)^n(\lambda-\lambda_1)^{k_1}\cdot\ldots\cdot(\lambda-\lambda_s)^{k_s} \ (\forall i\neq j: \ \lambda_i\neq\lambda_j \text{ и } \sum \limits_{i=1}^s k_i = \dim V)$$
    $$V = K_1\oplus\ldots\oplus K_s, \ \text{где } K_i = \text{Ker}(\phi-\lambda_\text{id}d)^{k_i} - \text{корневое подпространство}$$
    $$V_{\lambda_i} = \{x\in V|\phi(x) = \lambda_ix\}, \  \dim V_{\lambda_i}\leqslant k_i = \dim K_i$$
    Так как $K_i$ - инвариантное подпространство относительно оператора $\phi$, можно рассмотреть ограничение: 
    $$(\phi-\lambda_i \text{ id})|_{K_i} := B_i$$
    Из определения $k_i$ следует, что $B_i^{k_i}=0$, то есть $B_i$ - нильпотентный оператор.\\
    Обозначим $h_i$ - показатель нильпотентности оператора, т.е. $B_i^{h_i}=0$,\\ 
    но $B_i^k\neq0$ $\forall k < h_i$\\
    В базисе, согласованном с этим разложением: 
    $$A_{\phi} = \begin{pmatrix}
    \text{\fbox{$A_1$}}\\
    \null & \text{\fbox{$A_2$}}\\
    \null & \null & \ddots\\
    \null & \null & \null & \text{\fbox{$A_s$}}
    \end{pmatrix}$$
    где $A_i = A_{\phi_{k_i}}$ - марица порядка $k_i, \  A_i-\lambda_iE_{k_i} = B_i, \ B_i^{k_i}=0$\\
    Обозначим $K_i :=K$, $B_i :=B$, $k_i :=k$, тогда: 
    $$\forall \bar{x}\in K: \  B^k(x) = 0$$
    если $x\neq0$, то $\exists$ наименьшее значение $m$: 
    $$B^m(x) = 0, \ B^{m-1}(x)\neq_0 \ (m\leqslant h)$$ 
    Назовём это высотой вектора $x$.\\
    Для фиксированного вектора $x\neq0$ (высоты $m$) рассмотрим векторы: 
    $$x, \ B^0x, \ Bx, \ldots,B^{m-1}x, \ B^mx = 0$$
    \begin{definition}
        Векторы \{$x,\ B^0x,\ Bx,\ \ldots,\  B^{m-1}x = 0$\} называются жордановой цепочкой.
    \end{definition}  
    \begin{lemma}
        Вышеуказанные векторы являются линейно независимыми.
    \end{lemma}
    \begin{proof}
        Предположим, что: 
        $$\alpha_0x+\alpha_1B^0x+\ldots+\alpha_{m-1}B^{m-1}x=0$$
        Подействуем на это равенство оператором $B^{m-1}$: $$\alpha_0B^{m-1}x = 0 \ \Longrightarrow \ \alpha_0 = 0$$ 
        На оставшиеся векторы подействуем оператором $B^{m-2}$:
        $$\alpha_1B^{m-1}x = 0 \ \Longrightarrow \ \alpha_1 = 0$$
        и т.д. Получим, что $\forall i = \overline{0,m-1}: \ \alpha_i = 0 \ \Longrightarrow$ векторы являются линейно независимыми.
    \end{proof}
    \begin{definition}
        Подпространство, натянутое на эти векторы: $$\langle x,\ B^0x,\ Bx,\ \ldots,\  B^{m-1}x \rangle$$
        называется циклическим подпространством, порождённым жордановой цепочкой. Данное подпространство обозначим $U$, $\dim U_x = m$.\\
        Обычно векторы жордановой цепочки нумеруют с конца, то есть:
        $$a_1 = B^{m-1}x, \ a_2 = B^{m-2}x, \ldots, a_m = x$$ 
        Тогда $a_1$ - собственный вектор для $B$, и для $\forall j = \overline{2,m}: \ a_{j-1} = Ba_j$\\
        Векторы $a_j$ называются присоединёнными к вектору $a_{j-1}$\\
        К вектору $a_1$: \ $a_2$ - присоединённый, $a_3$ - второй присоединённый и т.д. 
    \end{definition}
    \begin{definition} $\\$ 
        Матрица ограничения оператора $B$ на подпространство $U_x = \langle a_1\ldots a_m\rangle$ : 
        $$B|_{U_x} = \begin{pmatrix}
        0 & 1\\
        \null & 0 & 1\\
        \null & \null & \ddots & \ddots\\
        \null & \null & \null & 0 & 1\\
        \null & \null & \null & \null & 0
        \end{pmatrix} = J_k(0)$$ 
        называется жордановой клекткой с собственным значением $\lambda = 0$
        $$\lambda=\lambda_i: \ A_{\phi|_{U_x}} = \begin{pmatrix}
        \lambda_i & 1\\
        \null & \lambda_i & 1\\
        \null & \null & \ddots & \ddots\\
        \null & \null & \null & \lambda_i & 1\\
        \null & \null & \null & \null & \lambda_i
        \end{pmatrix}=J_k(\lambda_i)$$ 
        - жорданова клектка с собственным значением $\lambda = \lambda_i$, где: 
        $$\phi(a_2) = a_1+\lambda_ia_2, \ \phi(a_{j+1}) = a_j+\lambda_ia_{j+1}$$
    \end{definition} 
    \begin{theorem} \textbf{Жордана} \\
        \tab[0.5cm]Если все характеристические корни опертора $\phi: \ V \to V$ принадлежат полю $\F$, то $V$ является прямой суммой циклических подпространств для оператора $\phi$. Это равносильно тому, что в $V$ существует базис, составленный из жордановых цепочек. Такой базис называется жордановым базисом.\\
        \tab[0.5cm]Если жордановыйй бадис уже построен: Пусть имеются $r$ жордановых цепочек, отвечающих собственным значениям $\lambda_1, \ldots, \lambda_r$, необязательно различным, длины которых $m_1,\ldots,m_r$ соответственно, тогда в этом базисе:
        $$A_{\phi} = \begin{pmatrix}
        \text{\fbox{$J_{m_1}(\lambda_1)$}} & \null & \null & 0 \\
        \null & \text{\fbox{$J_{m_2}(\lambda_2)$}}\\
        \null & \null & \ddots\\
        0 & \null & \null & \text{\fbox{$J_{m_r}(\lambda_r)$}}\\
        \end{pmatrix} \ \ \ \sum \limits_{i=1}^rm_i = n = \dim V$$ 
        - жорданова матрица - жорданова нормальная форма (ЖНФ) матрицы $A_{\phi}$.
    \end{theorem}
    \begin{theorem}\textbf{Жордана (матричная формулировка)} \\
        Для любой матрицы $A \in M_n(\F)$, все характеристические корни которой $\in \F$, $\exists$ матрица $C \ (\det C \neq 0)$ такая, что: 
        $$C^{-1}AC = J$$
        - жорданова матрица. При этом жордановы клетки определены для матрицы $A$ единственным образом с точностью до расположения клеток на диагонали жордановой матрицы.  
    \end{theorem}
    \begin{remark}
        Матрицу $A$ можно интерпретировать как матрицу линейного оператора $\phi$, для него верна теорема Жордана.  
    \end{remark}
    \begin{proof} (См. Шафаревич И.Р. "Линейная алгебра")\\
        Доказательство достаточно провести для ограничения оператора на каждое корневое подпространство $K_i$, в этом случае обозначаем $B = (\phi - \lambda_i \text{id})$,\\ 
        где $B$ - нильпотентный оператор
        % P.S. - Доказательство после колока по Матану разберу и напишу :)
        \begin{lemma}
            Если $B$ - такой оператор в пространстве $V$, что: 
            $$\text{Im}B = B(V) \subset V$$
            то $V$ обладает $(n-1)$-мерным инвариантным подпространством.     
        \end{lemma} 
        Пусть $e_1,...,e_m$ - базис в $\text{Im}B, \ m<n = \dim V$\\
        Дополним его до базиса в $V$ векторами $e_{m+1},...,e_n$, тогда:
        $$\langle e_1,...,e_{n-1} \rangle - \text{инвариантное подпространство:}$$
        $$\forall w = \sum \limits_{i=1}^{n-1}\beta_ie_i \Longrightarrow B_w = \sum \limits_{i=1}^{n-1}\beta_iBe_i \in \text{Im}B \subseteq W$$ 
        $\Longrightarrow $ W - инвариантное подпространство\\
        Ниже будем считать, что $B: \ V \to V$ - нильпотентный оператор, \\
        $\dim V = n, \ W - (n-1)$-мерное инвариантное подпространство в $V$\\
        Будем проводить индукцию по $n$:\\
        Если $n=1$, то $B = 0$ и любой базис - жорданов\\
        Пусть $n>1$, предположение индукции: в $W \ \exists$ базис для $B|_w$\\
        Выберем вектора $a \in V\setminus W$. По предположению индукции:
        $$W = U_1 \oplus ... \oplus U_r$$
        Если вектор $a$ - собственные для $B$, а т.к. он ЛНЗ с векторами из $W$, то: 
        $$V = U_1 \oplus ... \oplus U_r \oplus \langle a \rangle$$ 
        - нужное разложение пространства $V$\\
        Если $a$ - не собственный, то он порождает жорданову цепочку некоторой длины $m$, которая не содержится в $W$.          
    \end{proof}
    \begin{lemma}
        Путсь $U$ - циклическое подпространство, порожденное корневым вектором $e$ высоты $m$. Тогда $\forall y \in U$ представляется в виде:
        $$y = f(B)e, \ \text{где } f(t) - \text{минимальной степени } \leq n-1$$     
    \end{lemma} 
   

