\section{Евклидовы аффинные пространства}
\begin{definition}
    Аффинное простарнство $(\mathbb{A},\mathcal{E})$ - евклидово, если $\mathcal{E}$ - евклидово пространство (над $\R$), $\mathcal{E}$ ассоциированно с пространством точек $\mathbb{A}$.\\
    Расстояние определяется как
    \[\rho(p,q)=|\overline{pq}|\]
    Для трех точек $a,b,c$ угол между лучами $(ab)$ и $(ac)$ - это угол между векторами $\overline{ab}$ и $\overline{ac}$ (если они ненулевые).
\end{definition} 

\begin{definition} $\\$
    Расстояние от точки $p_1\in \mathbb{A}$ до плоскости $P=p_0+U,\ V\supset U\ne \{0\}$.\\
    Либо $p_1\in P$, либо $\overline{p_0p_1}\not\in U$.\\
    Можно рассматривать подпространтсво:
    \[ \widetilde{U}=\langle \overline{p_0p_1}+U \rangle\supset V,\ \overline{p_0p_1}=y+z,\ y\in U,\ z\in U^{\perp}\Longrightarrow \min|\overline{p_1q}|=|z|\]
\end{definition} 

\begin{definition}
    Параллелепипед с одной вершиной $p_0$ и ребрами $a_1,\dots,a_m$, где $m\leq n,\ a_i\in \mathcal{E}$:
    \[\Pi_{\langle p_0,a_1,\dots,a_m \rangle}=\{p_0+\sum\limits_{i=1}^{m}\lambda_i a_i: 0\leq \lambda_i\leq 1\}\]
\end{definition} 
Определим $m$-мерный объем рекурсивно:
для $m=1$:
\[V(\Pi_{1})=|a_1|\]
\[V(\Pi_m)=(a_m)_{\perp}\cdot V_{\{p_0,a_1,\dots,a_{m-1}\}}\]
где $(a_m)_{\perp}$ - ортогональная сотавляющая ребра $a_m$ отностительно подпространства $\langle a_1,\dots, a_{m-1} \rangle$.\\
Пусть $a_1,\dots,a_m$ линейно независимы. Тогда:
\[V_{{p_0,a_1,...,a_m}}=|G_{\{a_1,...,a_m\}}|\]

Можно ортогонализовать векторы $a_1,\dots,a_m$, причем матрица перехода от $a_1,\dots,a_m$ к $b_1,\dots,b_m$, где $b_1,\dots,b_m$ получены из алгоритма ортогонализации, выглядит так:
\[C=\begin{pmatrix}
    1 & \null & *\\
    \null & \ddots & \null\\
    0 & \null & 1
\end{pmatrix}\]
\[|G_{\{a_1,\dots,a_m\}}| = |G_{\{b_1,\dots,b_m\}}| = \begin{vmatrix}
    |b_1^2| & \null & 0\\
    \null & \ddots & \null\\
    0 & \null & |b^2_m|
\end{vmatrix} = |b_1|^2 \cdot ... \cdot |b_m|^2\]
Значит:
\[\rho(p_1,P)=\frac{\sqrt{|G_{\{a_1,\dots,a_m,\overline{p_0p_1}\}}|}}{\sqrt{|G_{\{a_1,\dots,a_m\}}|}}\]
Если $P_1=p_1+U_1, P_2=p_2+U_2$ - две афиинные плоскости в аффинном пространстве, то назовем:
\[\rho(P_1,P_2)=\inf\{|\overline{pq}|: p\in P_1, q\in P_2\}\]
\begin{theorem}
    $\rho(P_1,P_2)$ равно длине ортогональной сотавляющей вектора $\overline{p_1p_2}$ относительно $U_1+U_2$
\end{theorem} 
\begin{remark}
    Если $P_1\cap P_2\ne \emptyset$, то $\rho(P_1,P_2)=0,\ \overline{p_1p_2} \in U_1+U_2$, так что $(p_1,p_2)_{\perp}=0$, что не противоречит утверждению теоремы.
\end{remark} 
\begin{proof}
    Обозначим $W=U_1+U_2$, тогда $\mathcal{E}=W\oplus W^{\perp}$. Обозначим 
    \[\overline{p_1p_2}=v=v_{\parallel}+v_{\perp},\ v_{\parallel}\in W,\ v_{\perp}\in W^{\perp}\] 
    Попробуем доказать, что существуют 
    \[a=p_1+u^0_1\in P_1,\ b=p_2+u^0_2\in P_2\] 
    такие, что $\overline{ab}=v_{\perp}$. \\
    Выберем произвольные точки $x=p_1+u_1\in P_1,\ y=p_2+u_2\in P_2$. Тогда:
    \begin{multline*}
        \rho^2(x,y)=|\overline{yx}|^2=|\overline{p_2p_1}+u_1-u_2|^2=|v+u_2-u_1|^2=\\
        =|(v_{\parallel}+u_2-u_1)+v_{\perp}|^2=|v_{\parallel}+u_2-u_1|^2+|v_{\perp}|^2\geq |v_{\perp}|^2
    \end{multline*}
    где $v_{\perp}\in (U_1+U_2)^{\perp}$. Равенство достигается, если $v_{\parallel}=u_2-u_1 \Rightarrow \exists\ u_1, u_2$ такие, что $a=p_1+u_1,\ b=p_2+u_2: |\overline{ab}|=v_{\perp}$.
\end{proof} 
\begin{consequense}
    Прямая $l=a+\langle \overline{ab} \rangle=(p_1+u_1)+\langle (\overline{p_1p_2})_{\perp} \rangle$ является общим перпендикуляром этих двух плоскостей.
\end{consequense} 
%ПАРАГРАФ
\subsection{Афиинные отображения и преобразования}
Пусть $(\mathbb{A}_1,V_1)$ и $(\mathbb{A}_2, V_2)$ - аффинные пространства над одним и тем же полем.
\begin{definition}
    Отображение $\Phi: \mathbb{A}_1 \to \mathbb{A}_2$ называется аффинно-линейным отображением, если существует линейное отображение $\phi: V_1\to V_2$ такое, что
    \[\forall a,b\in \mathbb{A}_1: \overline{\Phi(a)\Phi(b)}=\phi(\overline{ab}) \eqno(1)\]
    Такое определение равносильно следующему:
    \[\forall a,b\in \mathbb{A}_1 : \ \Phi(b)=\Phi(a)+\phi(\overline{ab}) \eqno(2)\]
\end{definition} 
\begin{subtheorem} \tab
    \begin{enumerate}
        \item Пусть 
        \[\mathbb{A}_1 \xrightarrow{\Phi_1} \mathbb{A}_2 \xrightarrow{\Phi_2} \mathbb{A}_3\]
        где $\Phi_1,\Phi_2$ - аффинно-линейны, то
        \[\Phi=\Phi_2\cdot \Phi_1: \ \mathbb{A}_1\to \mathbb{A}_3\]
        тоже аффинно-линейно с линейной частью $\phi=\phi_2\cdot \phi_1$
        \item $\mathbb{A}_1 \xrightarrow{\Phi} \mathbb{A}_2$ биективно $\Longleftrightarrow \phi$ - биективно, при этом $\Phi^{-1}$ является \\аффинно-линейным с линейной частью $\phi^{-1}$.
    \end{enumerate}
    
\end{subtheorem}