%chktex-file 1 %chktex-file 3 %chktex-file 8 %chktex-file 9 %chktex-file 10 %chktex-file 11 %chktex-file 12 %chktex-file 13 %chktex-file 16 %chktex-file 17 %chktex-file 18 %chktex-file 25 %chktex-file 26 %chktex-file 35 %chktex-file 36 %chktex-file 37 %chktex-file 40 %chktex-file 44 %chktex-file 45
\section{Пересечение и сумма подпространств}
  \begin{subtheorem}\tab
    \begin{enumerate}
      \item Если $U_i \ (i\in I)$ - подпространство $V$, то $W = \underset{i\in I}{\cap}U_i$ тоже подпространство в $V$;
      \item Объединение подпространств может НЕ быть подпространством даже для двух подпространств.
      
      \[
        \begin{tikzpicture}
          \draw[->] (-1, 0) -- (6, 0) node[right] {x};
          \draw[->] (0, -1) -- (0, 6) node[above] {y};
          \draw[fill] (0, 0) circle (2pt);        
          \draw[thick] (-0.4, -1) -- (2, 5) node[above] {$U$};
          \draw[thick] (-1, -0.4) -- (5, 2) node[right] {$W$};
          \draw[->] (0, 0) -- (1, 2.5) node[left] {$u$};
          \draw[->] (0, 0) -- (2.5, 1) node[above] {$w$};
          \draw[->] (0, 0) -- (3.5, 3.5) node[right] {$u+w\not\in U\cup W$};
          % \draw[thick] (-0.7, -0.7) -- (4.5, 4.5);
      \end{tikzpicture}
      \]

    \end{enumerate}
  \end{subtheorem} 
  \begin{proof}
    1. $\overline{0} \in W$, т.к. $\overline{0} \in U_i, \ \forall i\in I$. \vspace{0.2cm}\\
    Если $x,y \in U_i, \ \forall i\in I \Longrightarrow x+y \in U_i, \ \forall i\in I \Longrightarrow x+y \in \underset{i\in I}{\cap}U_i$ \vspace{0.15cm}\\
    Если $x \in U_i, \ \forall i\in I, \ \forall \lambda \in F \Longrightarrow \lambda x \in U_i, \ \forall i\in I \Longrightarrow \lambda x \in \underset{i\in I}{\cap}U_i$  
  \end{proof}
  \begin{remark}
    Если $U_1, U_2$ - подпространства в $V$ и $Q$ - любое подпространство, которое содержит $U_1$ и $U_2$, то оно содержит и сумму $u_1+u_2$, если $u_i \in U_i, \ i =1,2$       
  \end{remark}
  \begin{definition}
    Суммой подпространств $U_1,...,U_m \subseteq V$ назовем: $$U_1 + ... + U_m = \{x_1+...+x_m \ | \ x_i \in U_i\}$$   
  \end{definition}
  \begin{subtheorem}
    $U_1 + ... + U_m$ - подпространство в $V$  
  \end{subtheorem}
  \begin{theorem} (Формула Грассмана)\\
    Если $U_1,U_2$ - подпространства в $V, \ \dim U_1 < \infty, \ \dim U_2 < \infty$, то 
    $$\dim (U_1+U_2) = \dim U_1 + \dim U_2 - \dim (U_1 \cap U_2)$$   
  \end{theorem}
  \begin{proof}
    Пусть $\dim U_i = n_i, \ \dim (U_1 \cap U_2) = s$
    Выберем $c_1,...,c_s$ - базис $U_1 \cap U_2$, дополним до базиса в $U_1$ векторами $a_1,...,a_{n_1-s}$ и до базиса в $U_2$ векторами $b_1,...,b_{n_2-s}$.\\ Тогда векторы $c_1,...,c_s,a_1,...,a_{n_1-s},b_1,...,b_{n_2-s}$ - образуют базис в $U_1 + U_2$
    \begin{enumerate}
      \item Они порождают $U_1 + U_2:$
      $$\forall u = u_1+u_2 = (\sum \alpha_i a_i + \sum x_i c_i) + (\sum \beta_i b_i + \sum \delta_i c_i)$$
      \item Они ЛНЗ. Рассмотрим линейную комбинацию: 
      $$\sum \limits_{i=1}^{n_1-s} \alpha_i a_i + \sum \limits_{k=1}^{n_2-s} \beta_k b_k + \sum \limits_{j=1}^{s} \gamma_j c_j = 0$$
      $$\sum \limits_{i=1}^{n_1-s} \alpha_i a_i = -\sum \limits_{k=1}^{n_2-s} \beta_k b_k - \sum \limits_{j=1}^{s} \gamma_j c_j \in U_1 \cap U_2$$
      Левая часть должна раскладываться по $\{c_j\} \Longrightarrow $ $\sum \limits_{i=1}^{n_1-s} \alpha_i a_i = 0 \Longrightarrow a_i$ - ЛНЗ $\Longrightarrow \forall i: \ \alpha_i = 0$ \\
      Тогда $\sum \limits_{k=1}^{n_2-s} \beta_k b_k + \sum \limits_{j=1}^{s} \gamma_j c_j = 0 \Longrightarrow \{b_k,\gamma_j\}$ - ЛНЗ $\Longrightarrow \forall k,j: \ \beta_k = \gamma_j = 0$  
    \end{enumerate}
    \begin{algorithm}
      Пусть $U_1 = \langle a_1,...,a_{n_1} \rangle, \ U_2 = \langle b_1,...,b_{n_2} \rangle$, известны координаты всех этих векторов. Составим матрицу:
      $$\begin{pmatrix}
        A & \vline & B
      \end{pmatrix} = \begin{pmatrix}
        a_1^{\uparrow},...,a_{n_1}^{\uparrow} & \vline & b_1^{\uparrow},...,b_{n_2}^{\uparrow}
      \end{pmatrix}$$
      $\dim (U_1 + U_2) = rk (A | B)$
      $$\begin{pmatrix}
        A & \vline & B
      \end{pmatrix} \xrightarrow[\text{строк}]{\text{ЭП}} \begin{pmatrix}
        a_1^{\uparrow},...,a_{n_1}^{\uparrow} & \vline & \underbrace{b_1^{\uparrow},...,b_m^{\uparrow}}_{\text{попало в базис}} ,b_{m+1}^{\uparrow},...,b_{n_2-m}^{\uparrow}
      \end{pmatrix}$$
      Можно записать: $$b_j = \sum \limits_{i=1}^{n_1} \alpha_i a_i + \sum \limits_{k=1}^{m} \beta_{k_j} b_k \Longrightarrow b_j - \sum \limits_{k=1}^{m} \beta_{k_j} b_k = \sum \limits_{i=1}^{n_1} \alpha_i a_i \in U_1 \cap U_2$$   
    \end{algorithm}
  \end{proof} 
  \begin{exercise}
    Верна ли аналогичная формула для трех подпространств? 
  \end{exercise}