\section{Прямая сумма подпространств и пространств}
    \begin{definition}
        Сумма $U_1+...+U_m$ подпространств $U_i \subset V, \ 1\leq i \leq m$ называется прямой суммой, если 
        $\forall u \in U_1+...+U_m$  представим в виде: \\$u = u_1+...+u_m \ (u_i \in U_i)$  единственным образом    
    \end{definition} 
    Пусть $m=2, V$ - конечномерное пространство, $U_{1,2}$ - подпространства $V$
    \begin{theorem}
        Следующие условия равносильны: 
        \begin{enumerate}
            \item $U = U_1 + U_2$ - прямая сумма
            \item $U_1 \cap U_2 = \{0\}$
            \item $\dim (U_1 + U_2) = \dim U_1 + \dim U_2$
            \item Базис $U_1 + U_2$ - объединение базисов слогаемых    
        \end{enumerate} 
    \end{theorem} 
    \begin{proof}\tab
        \begin{itemize}
            \item[$1. \to 2.$] Допустим $u \in U_1 \cap U_2 \Longrightarrow v = v + 0 = 0 + v \Longrightarrow v = 0$
            \item[$2. \to 3.$] По формуле Грассмана: 
            $$\dim (U_1 + U_2) = \dim U_1 + \dim U_2 - \underbrace{\dim (U_1 \cap U_2)}_{0}$$
            \item[$3. \to 4.$] Ввиду доказательства формулы Грассмана. Если $$\sum \limits_{i} \alpha_i a_i + \sum \limits_{j} \beta_j b_j = 0 \Longrightarrow \sum \limits_{i} \alpha_i a_i = \sum \limits_{j} (-\beta_j) b_j \in U_1 \cap U_2 = \{0\}$$
            $$\Longrightarrow  \text{все } a_i \text{ и } b_i \text{ равны нулю}$$
            \item[$4. \to 1.$] $\forall u \in U_1 + U_2: $ $$
            u = (\sum \limits_{i} \alpha_i a_i) + (\sum \limits_{j} \beta_j b_j)$$ 
            - разложение по базису единственно  
        \end{itemize}
    \end{proof}
    \begin{theorem}
        Следующие условия равносильны: 
        \begin{enumerate}
            \item $U = U_1 + U_2$ - прямая сумма
            \item $\forall i, \ 1 \leq i \leq m, \ U_i \cap (\sum \limits_{j \neq i}U_j) = \{0\}$
            \item $\dim (U_1 + U_2) = \dim U_1 + \dim U_2$
            \item Базис $U_1 + U_2$ - объединение базисов слогаемых    
        \end{enumerate} 
    \end{theorem}
    \begin{exercise}
        Доказать
    \end{exercise} 
    \begin{example1} того, что условия $U_i \cap U_j = \{0\}, \ i \neq j$ недостаточно для прямой суммы: (РИСУНОК)\\
    $v_1,v_2,v_3$ - ЛЗ $\Longrightarrow $ представление не единственным образом
    \end{example1}
    \begin{lemma}
        Любой ЛНЗ набор векторов $a_1,...,a_m$ в $n$-мерном векторном пространстве $V \ (m < n)$ можно дополнить до базиса в $V$.  
    \end{lemma}
    \begin{proof}
        $\underset{\bullet }{\oplus}$ 
    \end{proof} 