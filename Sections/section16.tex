%chktex-file 1 %chktex-file 3 %chktex-file 8 %chktex-file 9 %chktex-file 10 %chktex-file 11 %chktex-file 12 %chktex-file 13 %chktex-file 16 %chktex-file 17 %chktex-file 18 %chktex-file 25 %chktex-file 26 %chktex-file 35 %chktex-file 36 %chktex-file 37 %chktex-file 40 %chktex-file 44 %chktex-file 45 %chktex-file 49
\section{Биленейные и квадратичные формы}
\begin{definition}
    Функция $b: \ V\times V \to \F$ называется билинейной функцией, если:
    \begin{enumerate}
        \item аддитивность: $$\forall x_1, x_2, y: \ b(x_1+x_2, y) = b(x_1, y)+ b(x_2, y)$$
        $$\forall x, y_1, y_2: \ b(x, y_1 + y_2) = b(x, y_1)+ b(x, y_2)$$
        \item однородность: 
        $$\forall x,y \in V , \  \forall \lambda \in \F: \ b(\lambda x, y) = \lambda b(x, y) = b(x, \lambda y)$$ 
    \end{enumerate}
\end{definition}
\begin{definition}
    $b(x, y)$ - называется симметрической, если: 
    $$\forall x, y \in V: \ b(y, x) = b(x, y)$$ 
\end{definition}
\begin{example}\tab
    \begin{enumerate}
        \item Симметрическая билинейная функция - скалярное произведение
        \item $V = Mn(\F): \ b(X,Y) = tr(XY)$
        \item $\beta(f, g) = \int \limits_a^b f(x)g(y)dx$  
    \end{enumerate}
\end{example}
\subsection{Запись билинейной функции в координатах}
Пусть в $V$ задан базис $e_1,...,e_n$, тогда:
$$b(\sum \limits_{i=1}^mx_ie_i, \sum \limits_{j=1}^my_je_j) = \sum \limits_{i,j=1}^mb(x_ie_i, y_je_j) = \sum \limits_{i,j=1}^m x_iy_j(e_i,e_j)$$
\begin{definition}
    Обозначим $b_{ij} = b(e_i, e_j)$, тогда $B_e=b_{ij}$ - матрица билинейной функции $b(x, y)$ в базисе $e$\\
    Тогда:
    $$b(x, y) = \sum \limits_{i,j=1}^n x_ib_{ij}y_j = \begin{pmatrix}
        x_1 & \cdot & x_n
    \end{pmatrix}\begin{pmatrix}
        y_1\\
        \vdots \\
        y_n
    \end{pmatrix} = X^TBY \eqno (1)$$ 
    
\end{definition} 
\subsection{Изменение матрицы билинейной формы при замене базиса}
Пусть $e' = Ce$, \ $C$ - матрица перехода от $e$ к $e'$\\
Тогда: 
$$X = CX', \ Y = CY' \eqno (2)$$
По определению матрицы билинейной функции, в новом базисе: 
$$b(x, y) = X'^TB'Y' \ \ (B' = Be')$$ 
Подставим в формулу $(1)$ выраженеие $(2)$:
$$b(x, y) = X'^TC^TBCY' = X'^T(C^TBC)Y' = X'^TB'Y' \ \ (\forall X',Y' \in \F^n)$$
$$\Longrightarrow B' = C^TBC \ \ (\forall i,j: \ X':= E_i, \ Y':=E_j)$$    
\begin{consequense}\tab
    \begin{enumerate}
        \item $\text{rk}B' = \text{rk}B$
        \item $\F = \R \Longrightarrow \sgn (\det B') = \sgn (\det B)$  
    \end{enumerate}
\end{consequense} 
\begin{definition}
    Билинейная функция называетя кососиметрической (при char $\F \neq 2$), если: 
    $$\forall x, y \in V: \ b(x, y) = -b(y, x)$$  
\end{definition} 
\begin{subtheorem} $(*)$ 
    Любая билинейная функция над $\F: \ \text{char}\F \neq 2$ единственным образом представляется в виде:
    $$b(x, y) = b_+(x, y) + b_-(x, y), \ \text{ где } b_+(x, y) \equiv b_+(y, x), \ b_-(x, y) \equiv -b(y,x)$$  
\end{subtheorem}
\begin{proof}
    Если же есть равенство:
    $$\begin{cases}
        b(x, y) = b_+(x, y) + b_-(x, y)\\
        b(y, x) = b_+(x, y) - b_-(x, y)
    \end{cases} \Longrightarrow $$
    $$b_+(x,y) = \frac{b(x,y)+b(y,x)}{2}, \ b_-(x,y) = \frac{b(x,y)-b(y,x)}{2}
    $$ 
\end{proof} 

\begin{subtheorem}
    Билинейная функция $b(x,y)$ симметрична (кососимметрич-\\-на)$\Longleftrightarrow$ в любом базисе $e$: 
    $$B_e^T=B_e \ (B_e^T=-B_e)$$
\end{subtheorem}
\begin{proof} (Докажем для симметрической, для кососимметрической \\аналогично)
    \begin{itemize}
        \item[ $\underline{\Longrightarrow}$] Пусть $B = (b_{ij})$, тогда $b_{ij}=b(e_i, e_j)$.
        $$\forall x, y\in V, \ b(x,y)= b(y,x) \Longrightarrow  
        b(e_j, e_i) = b(e_i, e_j)$$
        \item[$\underline{\Longleftarrow}$]  
        $$b(x,y)= X^TBY, b(y,x) = Y^TBX = (X^TB^TY)^T = (X^TBY)^T = b(x,y)$$
    \end{itemize}
\end{proof}
  Утверждение (1) $\Longleftrightarrow$ $\forall$ матрицы $B$ некоторой билинейной функции верно, что $B = B_++B_-$, где $B_+$ - матрица симметрической билинейной функции, а $B_-$ - матрица кососимметрической билинейной функции.
\begin{definition}
    Квадратичная функция, порождённая билинейной функцией $b(x,y)$ - это функция на $V$. 
    Обозчаем: $k(x):=b(x,x)$, если $k(x)\not\equiv0$.
\end{definition}
    Если $b$ - кососимметрическая функция, то $b(x,x)=0$ $\Longrightarrow$ $k(x)\equiv0$. В общем случае существует бесконечно много билинейных функций, порождающих одну и ту же квадратичную, таких, что: 
    $$b(x,y)=b_+(x,y)+b_-(x,y) \Longrightarrow  b(x,x)=b_+(x,x)$$
\begin{theorem}
    $\forall$ квадратичной функции $\exists!$ симметрическая билинейная функция, которая её порождает.
\end{theorem}
\begin{proof}
    Допустим, что $b(x,y) = b(y,x)$ - симметрическая билинейная функция и $k(x) = b(x,x)$. Тогда $\forall x, y\in V$:
    \begin{multline*}
        k(x+y) = b(x+y, x+y) = b(x,x)+b(x,y)+b(y,x)+b(y,y)=\\ 
        = b(x,x)+2b(x,y)+b(y,y) = k(x)+2b(x,y)+k(y)
    \end{multline*}
    Так как $\text{char} \F \neq 2$, то: 
    $$b(x,y)=\frac{k(x+y)-k(x)-k(y)}{2}$$
\end{proof}
\begin{definition}
    Билинейная функция $b(x,y) = \frac{k(x+y)-k(x)-k(y)}{2}$ называется поляризацией квадратичной функции $k$.
\end{definition}
    Далее будем считать матрицу квадратичной формы матрицей её полярной симметрической билинейной функции $b(x,y)$
    $$b(x,y)=\sum\limits_{i=1}^nb_{ii}x_iy_i+\sum\limits_{i<j}b_{ij}x_iy_j+\sum\limits_{i>j}b_{ij}x_iy_j$$
    $$\forall i, j: \ b_{ij}=b_{ji} \Longrightarrow  b(x,x)=k(x)=\sum\limits_{i=1}^nb_{ii}x_i^2+\sum\limits_{1\leqslant i<j\leqslant n}b_{ij}x_ix_j\eqno{\textbf{(1)}}$$

\begin{example1}
    Пусть $k(x_1,x_2, x_3)=3x_1^2+2x_1x_2-x_1x_3+x_2^2+6x_2x_3-7x_3^2$, тогда:
        $$B=\begin{pmatrix}
        3 & 1 & -\frac{1}{2}\\
        1 & 1 &  3\\
        -\frac{1}{2} &  3 & -7
        \end{pmatrix}$$
\end{example1}
\begin{definition}
    Пусть $b(x,y)$ - симметрическая или кососимметрическая билинейная функция и $\varnothing \neq L\leqslant V$. Ортогональным дополнением к $L$ относительно билинейной формы $b(x,y)$ называется: 
    $$L^{\perp}:=\{y\in V \ | \ b(x,y)=0, \ \forall x\in L\}$$
\end{definition}
\begin{remark}
    Запись $x\perp y$ означает, что $b(x,y)=0$.
\end{remark}
\begin{definition}
    $V^{\perp}=\{y\in V \ | \ b(x,y)=0, \ \forall x\in V\}$ - ядро формы.
\end{definition}
\begin{definition}
    Билинейная функция $b(x,y)$ называется невырожденной, если: 
    $$Kerb=V^{\perp}=\{0\}$$
\end{definition}
\begin{exercise}
    $b(x,y)$ - невырожденная функция $\Longleftrightarrow$ $\det B\neq0$.
\end{exercise}
\subsection{Квадратичные формы}
\begin{definition}
    Квадратичная форма в некотором базисе называется диагональной, если в этом базисе: 
    $$k(x_1, \ldots, x_n)=\sum\limits_{i=1}^n\alpha_ix_i^2, \text{ где } \alpha_i\in\F$$
\end{definition}
\begin{theorem}
    В конечномерном пространстве $V$ ($\text{char}\F\neq2$) $\exists$ базис, в котором эта форма диагональна.
\end{theorem}
\begin{proof} (Алгоритм Лагранжа - метод выделения полных квадратов)\\
    По формуле \textbf{(1)}: 
    $$k(x)=\sum\limits_{i=1}^nb_{ii}x_i^2+\sum\limits_{i<j}b_{ij}x_ix_j$$
    \begin{enumerate}
        \item Основной случай: \\
        $\exists i$ : $b_{ii}\neq0$ $\Longrightarrow$ можно перенумеровать неизвестные $x_1$, $\ldots$, $x_n$, так что $b_{11}\neq0$. Выделим в $k(x)$ все одночлены, содержащие $x_1$:
        $$k(x)=\sum\limits_{i=1}^nb_{11}x_1^2+2x_1\sum\limits_{i=2}^nb_{1i}x_i+\widetilde{k}(x_2,\ldots, x_n)$$ 
        и дополним выражение до квадрата:
        \begin{multline*}
            k(x) = b_{11}(x_1^2+2x_1\sum\limits_{i=2}^n\frac{b_{1i}}{b_{11}}x_i+(\sum\limits_{i=2}^n\frac{b_{1i}}{b_{11}}x_i^2))-\frac{(\sum\limits_{i=2}^nb_{1i}x_i)^2}{b_{11}}+\widetilde{k} = \\
            =b_{11}(x_1+\sum\limits_{i=2}^n\frac{b_{1i}}{b_{11}}x_i)^2+k_2(x_2, \ldots,x_n)
        \end{multline*}
        Затем для формы $k_2(x_2,\ldots, x_n)=\sum\limits_{i=2}^nb_{ii}'x_i^2+\sum\limits_{2\leqslant i<j\leqslant n}b_{ij}'x_ix_j$ найдём коэффициент $b_{jj}'\neq0$ и выделим квадрат как на предыдущем шаге. На каждом шаге число переменных уменьшается на единицу, а значит, за конечное число шагов (а именно $\leqslant n-2$) форма приобретёт диагональный вид.
        \item Особый случай: \\
        $\forall i: \ b_{ii}=0$, но так как $k(x)\not\equiv0$ $\Longrightarrow$ $\exists$ индексы $i$ и $j$ такие, что $b_{ij}\neq0$, то есть в выражение $k(x_i, x_j)$ входит одночлен $2b_{ij}x_ix_j$.\\
        Пусть $x_i=x_i'+x_j'$ и $x_j=x_i'-x_j'$, тогда $x_ix_j = x_i'^2-x_j'^2$, то есть появился квадрат с коэффициентом, не равным нулю $\Longrightarrow$ можно перейти к общему случаю.
    \end{enumerate}
\end{proof}
\begin{remark}
    В благоприятном случае, когда на первом шаге коэффициент при $x_1$ не равен нулю, на втором шаге коэффициент при $x_2$ не равен нулю и т.д., матрица замены будет иметь вид:
    $$C_{e\rightarrow e'}^{-1}=\begin{pmatrix}
    1 & \frac{b_{12}}{b_{11}} & \ldots & \frac{b{1n}}{b_{11}}\\
    0 & 1 & \ldots & \frac{b_{1n}}{b_{22}}\\
    \vdots & \null & \ddots & \vdots\\
    0 & 0 & \ldots & 1
    \end{pmatrix}$$ 
    - матрица с 1 на диагонали $\Longrightarrow$ $|C_{e\rightarrow e'}^{-1}|=1\neq0$.
\end{remark}
\begin{definition}
    Форма $k(x_1,\ldots,x_n)$ называется канонической(нормальной), если:
    \begin{enumerate}
        \item (над $\R$) в диагональном виде $\forall \alpha_i$ принимает только значения: -1, 0, 1
        \item (над $\CC$) в диагональном виде $\forall \alpha_i$ принимает только значения: 0, 1
    \end{enumerate}
\end{definition}
\begin{example}\tab
    \begin{enumerate}
        \item Пусть $\F=\R$:
        $$k(x)=b_{11}x_1^2+b_{22}x_2^2+\ldots+b_{nn}x_n^2 = \alpha_1x_1^2+\alpha_2x_2^2+\ldots+\alpha_nx_n^2$$
        Если $rkB=r \Longrightarrow  k(x)=\alpha_1x_1^2+\alpha_2x_2^2+\ldots+\alpha_rx_r^2(\alpha_{r+1}=\ldots=\alpha_n=0)$.\\
        Если $\alpha_i>0$, то введём обозначение: 
        $$\widehat{x_i}=\sqrt{\alpha_i}x_i \Longrightarrow k=\widehat{x_1}^2+\ldots+\widehat{x_p}^2-\widehat{x_{p+1}}^2-\ldots-\widehat{x_r}^2$$
        где $p$ - количество коэффициентов $\alpha_i>0$.\\
        Если $\alpha_i < 0 \Longrightarrow  \widehat{x_i} = -\sqrt{\alpha_i}x_i$.
        \item Пусть $\F=\CC$: 
        $$\forall i=\overline{1,r}: \ \widehat{x_i}=\sqrt{\alpha_i}x_i \Longrightarrow k=\widehat{x_1}^2+\ldots+\widehat{x_r}$$
    \end{enumerate}
\end{example}
\begin{theorem} \textbf{единственности (закон инерции)} \\
    Для данной квадратичной формой над $\R$ числа $p$ и $q$ (количество +1 и \\количество -1 в каноническом виде) определены единственным образом
\end{theorem} 

    
