%chktex-file 1 %chktex-file 3 %chktex-file 8 %chktex-file 9 %chktex-file 10 %chktex-file 11 %chktex-file 12 %chktex-file 13 %chktex-file 16 %chktex-file 17 %chktex-file 18 %chktex-file 25 %chktex-file 26 %chktex-file 35 %chktex-file 36 %chktex-file 37 %chktex-file 40 %chktex-file 44 %chktex-file 45 %chktex-file 49
\section{Матрицы линейного отображения}
    Пусть: $\mathcal{E} = \{e_1,...,e_n\}$ - базис в $V_1$; \ $\mathcal{F} = \{f_1,...,f_m\}$ - базис в $V_2$
    \begin{multline*}
        \forall x \in V_1 : \ x = \sum \limits_{j=1}^n x_je_j \Longrightarrow \phi(x) = \sum \limits_{j=1}^n x_j \phi(e_j)= \\ 
        = \{\phi(e_j) = \sum \limits_{i=1}^m a_{ij}f_i\} = \sum \limits_{j=1}^n \sum \limits_{i=1}^m x_ja_{ij}f_i
    \end{multline*}
    \begin{definition}
        Назовем $A = (a_{ij}) = A_{\phi,e,f}$ - матрицей $\phi$ в базисах $\mathcal{E}$ и $\mathcal{F}$.\\ 
        Обозначается: $Y_f = A_{\phi,e,f} \cdot X_e$ \ (где $Y$ - столбец координат $\phi(x)$).    
    \end{definition} 
    \begin{remark}
        Для линейного оператора $\phi: \ V \to V, \ A_{\phi,e}\equiv A_{\phi,e,e}$ 
    \end{remark}
    % Пусть $x = \sum \limits_{j=1}^nx_ie_i, \phi: \ V_1 \to V_2$ - линейное отображение. Тогда $y(x) = \sum \limits_{j=1}^nx_i \phi(e_i), \phi(e_i) = \sum \limits_{i=1}^ma_{ij}f_i$ и $A_{\phi, e,f} = (a_{ij})$ - начало 6 лекции
    \begin{algorithm} 
        Вычисление $\text{Ker}\ \phi$ и $\text{Im}\ \phi$ с помощью матрицы $A_\phi :$
        \begin{enumerate}
            \item $\text{Ker} \hspace{0.09cm} \phi = \{x = \mathcal{E} \cdot x_\mathcal{E} \ : \ A_\phi \cdot x_\mathcal{E} = 0\}; \ \dim (\text{Ker} \hspace{0.09cm} \phi) = n - \text{rk} A_\phi$
            \item $\text{Im} \hspace{0.09cm} \phi = \langle \phi(e_1),...,\phi(e_n) \rangle = \{y = f \cdot Y_f \ : \ Y_f = A_\phi \cdot x_\mathcal{E}\}$ \\
            $Y \in \text{Im} \hspace{0.09cm} \phi \Longleftrightarrow $ СЛУ $A_\phi \cdot x_\mathcal{E} = Y$ совместна $\Longrightarrow \dim (\text{Im} \hspace{0.09cm} \phi) = \text{rk} A_\phi$ \\ (т.е. не зависит от базиса);
            \item $\dim (\text{Im}\hspace{0.09cm} \phi) + \dim (\text{Ker}\hspace{0.09cm} \phi) = \dim V_1$   
        \end{enumerate}
    \end{algorithm}
    \subsection{Изменение матрицы линейного отображения при замене координат}
    \begin{subtheorem}
        Пусть $\mathcal{E} = (e_1,...,e_n)$ -  старый, а \ $\mathcal{E}' = (e_1',...,e_n')$ - новый базисы в $V_1$ и $\mathcal{F} = (f_1,...,f_n)$ -  старый , а \ $\mathcal{F}' = (f_1',...,f_n')$ - новый базисы в $V_2$,
        $C$ - матрица перехода из $\mathcal{E}$ в $\mathcal{E'}$, а $D$ - матрица перехода из $\mathcal{F}$ в $\mathcal{F'}$. Тогда:
        $$A_{\phi, \mathcal{E}', \mathcal{F}'} = D^{-1}\cdot A_{\phi, \mathcal{E}, \mathcal{F}} \cdot C$$ 
    \end{subtheorem}
    \begin{proof}
        Воспользуемся формулами связи координат векторов:
        $$\forall x \in V_1: \ x_\mathcal{E} = \undermat{C}{C_{\mathcal{E} \to \mathcal{E}'}}\cdot x_{\mathcal{E}'} \ \text{и} \ \ \forall y \in V_2: \ y_\mathcal{F} = \undermat{D}{C_{\mathcal{F} \to \mathcal{F}'}}\cdot y_{\mathcal{F}'} $$
        Тогда формулы имеют вид:
        $$\underset{(*)}{Y_\mathcal{F} = A_{\phi, \mathcal{E}, \mathcal{F}} \cdot x_\mathcal{E}} \ \text{и} \ \ \underset{(**)}{Y_{\mathcal{F}'} = A_{\phi, \mathcal{E}', \mathcal{F}'} \cdot x_{\mathcal{E}'}}$$  
        Умножим $(*)$ слева на $D^{-1}$, а также запишем выражение $x_\mathcal{E}$ через $x_{\mathcal{E}'}$:\\
        $\tab[0.6cm] \forall x_{\mathcal{E}'} \in F^n:$ 
        $$ \ D^{-1} \cdot Y_\mathcal{F} = D^{-1} \cdot (A_{\phi, \mathcal{E}, \mathcal{F}} \cdot C) \cdot x_{\mathcal{E}'} \Longleftrightarrow Y_{\mathcal{F}'} = (D^{-1} \cdot A_{\phi, \mathcal{E}, \mathcal{F}} \cdot C) \cdot x_{\mathcal{E}'}$$
        Возьмем $x_{\mathcal{E}'} = E_j, \ j = 1,...,n$ 
    \end{proof}
    \begin{remark}
        Для линейного оператора $\phi: \ V \to V:$ 
        $$A_{\phi, \mathcal{E}'} = C^{-1}_{\mathcal{E} \to \mathcal{E}'} \cdot A_{\phi, \mathcal{E}} \cdot C_{\mathcal{E} \to \mathcal{E}'}$$
    \end{remark}
    \begin{consequense}\tab
        \begin{enumerate}
            \item Для любого линейного отображения ранг его матрицы инвариантен при замене базиса 
            $$\text{rk} \hspace{0.09cm} A_{\phi, \mathcal{E}', \mathcal{F}'} = \text{rk} \hspace{0.09cm} A_{\phi, \mathcal{E}, \mathcal{F}};$$
            \item Для любого линейного оператора оперделитель и след его матрицы инвариантны при замене базиса
            $$\det (A_{\phi, \mathcal{E}'}) = \det(A_{\phi, \mathcal{E}})$$
            $$\text{tr} \hspace{0.09cm} (A_{\phi, \mathcal{E}'}) = \text{tr} \hspace{0.09cm} (A_{\phi, \mathcal{E}})$$
        \end{enumerate}
    \end{consequense} 
    \begin{proof} \tab 
        \begin{enumerate}
            \item Матрицы $C$ и $D$ невырождены, значит достаточно доказать, что \\
            $\text{rk}\hspace{0.09cm} A=\text{rk}\hspace{0.09cm} (AC)$, где $C$ - невыроджена.
            $$\begin{cases}
                B = A \cdot C \Longrightarrow \text{rk} \hspace{0.09cm} B \leq \text{rk} \hspace{0.09cm} A\\
                A = (A \cdot C) \cdot C^{-1} \Longrightarrow \text{rk} \hspace{0.09cm} A \leq \text{rk} \hspace{0.09cm} (AC)
            \end{cases} \Longrightarrow \undermat{\text{rk} \hspace{0.09cm} (AC) = \text{rk} \hspace{0.09cm} A}{\text{rk} \hspace{0.09cm} (AC)\leq \text{rk} \hspace{0.09cm} A \leq \text{rk} \hspace{0.09cm} (AC)}$$
            \item $\text{det}\hspace{0.09cm} (C^{-1}AC)=\text{det}\hspace{0.09cm} C^{-1}\cdot \text{det}\hspace{0.09cm} A\cdot \text{det}\hspace{0.09cm} C=\text{det}\hspace{0.09cm} A$
            \item $\text{tr} \hspace{0.09cm} (AC) = \text{tr} \hspace{0.09cm}(CA) \Longrightarrow \text{tr} \hspace{0.09cm}\left[C^{-1} \cdot (AC)\right] = \text{tr} \hspace{0.09cm}\left[(AC)\cdot C^{-1}\right] = \text{tr} \hspace{0.09cm}A$    
        \end{enumerate}
    \end{proof}  
    \begin{theorem}
        Пусть $a_1,...,a_n$ - ЛНЗ векторы в $V_1 \ (\dim V_1 = n), \ b_1,...,b_n$ - случайные векторы в $V_2\ (\dim V_2 = m)$. Тогда $\exists !$ линейное отображение $\phi: \ V_1 \to V_2: \phi(a_j) = b_j, \ j = 1,...,n$     
    \end{theorem}
    \begin{proof} $\\$ 
        Пусть в некотором базисе $\mathcal{E}$ пространства $V_1$ вектор $a_j \sim a_j^\uparrow$ - столбец координат, в базисе $f$ пространства $V_2$ вектор $b_j \sim b_j^\uparrow$\\
        По условию, $\forall j = 1,...,n: \ A_\phi \cdot a_j^\uparrow = b_j^\uparrow \Longrightarrow A_\phi(a_1^\uparrow,...,a_n^\uparrow) = (b_1^\uparrow,...,b_n^\uparrow)$ или $A_\phi \cdot A = B$, где $A_\phi$ - искомая матрица.\\
        Отсюда получаем, что $A_\phi = B \cdot A^{-1}$ (т.к. $a_1,...,a_n$ ЛНЗ).\vspace{0.5cm}
        $$\begin{pmatrix}
            A \\ \hline B
        \end{pmatrix} \xrightarrow[\text{строк}]{\text{ЭП}} 
        \begin{pmatrix}
            E \\ \hline  A_\phi
        \end{pmatrix}, \ 
        \begin{pmatrix}
            A \\ \hline  B
        \end{pmatrix} \to 
        \begin{pmatrix}
            A \\ \hline  B
        \end{pmatrix} \cdot C_{\text{эл}} = 
        \begin{pmatrix}
            AC \\ \hline  BC
        \end{pmatrix}$$ \vspace{0.5cm}
        Если $AC = E$, то $C = A^{-1}$ и $BC = BA^{-1} = A_\phi$    
    \end{proof} 
    \begin{theorem}
        Если $\dim V_1 < \infty,  \ \phi: \ V_1 \to V_2$ - линейное отображение, то 
        $$\text{Im} \hspace{0.09cm} \phi \cong V_1 / \text{Ker} \hspace{0.09cm}  \phi$$  
    \end{theorem} 
    \begin{proof}
        Базис ядра дополним до базиса пространства $V_1$ векторами $e_1,...,e_s$. Тогда любой $v \in V_1$ можно записать в виде: $$v = \sum \limits_{i=1}^sx_ie_i + u, \text{где } u \in \text{Ker} \hspace{0.09cm}  \phi$$ 
        По этому в факторпространстве базис составляет классы $\overline{v} + u = \sum \limits_{i=1}^sx_i \overline{e_i}$\\
        Рассмотрим отношение $\overline{\phi}: \ V_1/u \to V_2$, где $\overline{\phi}(\overline{v}) = \overline{\phi}(v+u) := \phi(v)$\\
        Отсюда $w = \overline{\phi}(\overline{v})$. Получаем, что $\phi$ - cюръективное линейное отображение \\
        (т.к. $\forall w \in V_2 \ \exists \ v \in V_1: \ \phi(v) = w$). Также $\text{Ker} \hspace{0.09cm}  \overline{\phi} = \{0\} = \{\text{Ker} \hspace{0.09cm} \phi\}$, потому что если $\overline{\phi}(\overline{v}) = 0$, то $\phi(v) = 0$, т.е. $v \in \text{Ker} \hspace{0.09cm} \phi = u \Longrightarrow v \in U \Longrightarrow \overline{v} = u = \{0\}$     
    \end{proof}
    