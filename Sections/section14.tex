%chktex-file 1 %chktex-file 3 %chktex-file 8 %chktex-file 9 %chktex-file 10 %chktex-file 11 %chktex-file 12 %chktex-file 13 %chktex-file 16 %chktex-file 17 %chktex-file 18 %chktex-file 25 %chktex-file 26 %chktex-file 35 %chktex-file 36 %chktex-file 37 %chktex-file 40 %chktex-file 44 %chktex-file 45 %chktex-file 49
\section{Корневые подпространства}
    $\phi:\ V \to V$ - линейный оператор над $\F, \ \dim V = n$\\
    Все корни характеристического многочлена для $\phi$ принадлежат $F$ так, что:
    $$\chi_\phi(\lambda) = (-1)^n(\lambda-\lambda_1)^{k_1} \cdots (\lambda - \lambda_p)^{k_p}\cdots (\lambda-\lambda_s)^{k_s}, \ \forall i \neq j: \  \lambda_i \neq \lambda_j, \ \sum \limits_{i=1}^sk_i = n$$
    Рассмотрим: 
    $$\frac{1}{\chi_\phi(\lambda)} = \frac{f_1(\lambda)}{(\lambda-\lambda_1)^{k_1}} + ... + \frac{f_s(\lambda)}{(\lambda-\lambda_s)^{k_s}} \ \ | \cdot \chi_\phi(\lambda)$$ $$\Longrightarrow 1 = f_1 (\lambda) \prod\limits_{i\neq 1}(\lambda-\lambda_i)^{k_i} + ... + f_s (\lambda) \prod\limits_{i\neq s}(\lambda-\lambda_i)^{k_i}$$
    $$1= q_1(\lambda) + ... + q_s(\lambda) \Longrightarrow  \text{id} = q_1(\phi) + ... + q_s(\phi) = Q_1 + ... + Q_s$$
    $$\forall x \in V: \ x=Q_1(x) + ... + Q_s(x) \Longrightarrow V = \text{Im} (Q_1) + ... + \text{Im} (Q_s)$$
    Обратим внимание, что:
    $$\forall i \neq j: \ Q_iQ_j = Q_jQ_i = 0$$
    Т.к. в $q_i(\lambda)q_j(\lambda)$ входят все множители, входящие в разложение $\chi_\phi(\lambda) \Longrightarrow $ по теореме Гамильтона-Кэли: 
    $$q_i(\phi)q_j(\phi) = 0$$
    Умножим равенство $\text{id} = Q_1+...+Q_i+...+Q_s$ на $Q_i:$ $$\Longrightarrow Q_i\text{id} = Q_i = Q_iQ_1+...+Q_iQ_i+...+Q_iQ_s = Q_i^2\Longrightarrow Q_i^2 = Q_i$$
    \begin{definition}
        $Q_i^2 = Q_i$ - идемпотентный оператор.
    \end{definition}  
    
    Введем обозначение $K_i = \text{Im}Q_i$
    
    \begin{subtheorem}
        $V = K_1 \oplus ... \oplus K_s$
    \end{subtheorem}
    \begin{proof}
        Пусть $x = y_1 + ... + y_s, \ y_i = Q_i(x_i)$. Тогда:
        $$Q_i(x) = Q_i(Q_1(x_1)) + ... + Q_s(Q_i(x_s)) = Q_i(Q_i(x_i)) = Q_i(x_i) = y_i$$
        Отсюда разложение любого вектора из $V$ в сумму векторов из $K_1,...,K_s$ единственно, т.е. $V = K_1 \oplus ... \oplus K_s$.
    \end{proof}
    \begin{definition}
        Подпространство $K_i = \text{Im} Q_i$ назовем корневым подпространством, отвечающим корню $\lambda_i$.
    \end{definition}
    \begin{remark}
        $q_i(\lambda) = \frac{f_i(\lambda)\cdot \chi_\phi(\lambda)}{(\lambda-\lambda_i)^{k_i}} = f_i(\lambda)\prod\limits_{j \neq i}(\lambda-\lambda_j)^{k_j}; \ \ Q_i = q_i(\phi); \ \ K_i = \textup{Im}Q_i$.
    \end{remark} 
    \begin{subtheorem}\tab
        \begin{enumerate}
            \item Корневые подпространства инвариантны
            \item $K_i = \text{Ker} (\phi - \lambda_i\cdot \text{id})^{k_i}, \ 1\leq i \leq s$ 
        \end{enumerate}
    \end{subtheorem}
    \begin{proof}\tab 
        \begin{enumerate}
            \item Докажем, что для линейного оператора $\phi$ и многочлена $q(\lambda)$ подпространство $q(\phi)(V)$ инвариантно:
            $$q(\lambda) = a_0+ a_1 \lambda + ... + a_m \lambda^m, \ \ q(\phi) = a_0+ a_1 \phi + ... + a_m \phi^m$$
            Возьмем $v \in \text{Im}\hspace{0.07cm}q(\phi)(V) \Longrightarrow$ 
            \[\exists \ u \in V: \ v = q(\phi)(u) \Longrightarrow \phi(v) = (\phi \cdot q(\phi))(u) = q(\phi)(\phi(u)) \in \text{Im}\hspace{0.07cm}q(u) \]
            так как оператор $\phi$ и любой $q(\phi)$ перестановочны.\\
            Так как $K_i = Q_i(V) = q_i(\phi)(V)$, из доказаноого выше следует, что $K_i$ инвариантно.
            \item $\forall x_i \in \text{Im}\hspace{0.07cm}Q_i \Longrightarrow x_i = Q_i(u_i)$
            $$(\phi-\lambda_i \cdot \text{id})^{k_i}\hspace{0.07cm}(x_i) = f_i(\phi) \cdot \underbrace{(\phi-\lambda_i \cdot \text{id})^{k_i} \cdot \prod\limits_{j \neq i}(\phi-\lambda_j E)^{k_j}}_{\chi_\phi(\phi)}(u_i) = 0$$
            $\Longrightarrow K_i \subseteq \text{Ker}(\phi-\lambda_i \cdot \text{id})^{k_i}$\\ 
            Обратно: пусть $y_i \in \text{Ker}(\phi-\lambda_i \cdot \text{id})^{k_i}$. Знаем, что $y_i = Q_1(y_i) + ... + Q_s(y_i)$, причём в $Q_j$ при $j \neq i$ содержится множитель $(\phi-\lambda_i \cdot \text{id})^{k_i}$. Отсюда $Q_j(y_i) = 0$ при $j \neq i$, т.е. $y_i = Q_i(y_i) \Rightarrow y_i \in K_i \Rightarrow \text{Ker}(\phi-\lambda_i \cdot \text{id})^{k_i} \subseteq K_i$.
        \end{enumerate}
    \end{proof} 
    \begin{theorem}
        Размерность $K_i$ равна алгебраической кратности корня $\lambda_i$.
    \end{theorem}
    \begin{proof}
        Рассмотрим ограничение оператора $\phi - \lambda_i\cdot\text{id}$ на $K_i$. Так как полученный оператор нильпотентный (из предыдущей теоремы), его единственное собственное значение равно $0$, т.е. оператор $\phi$ в ограничении на $K_i$ имеет единственное собственное значение $\lambda_i$, причём его алгебраическая кратность для ограничения равна размерности $K_i$.\\
        Выберем базис в $K_i$, дополним его до базиса $V$ и рассмотрим матрицу оператора в нём. Из инвариантности $K_i$ она будет иметь вид 
        $$\begin{pmatrix}
            B & \vline & D \\ \hline 0 & \vline & C
        \end{pmatrix}$$
        где $B$ - матрица $\phi \hspace{0.05cm} |_{K_i}$. Из её характеристического многочлена очевидно, что алгебраическая кратность $\lambda_i$ для ограничения не может превосходить алгебраической кратности $\lambda_i$ для всего оператора. Значит, $\dim K_i$ не превосходит алгебраической кратности $\lambda_i$.\\
        Осталось заметить, что $\dim V$ равна сумме алгебраических кратностей всех собственных значений и $V = K_1 \oplus ... \oplus K_s \Rightarrow \dim V = \dim K_1 + ... + \dim K_s$. Значит, $\dim K_i$ равна алг. кратности $\lambda_i$.      
    \end{proof}