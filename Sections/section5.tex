%chktex-file 1 %chktex-file 3 %chktex-file 8 %chktex-file 9 %chktex-file 10 %chktex-file 11 %chktex-file 12 %chktex-file 13 %chktex-file 16 %chktex-file 17 %chktex-file 18 %chktex-file 25 %chktex-file 26 %chktex-file 35 %chktex-file 36 %chktex-file 37 %chktex-file 40 %chktex-file 44 %chktex-file 45 %chktex-file 49\section{Линейные отображения и функции}
    Пусть $V_1, V_2$ - векторные пространства над полем $\F$.
    \begin{definition}
        Отображение $\phi: V_1 \rightarrow V_2$ называется линейным отображением $V_1$ в $V_2$, если:
        \begin{enumerate}
            \item $\forall v_1, v_1'\in V_1 \ : \ \phi(v_1 + v_1') = \phi(v_1) + \phi(v_1')$;
            \item $\forall v \in V_1, \lambda \in \F \ : \ \phi(\lambda v) = \lambda\phi(v)$;
        \end{enumerate}
        Из курса $\textup{I}$ семестра известно, что $\phi (0_{v_1}) = 0_{v_2}$  \\
        Обозначается: $\textup{Ker}(\phi)$ - ядро $\phi$
        $$\textup{Ker}(\phi) = \{v \in V_1 \ | \ \phi(v) = 0_{v_2}\}, \ \text{Im}(\phi) = \phi(V_1)$$ 
    \end{definition}

