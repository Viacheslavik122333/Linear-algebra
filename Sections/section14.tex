\section{хуйхуйхуй}
\section{Корневые подпространства}
    $\phi:\ V \to V$ - линейный оператор над $\F, \ \dim V = n$\\
    Все корни характеристического многочлена для $\phi$ принадлежат $F$ так, что:
    $$\chi_\phi(\lambda) = (-1)^n(\lambda-\lambda_1)^{k_1} \cdots (\psi - \psi_s), \ (\lambda-\lambda_)^{k_s}, \ \forall i \neq j: \  \lambda_i \neq \lambda_j, \ \sum \limits_{i=1}^sk_i = n$$
    Рассмотрим: 
    $$\frac{1}{\chi_\phi(\lambda)} = \frac{f_1(\lambda)}{(\lambda-\lambda_1)^{k_1}} + ... + \frac{f_s(\lambda)}{(\lambda-\lambda_s)^{k_s}} \ \ | \cdot \chi_\phi(\lambda)$$ $$\Longrightarrow 1 = f_1 (\lambda) \prod\limits_{i\neq 1}(\lambda-\lambda_i)^{k_i} + ... + f_s (\lambda) \prod\limits_{i\neq s}(\lambda-\lambda_i)^{k_i}$$
    $$1= q_1(\lambda) + ... + q_s(\lambda) \Longrightarrow  \text{id} = q_1(\lambda) + ... + q_s(\lambda) = Q_1 + ... + Q_s$$
    $$\forall x \in V: \ Q_1(x) + ... + Q_s(x) \Longrightarrow V = \text{id} Q_1 + ... + Q_s$$
    Обратим внимание, что:
    $$\forall i \neq j: \ Q_iQ_j = Q_jQ_i = 0$$
    Т.к. в $q_i(\lambda)q_j(\lambda)$ входят все множители, входящие в разложение $\chi_\phi(\lambda) \Longrightarrow $ по теореме Гамильтона-Кэли: 
    $$q_i(\phi)q_j(\phi) = 0$$
    Умножим равентсво $\text{id} = Q_1+...+Q_i+...+Q_s$ на $Q_i:$ $$\Longrightarrow \text{id}Q_i = Q_i = Q_iQ_1+...+Q_iQ_i+...=Q_iQ_s \Longrightarrow Q_i^2 = Q_i$$
    \begin{definition}
        $Q_i^2 = Q_i$ - идемпотентный оператор
    \end{definition}  
    Разложение $\forall$ вектора $x$ в сумму: $Q_1(x) + ... + Q_s(x)$ - единстванно\\
    Докажем равенство: $x = y_1+ ... + y_s$, надо доказать, что $y_i = x_i$
    Из равенства $x = Q_1(x_1) + ... + Q_s(x_s) \Longrightarrow Q_i(x) = Q_i(Q_i(x)) = Q_i(x_i) \Longrightarrow x_i = y_i$, где $y_i \in \text{Im}Q_i$
    
    Введем обозначение $K_i = \text{Im}Q_i$
    
    Докажем, что $V = K_1 \oplus ... \oplus K_s$
    
    \begin{definition}
        Подпространство $K_i = \text{Im} Q_i$ назовем корневым подпространством, отвечающим корню $\lambda_i$
    \end{definition}
    $q_i(\lambda) = \frac{f_i(\lambda)\cdot \chi_\phi(\lambda)}{(\lambda-\lambda_i)^{k_i}}$ 
    \begin{subtheorem}\tab
        \begin{enumerate}
            \item Корневые подпространства инвариантны
            \item $K_i = \text{Ker} (\phi - \lambda_i\cdot \text{id})^{k_i}, \ 1\leq i \leq s$ 
        \end{enumerate}
    \end{subtheorem}
    \begin{proof}\tab 
        \begin{enumerate}
            \item Для линейного оператора $\phi$ и линейного $q(\lambda)$ подпространство $q(\phi)(V)$ инвариантно
            $$q(\lambda) = a_0+ a_1 \lambda + ... + a_m \lambda^m, \ \ q(\phi) = a_0+ a_1 \phi + ... + a_m \phi^m$$
            Возьмем $v = \text{Im}q(v) \Longrightarrow \exists \ u \in V: \ v = q(\phi)(u) \Longrightarrow \phi(v) = (\phi \cdot q(\phi))(u) = q(\phi)(\phi(u)) \in \text{Im}q(u)$
            Оператор $\phi$ и любой $q(\phi)$ перестановочны
            \item $\forall x_i \in \text{Im}Q_i \Longrightarrow x_i = Q_i(u_i)$
            $$(\phi-\lambda_i \cdot \text{id})^{k_i} = \underbrace{(\phi-\lambda_i \cdot \text{id})^{k_i} \cdot \prod\limits_{j \neq i}(\phi-\lambda_j E)^{k_j}}_{\chi_\phi(\phi)}(u_i) = 0$$
            $\Longrightarrow K_i \subseteq \text{Ker}(\phi-\lambda_i \cdot \text{id})^{k_i}$     
        \end{enumerate}
    \end{proof} 