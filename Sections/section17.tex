%chktex-file 1 %chktex-file 3 %chktex-file 8 %chktex-file 9 %chktex-file 10 %chktex-file 11 %chktex-file 12 %chktex-file 13 %chktex-file 16 %chktex-file 17 %chktex-file 18 %chktex-file 25 %chktex-file 26 %chktex-file 35 %chktex-file 36 %chktex-file 37 %chktex-file 40 %chktex-file 44 %chktex-file 45 %chktex-file 49
\section{Евклидовы пространства и их обобщения}
\subsection{Основные понятия и утверждения}
Основное поле - $\F = \R$.
\begin{definition}
    Вещественное конечномерное векторное пространство $\mathcal{E}$ называется евклидовым, если на $\mathcal{E}$ задано скалярное произведение $(x,y)$.
\end{definition}
\begin{definition}
    Скалярное произведение $(x,y)$ - симметрическая билинейная функция такая, что соответственная квадратичная форма $(x,x)$ положительно определена.
\end{definition} 
\begin{definition}
    Длина (норма) вектора $x\in\mathcal{E}$: $|x| = \sqrt{(x,x)}$.
\end{definition}
\begin{theorem} \textbf{Неравенство Коши-Буняковского-Шварца}\\
    $\forall \ x,y\in\mathcal{E}: \ |(x,y)|\leqslant|x|\cdot|y|$, причём равенство выполнено $\Leftrightarrow x \parallel y$ (либо $x = 0$ или $y = 0$, либо $y = \lambda x$).
\end{theorem}
\begin{proof}
    Рассмотрим функцию $f(t) = (tx-y, tx-y) = t^2(x,x) -2t(x,y) + (y,y) \geqslant 0$. Это квадратичная функция относительно $t$:
    $$f(t)\geqslant 0 \Leftrightarrow \frac{\mathcal{D}}{4} = (x,y)^2 - (x,x)(y,y) \leqslant 0 \Rightarrow (x,y) \leq \sqrt{(x,x)(y,y)} = |x|\cdot|y|$$
    Равенство выполнено $\Leftrightarrow (tx-y, tx-y) = 0 \Rightarrow y = tx$.
\end{proof}
\begin{theorem} \textbf{Неравенство треугольника}\\
    $\forall x, y \in \mathcal{E} : \ |x+y| \leq |x| + |y|$ \ (равенство выполнено $\Longleftrightarrow x \uparrow \uparrow y$ ) 
\end{theorem}
\begin{proof}
    $$(x+y, x+y) = (x, x) + 2(x, y) + (y, y) \leq |x|^2 + 2|x||y| = (|x|+|y|)^2$$
    $$|x+y|^2 \Longleftrightarrow |x+y| \leq |x| + |y|$$  
\end{proof}
Координатная запись: пусть в $V$ фиксированный базис $e_1,...,e_n$, то: 
$$(x,y) = (\sum \limits_{i=1}^nx_ie_i), \sum \limits_{j=1}^ny_je_j = \sum \limits_{i,j=1}^nx_iy_j(e_i,e_j)$$
\begin{definition}
    $G_e = ((e_i,e_j))$ - матрица Грама базиса $e$
    $$G_e^T = G_e$$ 
    Т.к. $(x,x)$ - положительно определенная квадратичная форма, то матрица: 
    $$G_e = (g_{ij})$$ 
    может служить матрицей Грама $\Longleftrightarrow \triangle_1 >0,...,\triangle_n > 0$ \\
    В частности: \ $\det G_e >0$ (определитель Грама)   
\end{definition}
\begin{center}
    \fbox{\text{ $(x,y) = X^TG_eY$ }}
\end{center}
\begin{definition}
    $$x \perp y \Longleftrightarrow (x,y) = 0$$ 
\end{definition}
\begin{definition}
    Базис $e_1,...,e_n$ называется ортогональным, если: 
    $$e_i \perp e_j \ \text{ при } i \neq j$$
\end{definition}
\begin{consequense}
    $e_1,...,e_n$ - ортогональный базис, если $(e_i,e_j) = \delta_{ij}$.  
\end{consequense}
\begin{consequense}
    Если базис ортогональный, то $G = E$ и $(x,y) = \sum \limits_{i=1}^nx_iy_i$.
\end{consequense}
\begin{theorem} Пусть $e' = eC_{e\to e'}$ - новый базис. Тогда: 
    \begin{enumerate}
        \item Если $e$ и $e'$ ортогональные, то $C_{e \to e'}$ ортогональна;
        \item Если $e$ ортогональный базис и $C_{e \to e'}$ ортогональная матрица $\Longrightarrow e' = eC$ - ортогональный базис.
    \end{enumerate}
\end{theorem}
\begin{remark}
    $C$ - ортогональная, если $C^TC = E$  
\end{remark} 
\begin{proof} \tab
    \begin{enumerate}
        \item По определению матрицы перехода $C_{e \to e'} = \begin{pmatrix}
            e_1^\uparrow & \cdots & e_n^\uparrow
        \end{pmatrix}$\\
        $C^T_{e\to e'} = \begin{pmatrix}
            e_1^\rightarrow  \\ \vdots \\ e_n^\rightarrow 
        \end{pmatrix}$ Обозначим $d_{ij}$ - $(ij)$элемент матрицы $C^TC:$ 
        $$d_{ij} = e_i'^{\rightarrow} \cdot e_j'^{\uparrow} = (e'_i,e_j') = \delta_{ij}$$
        т.к. базис $e$  ортогональный $\Longrightarrow d_{ij} = \delta_{ij} \Longrightarrow C^TC = E$
        \item Рассмотрим $e' = eC_{e\to e'}$, тогда $e_j^{\uparrow}$ - это $j$ столбец матрицы $C_{e\to e'}$\\
        По условию $C^TC = E \Longleftrightarrow e_i'^{\rightarrow} \cdot e_j'^{\uparrow} = \delta_{ij} = (e'_i,e_j')$
    \end{enumerate}    
\end{proof}
\begin{lemma}
    Если $a_1,...,a_m \in \mathcal{E}$ - ортогональная система векторов\\ 
    \tab[13cm] $\Longrightarrow  a_1,...,a_m$ ЛНЗ  
\end{lemma}
Т.о. $\forall x \in \mathcal{E}$ единственным образом разлагается в сумму $x = x_{\shortparallel} + x_{\perp}$
\begin{center}
    \includegraphics[width=10cm]{image/Asymptote/3/linal-3-1.pdf}
\end{center}
$x_{\shortparallel} \in U, \ \ x_{\shortparallel}$ - ортогональная проекция вектора $x$ на $U$\\
$x_{\perp} \in U^{\perp}, \ x_{\perp}$ - ортогональная составляющая $x$ относительно $U$
\begin{example1}\tab
    \begin{center}
        \includegraphics[width=8cm]{image/Asymptote/4/linal-4-1.pdf}
        $U = \langle 1,t,...,t^{n-1} \rangle$ 
    \end{center}
    Надо подобрать такой многочлен $p(t) \in U$, чтобы: 
    $$\parallel f(t) - p(t)\parallel \ = \min$$
    Где $p(t) = f(t)$ - псевдорешение  
\end{example1}
\subsection*{Как конкретно находить такое разложение?}
\begin{itemize}
    \item[\textbf{1 способ:}] Выбрать ортогональный базис в $U$ и дополнить его до ортогонального базиса в $\mathcal{E}$\\
    Тогда:
    $$x = \underbrace{\sum \limits_{i=1}^m(x_i,e_i)e_i}_{x_\shortparallel}  + \underbrace{\sum \limits_{i=m+1}^n(x_i,e_i)e_i}_{x_\perp} = x_\shortparallel + x_\perp$$
    \item[\textbf{2 способ:}] Выбрать в $U$ произвольный базис $a_1,...,a_m$ и искать разложение в виде:
    $$x = \sum \limits_{i=1}^m \alpha_ia_i + x_\perp \ \ | \cdot a_j \Longrightarrow (x_i,a_j) = \sum \limits_{i=1}^m \alpha(a_i,a_j) + \underbrace{(x_\perp + a_j)}_{=0} $$
    Не однородная СЛУ с неизвестными $\alpha_i$, основная матрица: 
    $$((a_i,a_j)) = G_{\{a_1,...,a_m\}}$$
    где $\det G \neq 0 \Longrightarrow$ по теореме Крамера $\exists ! \ \alpha_1,...,a_m \Longrightarrow \exists ! \ x_\shortparallel \Longrightarrow x_\perp = x - x_\shortparallel$   
\end{itemize}
\begin{properties} \textbf{операций ортогонального дополнения} 
    \begin{enumerate}
        \item $(U^\perp)^\perp = U$
        \item $(U_1 + U_2)^\perp = U_1^\perp \cap U_2^\perp$
        \item $(U_1 \cap U_2)^\perp = U_1^\perp + U_2^\perp$
    \end{enumerate}
\end{properties}
\begin{proof}\tab
    \begin{enumerate}
        \item Пусть $x\in U,\ y\in U^\perp$, тогда: 
        $$(y,x)=0\ \forall y\in U^\perp \Rightarrow x\in (U^\perp)^\perp \Longrightarrow  U\subseteq (U^\perp)^\perp$$ 
        Причем: 
        $$\dim{(U^\perp)^\perp}=n-\dim{U^\perp}=n-(n-\dim{U})=\dim{U}\Longrightarrow  U=(U^\perp)^\perp$$

        \item Пусть $v\in U_1^\perp\cap U_2^\perp \Longrightarrow  v\perp U_1$ и $v\perp U_2 \Longrightarrow  \forall x=u_1+u_2$:
        $$(v,x)=(v,u_1)+(v,u_2)=0 \Longrightarrow  v\in (U_1+U_2)^\perp \Longrightarrow  U_1^\perp \cap U_2^\perp \subseteq (U_1+U_2)^\perp$$
        Если $w\in (U_1+U_2)^\perp$, то $\forall u_1\in U_1, \forall u_2\in U_2 :\ (w, u_1+u_2)=0$\\
        В частности: 
        $$\begin{cases}
            \forall u_1\in U_1: \ (w, u_1)=0 \ \Longrightarrow  w\in U_1^\perp\\
            \forall u_2\in U_2: \ (w, u_2)=0 \ \Longrightarrow  w\in U_2^\perp
        \end{cases} \Longrightarrow  w\in U_1^\perp\cap U_2^\perp$$
        То есть $(U_1+U_2)^\perp \subseteq U_1^\perp \cap U_2^\perp \Longrightarrow $ имеет место равенство:
        $$(U_1+U_2)^\perp=U_1^\perp\cap U_2^\perp$$

        \item Возьмем $(U_1^\perp+U_2^\perp)=(U_1^\perp)^\perp \cap (U_2^\perp)^\perp=U_1 \cap U_2$
        $$\Longrightarrow  ((U_1^\perp+U_2^\perp)^\perp)^\perp=(U_1\cap U_2)^\perp \Longrightarrow  U_1^\perp+U_2^\perp=(U_1 \cap U_2)^\perp$$
    \end{enumerate}
\end{proof}
\begin{subtheorem}
    Вектор наименьшей длины, соединяющий точку из подпространства $U$ с концом вектора $x$ - $x_{\perp}$. 
\end{subtheorem}
\begin{proof}
    Обозначим $x_{\parallel} = y, x_{\perp} = z$, а вектор из начала $x$ в произвольную точку $U$ - вектор $v$ (РИСУНОК). Докажем, что $|x-v|\geqslant |z|$, причём равенство достигается при $v = y$:
    $$x - v = x - y + y - v = z + y - v$$
    Т.к. $z \in U^{\perp}, (y - v) \in U$,
    $$z \perp (y-v) \Rightarrow |x-v|^2 = |z|^2 + |y-v| \geqslant |z|^2$$
    причём равенство при $|y - v| = 0 \Rightarrow y = v$.
\end{proof}
Это подтвержает осмысленность определения $\rho(x, U) = |x_{\perp}|$.
\begin{exercise}
    Докажите отсюда, что $\angle(x, v) \geqslant \angle(x, y)$.
\end{exercise}
\begin{definition}
    Углом между вектором и подпространством будем называть $\angle(x, v) = \angle(x, y)$
\end{definition}

Частный случай: $\dim U = n-1$ ("гиперплоскость"):\\
В ортонормированном базисе $U$ задаётся уравнением $a_1x_1 +...+a_nx_n = 0$, а ортогональное дополнение $U^{\perp} = \langle n = (a_1,...,a_n)\rangle$ ($n$ - вектор нормали). Тогда:
$$\rho(x, U) = |x_{\perp}| = \frac{V_{\langle e_1,...,e_{n-1},x\rangle}}{V_{\langle e_1,...,e_{n-1}\rangle}}$$
где $V_{\langle a_1,...,a_k\rangle}$ - объём параллелепипеда, натянутого на $a_1,...,a_k$.
\begin{definition}
    $n$-мерным параллелепипедом с рёбрами $e_1,...,e_n$ называется
    $$\Pi_{\langle e_1,...,e_n\rangle} = \{v = \lambda_1e_1 + ... + \lambda_{n}e_{n}, 0\leqslant\lambda_i\leqslant1\}$$
\end{definition}
\begin{definition}
    В общем случае объём параллелепипеда определяется рекурсивно:
    $$V_{\langle e_1,...,e_n\rangle} = V_{\langle e_1,...,e_{n-1}\rangle} \cdot |e_{n\perp}|,  \ \text{где $e_{n\perp}$ - проекция $e_n$ на $\langle e_1,...,e_{n-1}\rangle$}$$
    Заметим, что если $e_1,...,e_n$ попарно ортогональны, то $V_{\langle e_1,...,e_n\rangle} = |e_1|\cdot...\cdot|e_n|$.
\end{definition}

Объём не изменится, если к векторам применить процесс ортогонализации (с унитреугольной матрицей перехода).\\
Тогда $V_{\langle e'_1,...,e'_n\rangle} = |e'_1|\cdot...\cdot|e'_n| = \sqrt{|G_{\{ e'_1,...,e'_n\}}|}$\\
В ортогональном базисе $G_{\{ e'_1,...,e'_n\}} = \begin{pmatrix} |e'_1|^2 &\null&0 \\ \null&\ddots&\null \\ 0&\null&|e'_n|^2 \end{pmatrix} \Rightarrow \\ \Rightarrow |G_{\{ e'_1,...,e'_n\}}| = |e'_1|^2\cdot...\cdot|e'_n|^2$
$$G_{e'} = C^TG_eC \Rightarrow |G_{e'}| = |C|^2|G_e| = |G_e|$$
\begin{exercise}
    Доказать отсюда. что если $U = \langle e_1,...,e_{n-1}\rangle$, то
    $$\rho^2(x, U) = \frac{|G_{\{ e_1,...,e_n\}}|}{|G_{\{ e_1,...,e_{n-1}\}}|}$$
\end{exercise}
\subsection{Линейные операторы в евклидовом пространстве}
Пусть $\mathcal{E}$ - евклидово пр-во, $\phi: \mathcal{E} \rightarrow \mathcal{E}$ - лин. оператор в $\mathcal{E}$.
\begin{definition}
    \begin{enumerate}
        \item Оператор $\phi^{*}: \mathcal{E} \rightarrow \mathcal{E}$ - сопряжённый к $\phi$, если
        $$\forall x,y \in \mathcal{E}: \ (\phi(x), y) = (x, \phi^{*}(y)) \eqno{(1)}$$
        \item Оператор $\phi$ - самосопряжённый, если
        $$\phi^{*} = \phi \Rightarrow \forall x,y \in \mathcal{E}: \ (\phi(x), y) = (x, \phi(y)) \eqno{(2)}$$
        \item Оператор $\phi$ - ортогональный, если
        $$\forall x,y \in \mathcal{E}: \ (\phi(x), \phi(y)) = (x, y) \eqno{(3)}$$
        В частности, для ортогонального $\phi: \forall x \in \mathcal{E} \ \ |\phi(x)| = |x|$.
    \end{enumerate}
\end{definition}
\textbf{Условия (1)-(3) через матрицу Грама}\\
    Пусть в $\mathcal{E}$ зафиксирован базис $e = (e_1,...,e_n) \ (\dim \mathcal{E} = n)$. Пусть $x = eX,\\
    y = eY, G_e = ((e_i, e_j))$ - матрица Грама базиса $e$, $A_{\phi}$ - матрица $\phi$ в базисе $e$.\\
    (1): $\forall X, Y \in \R^n$:
    $$(A_{\phi}X)^TG_eY = X^TA_{\phi}^TG_eY = X^TG_eA_{\phi^{*}}Y \Rightarrow A_{\phi}^TG_e = G_eA_{\phi^{*}} \ \ (1')$$
    (2): В частности, 
    $$\phi^{*} = \phi \Leftrightarrow A_{\phi}^TG_e = G_eA_{\phi} (2')$$
    Если $e$ - ортонормированный, то $G_e = E$, и $A_{\phi}^T = A_{\phi}$, т.е. $A_{\phi}$ - симметрическая матрица.\\
    (3): $\phi$ - ортогональный $\Leftrightarrow \ \forall X, Y \in \R^n$ выполнено:
    $$(A_{\phi}X)^TG_e(A_{\phi}Y) = G_eY \Rightarrow A_{\phi}^TG_eA_{\phi} = G_e (3')$$
    Если $G_e = E$, то $A_{\phi}^TA_{\phi} = E$, т.е. $A_{\phi}$ - ортогональная матрица.
\begin{theorem}\textbf{Свойства сопряжённых операторов}
    \begin{enumerate}
        \item $(\phi^{*})^{*} = \phi$;
        \item $\textup{Ker}\phi^{*} = (\textup{Im}\phi)^{\perp}$
        \item $\textup{Ker}\phi = (\textup{Im}\phi^{*})^{\perp}$
    \end{enumerate}    
\end{theorem}
\begin{proof}
    \begin{enumerate}
        \item В ортонормированном базисе:
        $$A_{\phi^{*}} = A_{\phi}^T \Rightarrow A_{{\phi^{*}}^{*}} = (A_{\phi^{*}})^T = (A_{\phi}^T)^T = A_{\phi}$$
        Т.к. в фиксированном базисе имеется взаимно однозначное соответствие операторов и их матриц, $(\phi^{*})^{*} = \phi$.
        \item Сравним размерности:
        $$\dim \textup{Ker}\phi^{*} = n - \textup{rk}(A_{\phi^{*}}) = n - \textup{rk}(A_{\phi}^T) = n - \textup{rk}(A_{\phi})$$
        $$\dim \textup{Im}\phi = \textup{rk} A_{\phi} \Rightarrow \dim (\textup{Im}\phi)^{\perp} = \textup n - {rk} A_{\phi}$$
        Докажем, что $\textup{Im}\phi \subseteq (\textup{Ker}\phi^{*})^{\perp}$  (отсюда $(\textup{Im}\phi)^{\perp} \subseteq \textup{Ker}\phi^{*}$):\\
        Пусть $v \in \textup{Im}\phi \Rightarrow v = \phi(x)$, $y \in \textup{Ker}\phi^{*}$. Тогда:
        $$(v, y) = (\phi(x), y) = (x, \phi^{*}(y)) = (x, 0) = 0 \Rightarrow v \perp \textup{Ker}\phi^{*}$$
        Т.к. размерности равны и $(\textup{Im}\phi)^{\perp} \subseteq \textup{Ker}\phi^{*}$, то $\textup{Ker}\phi^{*} = (\textup{Im}\phi)^{\perp}$.
        \item Следует из (2) подстановкой $\phi^{*}$ вместо $\phi$.
    \end{enumerate}
\end{proof}
\begin{consequense}\textbf{Теорема Фредгольма}
    СЛУ $AX = b$ с квадратной матрицей $A$ порядка $n$ совместна $\Leftrightarrow$ для любого $Y$ - решения однородной сопряжённой системы - выполнено условие $Y \perp b$. 
\end{consequense}
\begin{proof}
    $AX = b$ совместна $\Leftrightarrow b \in \textup{Im}A$.\\
    $Y \in \textup{Ker}\phi^{*} = \textup{Ker}A^T$. Т.к. $\textup{Ker}\phi^{*} = (\textup{Im}A)^\perp$, то система совместна $\Leftrightarrow b \perp \textup{Ker}\phi^{*}$  
\end{proof}
\subsection{Самосопряжённые операторы}
\begin{theorem}
    Для любого самосопряжённого оператора $\phi: \mathcal{E} \rightarrow \mathcal{E}$ в $\mathcal{E}$ существует базис из собственных векторов этого оператора.
\end{theorem}